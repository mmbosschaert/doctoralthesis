\thispagestyle{plain}
\begin{center}
    \Large
    \textbf{\thesistitle}
        
    \vspace{0.4cm}
    \large
    \thesissubtitle
        
    \vspace{0.4cm}
    \textbf{Maikel M. Bosschaert}
       
    \vspace{0.9cm}
    \textbf{Abstract}
\end{center}
This thesis is concerned with higher-order asymptotics to the homoclinic orbit
near the generic and transcritical codimension two Bogdanov--Takens bifurcation
in infinite dimensional systems generated by delay differential equations (DDEs).
First we will obtain accurate homoclinic asymptotics in the normal form. Then we
will preform the parameter-dependent center manifold reduction near the generic
and transcritical Bogdanov--Takens points. To achieve this, we rigorously derive
a method to translate asymptotics of solutions in the normal form for a local
bifurcation, to asymptotics of solutions on the parameter-dependent center
manifold. In particular, we allow for a time-reparametrization in the homological
equation, enabling us to consider orbital normal forms. The use of orbital normal
forms turn out to be particularly useful when obtaining third-order homoclinic
asymptotics near the transcritical Bogdanov--Takens bifurcations. Indeed, we show
that, by using orbital normal forms, these asymptotics can be obtained through a
simple transformation from the generic case.

Additionally, a detailed comparison is provided between applying different
Bogdanov--Takens normal forms (smooth and orbital), different phase conditions,
and different perturbation methods (regular and Lindstedt-Poincar\'e) to
approximate the homoclinic solution near Bogdanov--Takens points. In particular,
we will show that higher-order time approximations to the nonlinear time
transformation in the Lindstedt-Poincar\'e method are essential. 

Next to the codimension two Bogdanov--Takens bifurcation points, we also
preform the parameter-dependent center manifold reduction near the generalized
Hopf (Bautin), fold-Hopf, Hopf-Hopf and transcritical-Hopf bifurcations in
delay differential equations (DDEs). This allows us to initialize the
continuation of codimension one equilibria and cycle bifurcations emanating
from these codimension two bifurcation points.

Furthermore, the known existence theorem of a smooth finite dimensional
parameter-dependent center manifold for delay differential equations is
generalized to allow for the equilibrium under consideration to vanish, as is
the case in the zero-Hopf and the generic Bogdanov--Takens bifurcation points.
The proof is given at the abstract semigroup level using the framework of
perturbation theory for dual semigroups.

The non-hyperbolic cycles and homoclinic asymptotics are implemented in
\DDEBIFTOOL to start numerical continuation of these homoclinic curves. The
homoclinic predictor in \MATCONT has been corrected as well. The effectiveness of
the new predictors are demonstrated on numerous examples. In-depth treatments of
the examples are also provided, as well as the \MATLAB, \PYTHON, and \JULIA
source code to reproduce the obtained results.

Finally, we present a novel phenomenon in the study of the renormalization group
(RG). Namely, we found Shilnikov homoclinic orbits in the RG flow of a quantum
field theory, proving the existence of chaotic RG-behavior in the vicinity of a
fixed point.
