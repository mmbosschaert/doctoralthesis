% \makeatletter
% \def\convertto#1#2{\strip@pt\dimexpr #2*65536/\number\dimexpr 1#1}
% \makeatother
% \convertto{cm}{\the\textwidth} \\
% \convertto{cm}{\the\textheight} \\
% \convertto{cm}{\the\topmargin} \\
% \the\textwidth \\
A dynamical system is a system whose position in a state space
$X$ evolves with time according to an evolution operator $\phi$. The state space $X$ may be finite,
for example the Euclidean space $\mathbb R^n$, or infinite, such as the space of
continuous functions on a finite interval. The time can be either discrete, whose
set of values is the set of integer numbers $\mathbb Z$, or continuous, whose set
of values is the set of real numbers $\mathbb R$. Although, any set $T$ with the
algebraic structure is possible in general. The evolution operator $\varphi$
allows us to determine all future states given an initial state from the state
space.

It is assumed that the evolution operator itself does not change with time.
That is, the result of time evolution will depend only on the initial position
of the system and on the length of the evolution, but not on the moment when
the state of the system was initially registered. Thus, if our system was
initially at a state $x \in X$, it will find itself after time $t$ at a new
state, which is uniquely determined by $x$ and $t$, and thus can be denoted by
$\phi^t x$. By fixing $t\in T$, we obtain a transformation 
\[
    \phi^t \colon X \rightarrow X,\  x \mapsto \phi^t x.
\]
There are two natural properties associated with the evolution operator
\begin{itemize}
    \item $\phi^0 = \id$,
    \item $\phi^{s+t} = \phi^s \circ \phi^t$.
\end{itemize}
The first property implies that the system does not change its state
``spontaneously''. The second property describes how the transformations for
different times are related to each other. Namely, the evolution of the state
$x$ for time $s + t$ can be obtained by first applying the transformation
$\phi^t$  to $x$ and then by applying $\phi^s$ to the new state $\phi^t(x)$.
In other words, the evolution operators $\phi^t$ form a semigroup. 

The evaluation operator, which may be implicitly defined, can be rather simple.
However, the long term behavior as time goes to infinity may be very difficult to
predict by just letting the system evolve. The main objective in the study of
dynamical systems is to understand the asymptotic behavior of the states. For a
more in depth review on dynamical systems we refer to \cite{Kuznetsov2004}, and
the references therein.

Generally, the evaluation operator doesn't only depend on the states in the
state space, but also on parameters. The latter are quantities that do not
change with time, but assume different values depending on the specifies of the
application at hand. Then the behavior of the system may change depending on
different values of these parameters. What we therefore like to know is where
do those changes happen and what happens in these transitions. The most amazing
and remarkable aspect of studying these transitions is that you begin to notice
that there are universal types of transitions that occur.

Consider for example the following ordinary differential equation
\[
    \begin{cases}
    \begin{aligned}
        \frac{d}{dt} x_1 &= \alpha x_1  - x_2 + \sigma x_1(x_1^2 + x_2^2), \\
        \frac{d}{dt} x_2 &= x_1 + \alpha  x_2 + \sigma x_2(x_1^2 + x_2^2).
    \end{aligned}
    \end{cases}
\]
Here the dependence of the variables $x_{1,2}$ on the time $t\in \mathbb R$ has
been suppressed for readability. The state space is $X$ is the Euclidean space
$\mathbb R^2$, $\sigma$ is a constant with value $\pm 1$, and $\alpha$ is a
parameter in the parameter space $\mathbb R$. The evaluation operator is defined
implicitly through the differential equation, although in this specific situation
the operator can be computed explicitly. Transforming the system in polar
coordinates $(\rho, \theta)$ yields
\[
    \begin{cases}
    \begin{aligned}
        \frac{d}{dt} \rho &= \rho(\alpha + \sigma \rho^2), \\
        \frac{d}{dt} \theta &= 1.
    \end{aligned}
    \end{cases}
\]
It is easily seen that for $\sigma=-1$
\begin{itemize}
    \item $\alpha \leq 0$ the systems has a stable focus, and for 
    \item $\alpha > 0$ the systems has a unique stable periodic orbit,
        while the focus has become unstable.
\end{itemize}
In the case that $\sigma=1$ we have that for
\begin{itemize}
    \item $\alpha < 0$ the system has a stable focus and a unique unstable
        periodic orbit, while for
    \item $\alpha \geq 0$ the focus has become
        unstable and the periodic orbit no longer exists.
\end{itemize}
In other words, there is a discrete structural change in the state space when
continuously changing the parameter $\alpha$ through $0$. We say that
$\alpha=0$ is a bifurcation point. The above system is the (truncated)
canonical form, or normal form, for this type of bifurcation to occur. It is
known as the Poincar\'e–Andronov–Hopf bifurcation named after Henri Poincar\'e,
Aleksandr Andronov, and Eberhard Hopf and is usually abbreviated to the Hopf
bifurcation. One can show that adding higher order terms to the system does
not change the local behavior near the bifurcation point. Furthermore, this
bifurcation may take place in a higher dimensional system on a two-dimensional
invariant manifold (the center manifold) and is characterized by the single
condition that the linearization at the critical parameter value has one pair
of complex eigenvalues while there are no other eigenvalues present on the
imaginary axis. 

The case in which $\sigma=-1$ is called supercritical, while the case in which
$\sigma=1$ is called subcritical. Thus, in order to know which type of Hopf
bifurcation occurs in a particular system, it is necessary to calculate the
sign of $\sigma$, also known as the (normalized) first Lyapunov coefficient. One way to
obtain this coefficient is by first calculating the reduced system on the
center manifold and then preforming a normal form procedure. However, in
\cite{Coullet1983competinginstabilities} a \emph{normalization technique} is
described in which these two steps are combined into one single procedure. These
expressions are independent of the (finite) dimension of the phase space and
they involve only critical eigenvectors of the Jacobian matrix and its
transpose as well as higher order derivatives of the right-hand side at the
critical equilibrium. These properties make them suitable for both symbolic and
numerical evaluation.

Since, for the Hopf bifurcation, there is only one (genericity) condition to be
satisfied, this bifurcation belongs to the class of codimension one equilibria
bifurcations. If there is an additional parameter to be tweaked, an extra
condition can be met, leading to codimension two bifurcations. There are in
total five generic codimension two bifurcations of equilibria
\begin{itemize}
    \item cusp,
    \item generalized Hopf (Bautin),
    \item fold-Hopf (zero-Hopf),
    \item Hopf-Hopf (double-Hopf),
    \item Bogdanov--Takens.
\end{itemize} 
As with the codimension one Hopf bifurcation, different kinds of dynamics are
present near these codimension two bifurcation points depending on certain
normal form coefficients of the corresponding normal forms. In
\cite{Kuznetsov1999} the normalization technique was applied to obtain
expressions for the critical normal form coefficients of all generic
codimension one and two bifurcations of equilibria in ODEs, also see \cite[\S
8.7]{Kuznetsov2004}. 

From these codimension two bifurcation points emanate codimension one curves.
Given a general system in which a codimension two bifurcation occurs, no exact
solutions can, generally, be obtained for these codimension one curves.
However, approximations to the codimension one curves in the
(parameter-dependent) normal forms may be available. Then, if there is a
transformation to the parameter-dependent center manifold near the
codimension two bifurcation point, one can use these asymptotics to start
the continuation of the codimension one curves with numerical software.
Indeed, in \cite{Kuznetsov2008} the normalization technique was applied to
parameter-dependent normal forms to start the continuation of non-hyperbolic
cycles emanating from generalized Hopf, fold-Hopf, and Hopf-Hopf bifurcation
points of ODEs. The resulting predictors were implemented in the freely
available software package \MATCONT \cite{matcont1}, a \MATLAB toolbox for
continuation and bifurcation analysis of finite dimensional dynamical systems.
This makes it possible to verify transversality conditions and to initialize
the continuation of the nonhyperbolic cycles mentioned above. A similar
switching problem for iterated maps, i.e., where the time $T$ is discrete in
the dynamical system, was solved earlier in \cite{Govaerts2007maps}.

Similarly, in \cite{Kuznetsov2014improved} the normalization technique was used
to the smooth parameter-dependent normal form of the codimension two
Bogdanov--Takens bifurcation in order to start the continuation of the
codimension one homoclinic bifurcation curve emanating from the singularity.
However, in \cite{Kuznetsov2014improved} the objective wasn't merely to obtain
a homoclinic predictor, but a higher-order homoclinic predictor, specifically,
of order two. 

We now turn our attention to infinite dimensional systems generated by delay
differential equations (DDEs). In these systems the assumption that all
occurrences of time are present simultaneously is lifted. The type of delay equations
we will be considering in this thesis are called classical DDEs. They are of the
following form
\begin{equation}
    \dot{x}(t)= F(x_t, \alpha), \qquad t \ge 0,
\end{equation}
where $F: X \times \RR^p \to \RR^n$ is $C^k$-smooth for some $k \ge 1$ with
$F(0,0) = 0$ and the state space is given by $X \DEF C([-h,0],\RR^n)$. Here for
each $t \ge 0$, the \emph{history function} $x_t : [-h,0] \to \RR^n$ defined by
\[
  x_t(\theta) \DEF x(t + \theta), \qquad \forall\,\theta \in [-h,0].
\]

The main achievements of this thesis can be split into two parts:
\begin{enumerate}[label=\Roman*.]
    \item Initialization of non-hyperbolic cycles emanating from generalized
        Hopf, fold-Hopf, transcrical-Hopf, and Hopf-Hopf bifurcation points in DDEs.
    \item Initialization of homoclinic orbits emanating from generic and transcritical
        codimension two Bogdanov--Takens bifurcation points in DDEs using higher order
        asymptotics.
\end{enumerate}

For the first part we will generalize the parameter-dependent center manifold
theorem provided in \cite{diekmann1995delay}. The second part turns out to be
much more difficult. The first attempt to provide asymptotic approximations to the
homoclinic bifurcation curve near a generic codimension two Bogdanov--Takens
bifurcation point in general $n$-dimensional systems was made
in~\cite{Beyn_1994}. By applying a singular rescaling to the (one of the
equivalent) parameter-dependent normal form on the center manifold, a perturbed
planar Hamiltonian system is obtained. The unperturbed Hamiltonian system
contains an explicit homoclinic solution. A first-order correction in
parameter-space can subsequently be obtained by reformulating the problem as a
branching problem in a suitable Banach space, see~\cite{Beyn_1994}.  Then, by
using the regular perturbation method, higher-order approximations to the
homoclinic bifurcation curve can be obtained. Unfortunately, in~\cite{Beyn_1994},
even the first-order correction in the phase-space was not derived.  Nonetheless,
it was proven that the obtained homoclinic predictor converges to the true
solution under the Newton iterations in the perturbed Hamiltonian systems.

In~\cite{Kuznetsov2014improved} the work continued by obtaining a second-order
correction in parameter \emph{and} phase-space to the homoclinic bifurcation
curve for the perturbed Hamiltonian system. However, a new problem was
overlooked.  The normal form used in~\cite{Kuznetsov2014improved} is a normal
form for $C^\infty$-equivalence (also called \emph{smooth orbital equivalence}),
i.e., besides a $C^\infty$-coordinate change, also a time-reparametrization must
be taken into account, which was not the case in~\cite{Kuznetsov2014improved}. In
the subsequent paper~\cite{Gray-Scott2015}, this problem was resolved by
considering a smooth normal form for the Bogdanov--Takens bifurcation point,
which is a normal form for $C^\infty$-conjugacy (\emph{smooth equivalence}). 

In the follow-up paper~\cite{Al-Hdaibat2016}, progress was made in obtaining a
uniform approximation in time of the homoclinic solution, using a generalization
of the Lindstedt-Poincar\'e method. This removes the so-called parasitic turns
near the saddle point, as observed in~\cite{Kuznetsov2014improved}. Although, as
pointed out by~\cite{Algaba_2019}, there were mistakes in the third-order
approximation with the Lindstedt-Poincar\'e method, the asymptotics for the
homoclinic predictor from \cite{Kuznetsov2014improved} for the smooth normal form
improved significantly in the phase space.

Nonetheless, the task of correctly lifting the asymptotics in the normal form to
the parameter-dependent center manifold was not accomplished. Effectively, only
the zeroth-order approximation to the homoclinic solutions in the phase space,
i.e., a transformed homoclinic solution of the \emph{un}perturbed Hamiltonian
system, was available for a general $n$-dimensional system.

We will thus first revisit the second problem in the finite dimensional setting
of ODEs. After we solved the problem of correctly lifting the higher-order
homoclinic asymptotics obtained in the normal form to the parameter-dependent
center manifold, while preserving the approximation order, we turn our attention
to the infinite dimensional setting. 

We should remark here that while deriving the coefficients of the
parameter-dependent normal form and parameter-dependent center manifold
transformation of the codimension two bifurcation points in DDEs, we also obtain the
critical normal form coefficients, which were first derived in
\cite{Janssens:Thesis}.

Another codimension two bifurcation where a branch of homoclinic orbits may
emerge from, is the fold-Hopf bifurcation. However, the existence or non-existence
of the homoclinic branch cannot be fully determined by inspecting a finite number
of normal form coefficients. It is, therefore, also not surprising that no reliable
homoclinic predictors exist to start the continuation of the branch of homoclinic
orbits, although attempts have been made \cite{Budac:Thesis}. Provided that
homoclinic branches do emerge from the fold-Hopf bifurcation point, then there
are automatically two such branches. Furthermore, again under the
assumption the homoclinic branches exist, there are simple conditions, depending
on the normal form coefficients, to determine if the orbits are in fact Shilnikov
homoclinic solutions \cite{baldoma2020hopf} and imply the existence of chaos. We
will finish this thesis by showing how to obtain these homoclinic curves near a
fold-Hopf bifurcation point in a bi-antisymmetric tensor quantum field theories
with $O(N_1)\times O(N_2)$ symmetry.

\section{Structure of this thesis}
We start this thesis in the finite dimensional settings of ODEs. In
\cref{chapter:interplay} we rigorously derive correct third-order homoclinic
predictors for $n$-dimensional ODEs near generic codimension two Bogdanov--Takens
bifurcation points. We extend the normalization method to allow the inclusion of
a time-reparametrization. This allows us to use the orbital normal form for the
Bogdanov--Takens bifurcation. As a result, the homoclinic asymptotic up to order
three is independent of any normal form coefficients. Additionally, a detailed
comparison is made between applying different normal forms (smooth and orbital),
different phase conditions, and different perturbation methods (regular and
Lindstedt-Poincar\'e) to approximate the homoclinic solution near
Bogdanov--Takens points.  Examples demonstrating the correctness of the
predictors are given. The new homoclinic predictors are implemented in the
open-source \MATLAB/\OCTAVE continuation package \MATCONT. Complementary to
\cref{chapter:interplay} an
\href{https://mmbosschaert.github.io/MatCont7p2NewInitBTHom-/}{online Jupyter
Notebook} is provided in which 10 different models are considered using the new
homoclinic predictor and comparing different approximation methods in detail.

Then, shifting our attention to infinity dimensional systems,  we proof in
\cref{chapter:switching} the existence of a local parameter-dependent center
manifold in the abstract setting of dual perturbation theory. The proof
presented here lifts the restriction for the equilibrium to exists for all
nearby parameter values, as is done in \cite{diekmann1995delay}. Subsequently,
we show how the general results apply in the special case of DDEs. This allows
us to lift the normalization technique used in \cite{Kuznetsov2008} (applied to
parameter-dependent normal forms to start the continuation of nonhyperbolic
cycles emanating from generalized Hopf, fold-Hopf, and Hopf-Hopf bifurcation
points of ODEs) to the infinite dimensional setting of DDEs. The obtained
expressions give explicit formulas which have been implemented in the freely
available numerical software package \DDEBIFTOOL. While our theoretical results
are proven to apply more generally, the software implementation and examples
focus on DDEs with finitely many discrete delays.

Accompanying \cref{chapter:switching} is the supplement in
\cref{chapter:switching_supplement}. In this chapter we provide a complete
step-by-step walk-through of the examples in \cref{chapter:switching} using our
implementation in \DDEBIFTOOL. Additionally, we provide the code for simulation
near the bifurcation points, confirming our obtained results. The simulation is
either done with the built-in routine \lstinline|dde23| from \MATLAB
\cite{Shampine01solvingdelay} or the Python package \PYDELAY
\cite{Flunkert2009Flunkert}.

In \cref{chapter:BT_DDE} we combine the results from \cref{chapter:interplay} and
\cref{chapter:switching} to obtain homoclinic asymptotics near the generic
codimension two Bogdanov--Takens bifurcation in classical DDEs. Whereas the use
of the orbital normal form for the generic Bogdanov--Takens bifurcation was more
of a theoretical exercise in \cref{chapter:interplay}, it turns out to be rather
useful while treading the transcritical Bogdanov--Takens bifurcations. Indeed, by
a simple transformation \cref{eq:blowup} we directly obtain the third order
homoclinic asymptotics for the transcritical codimension two Bogdanov--Takens
bifurcation.

In the same style of \cref{chapter:switching,chapter:switching_supplement} there
is an accompanying supplement in \cref{chapter:BT_DDE_supplement} to
\cref{chapter:BT_DDE}. Thus, in this chapter we provide a complete step-by-step
walk-through of the examples in \cref{chapter:BT_DDE} using our implementation in
\DDEBIFTOOL. While doing so, we correct some mistakes made in the literature and
at the same time obtain some new results. Contrary to the supplement in
\cref{chapter:switching_supplement}, we will preform the simulations fully in the
programming language Julia.

Lastly, in \cref{chapter:chaoticRGFlow} we study a bi-antisymmetric tensor
quantum field theories with $O(N_1)\times O(N_2)$ symmetry. In particular, we
show that for special non-interger values of $N_1$ and $N_2$, the
Renormalization Group flow becomes chaotic. Thus, we allow $N_1$ and $N_2$ to
assume real values and treat them as bifurcation parameters. Using
analytical and numerical techniques we proof the existence of
Shilnikov homoclinic orbits near a zero-Hopf bifurcation point. As mentioned
above, no general predictors for the codimension one homoclinic bifurcation
curve is known to exists  in this case. Given the high dimensionality of the system, we
used Julia as our programming environment, providing us with great speed
and flexibility.


\section{Acknowledgments}
The work presented here would not have been possible without the help and
participation of many others, to whom I am grateful. First of all, I am grateful
to my supervisor Prof. dr. Peter de Maesschalck and co-supervisors Prof. dr.
Renato Huzak and Prof. dr. Yuri A. Kuznetsov. It was Peter who helped me
transition into the dynamical systems group at Hasselt University and gave me the
opportunity to revisit open problems encountered during my master thesis. Indeed,
the first two years of my assistant/Ph.D. position were at the computational
mathematics group at Hasselt University under supervision of Iuliu Sorin Pop and
co-supervision Prof. dr. Jochen Schuetz. Working as an assistant to these
energetic and enthusiastic professors for the first two years has been extremely
beneficial to me. Under their supervision, I've gained much insight into the field
of numerical mathematics (and drank more coffee). Assisting many numerical
courses and attending various talks provided by the computational mathematics
group during those two years provided me with tools that turned out to be very
helpful while solving the aforementioned open problems. Indeed, a common theme in
numerical analysis is the use of convergence plots, which are less often seen in
bifurcation analysis. It turned out to be very helpful in analyzing the
convergence order of the homoclinic approximations in higher-dimensional systems.
It is therefore also very unfortunate I didn't have enough time to work in both
groups simultaneously (next to the teaching duties).

The transition to the dynamical system group was made in May 2018. I'm very
grateful to Yuri for providing me a place to work at Utrecht University to
facilitate the collaboration.

I also like to thank Peter here once more for his support, and am especially
thankful for all the spontaneous meetings throughout the years under his
supervision. His high level of experience and great sense of humor always gave me
more energy to continue working on some problem once more.

Next, I would like to thank Sebastiaan Janssens. It was a pleasure to be working
with him on the switching paper published in 2019. I also like to thank him and
dr. Babette de Wolff for co-organizing with me the ``Delay-Days'' mini-symposia
at Hasselt University, online, and at Utrecht University. Next to his deep level
of understanding of perturbation theory for dual semigroups, he has a profound
writing style. When reading Sebastiaan's work, it feels as being on a journey,
rather than just reading mathematical results. I wish him all the best in
finishing his Ph.D. and hope to be working with him in the near feature once
more.

In addition to be working with Sebastiaan Janssens, I also had the pleasure of
co-supervising Bram Lentjes during his bachelor and master thesis. The master thesis
titled ``Center Manifolds and Periodic Normal Forms for Bifurcations of Limit
Cycles in DDEs'' under supervision of Yuri A. Kuznetsov has been a very
successful project. In fact, a first preprint is already available
\cite{Bram@2022}.

Furthermore, I'm very grateful to my girlfriend Veerle van Harten, who encouraged
me to apply for the assisted/Ph.D. position at Hasselt University in the first
place and was willing to move with me. I couldn't have done this without her.
Lastly, I would like to thank my mother Marjon van Uden, sister Mariska
Bosschaert, and grandmother Cornelius van Uden for their unconditionally support
throughout the years. Without the bi-weekly baby sitting of my mother on our
child Josephine Bosschaert during the COVID-19 pandemic, this thesis would not
have been.


\section{Publications}
The content of this thesis has been published or is submitted for publication,
see \cite{Bosschaert@Interplay,Switching2019,PhysRevD.105.065021}.
