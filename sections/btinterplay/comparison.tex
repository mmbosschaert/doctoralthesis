\ifthesis
\subsection{Comparison between smooth and orbital homoclinic predictors}
\else
\section{Comparison between smooth and orbital homoclinic predictors}
\fi
\label{sec:comparison_homoclinic_predictors}
Using the above transformations, we show that the homoclinic predictor for the
smooth normal form \cref{eq:BT_smooth_nf}, see
\cref{sec:asymptotics-for-homoclinic-solution-for-the-smooth-normal-form}, is
asymptotically equivalent to the orbital predictor derived in
\cref{sec:third_order_homoclinic_approximation_LP}. Thus, we assume that
\cref{eq:ODE} is given by 
\begin{equation}
    \label{eq:compare:f_normal_form} 
    f(x_1(t), x_2(t), \alpha_1, \alpha_2) := \left(\begin{array}{c}
            x_1(t),\\
            \alpha_1 + \alpha_2 x_1(t) + ax_0^2(t) + b x_0(t)x_1(t) \\
    \end{array}\right),
\end{equation}
where
\begin{equation*}
    g(x_1(t), x_2(t), \alpha_1, \alpha_2) =  a_1\alpha_2 x_0^2(t) + 
    b_1\alpha_2 x_0(t) x_1(t) + ex_0^{2}(t)x_1(t) + dx_0^3(t). \\
\end{equation*}

First we will focus on the predictors for the parameters. Using the procedure
outlined in \cref{subsec:center_manifold_tranformation_orbital}, we obtain that the
coefficients for the parameter transformation $K$ are given by
\begin{equation*}
\begin{gathered}
				K_{10} = \begin{pmatrix} 1 \\ \frac{ae -bd}{a^2} \end{pmatrix},\quad
				K_{01} = \begin{pmatrix} 0 \\ 1 \end{pmatrix},\quad
				K_{11} = \frac{3a_1b-4ab_1+2d}{ab}\begin{pmatrix}  1
								\\ \frac{(ae-bd)}{a^2} \end{pmatrix},\quad \\
				K_{02} = \begin{pmatrix} 0 \\ \frac{2a_1b-2ab_1+d}{ab} \end{pmatrix},\quad
				K_{03} = \begin{pmatrix} 0 \\ 0 \end{pmatrix}.
\end{gathered} 
\end{equation*}

From equation \cref{eq:alpha_espilon} we obtain the following approximation
\begin{equation}
\label{eq:compare:alpha_orbital}
\begin{pmatrix}
\bar \alpha_1^o \\[5pt] \bar \alpha_2^o
\end{pmatrix}
=
\begin{pmatrix}
-\frac{4a^3}{b^4} \epsilon^4
        + \frac{40a^3(-3a_1b+4ab_1-2d)}{7b^6} \epsilon^6
        + \frac{1152 a^3(-3a_1b + 4ab_1 -2d)}{2401 b^6} \epsilon^8
				+ \mathcal{O}(\epsilon^{10}) \\
\frac{10a}{7b} \epsilon^2
				+ \left( \frac{288b}{2401a} + \frac{50a(2a_1b-2ab_1+d)}{49b^3}
				+ \frac{4a(bd-ae)}{b^4} \right) \epsilon^4
				+ \mathcal{O}(\epsilon^{6})
\end{pmatrix}.
\end{equation}
Since the equation \cref{eq:compare:f_normal_form} is the smooth normal form, we can 
directly use \cref{eq:third_order_predictor_LP_tau_smooth} to obtain the
approximation
\begin{equation}
\label{eq:compare:alpha_smooth}
\begin{pmatrix}
\bar \alpha_1^s \\[5pt] \bar \alpha_2^s
\end{pmatrix}
=
\begin{pmatrix}
-\frac{4}{a} \epsilon^4 \\[5pt]
\frac{10b}{7a} \epsilon^2
				+ \frac{98 b (50 a b_1+73 d)-9604 a e-2450 a_1 b^2+288 b^3}{2401 a^2 b} \epsilon^4
				+ \mathcal{O}(\epsilon^6)
\end{pmatrix}.
\end{equation}
To compare these two predictors, we eliminate the parameter $\epsilon$ from both
equations. It would be tempting to first make a substitution for $\epsilon^2$ in
the equations. However, since for the orbits we need odd powers in $\epsilon$,
we will continue without this substitution. We assume $\alpha_1$ to be
positive, which implies that the coefficient $a$ is negative. The case that
$\alpha_1$ is negative is treated similarly and has been verified as well.

To eliminate $\epsilon$ from \cref{eq:compare:alpha_orbital}, we expand $\epsilon$
as a function of $\sqrt[4]{\alpha_1}$. Solving the resulting equation for real
positive $\epsilon$ we obtain
\begin{equation}
    \label{eq:compare:epsilon0}
    \epsilon^o(\alpha_1) = \frac{\sqrt[4]{-a} b}{\sqrt{2}a}\sqrt[4]{\alpha_1}-\frac{5 b  (-4 a
    b_1+3 a_1 b+2 d)}{28 \sqrt{2} a^2 \sqrt[4]{-a}}\sqrt[4]{\alpha_1} ^3 + \mathcal{O}(\sqrt[4]{\alpha_1}^5).
\end{equation}
Obviously, for the smooth predictor we obtain
\begin{equation*}
    \epsilon^s(\alpha_1) = \frac{\sqrt[4]{-a}}{\sqrt2} \sqrt[4]{\alpha_1}.
\end{equation*}
Here we added superscripts $o$ and $s$ to distinguish the different
$\epsilon$'s in the orbital and smooth homoclinic predictors,
respectively.

Substituting $\epsilon=\epsilon^o(\alpha_1)$ into the second equation of
\cref{eq:compare:alpha_orbital} yields
\begin{align*}
    \alpha_2(\alpha_1) ={}& \frac{5b}{7\sqrt{-a}} \sqrt{\alpha_1}
    + \frac{-49 b (50 a b_1+73 d)+4802 a e+1225 a_1 b^2-144 b^3}{4802 a^2} \alpha_1 \\ 
                          & + \mathcal{O}(\alpha_1^{3/2}).
\end{align*}
It can readably be seen that by eliminating $\epsilon$ from
\cref{eq:compare:alpha_smooth} using that $a<0$ we obtain the same expression,
i.e., the predictors agree up to the desired order.

Next, we turn our attention to the approximation of the homoclinic orbits. Due
to the various time transformations involved in the predictors, we compare the
asymptotic expansions of the orbital with the smooth homoclinic predictor.
Therefore, we can directly use the asymptotic obtained from the regular
perturbation method for both predictors. Thus, for the orbital predictor,
we will use 
\begin{equation}
   \label{eq:w0_w1_eta}
   \left\{
   \begin{aligned}
       w_0(\eta)  &= \frac{a}{b^2} \left( \sum_{i=0}^3 u_i(\frac{a}{b}\epsilon\eta) \epsilon^i +
       \mathcal{O}(\epsilon^4) \right)   \epsilon^2, \\
       w_1(\eta)  &= \frac{a^2}{b^3} \left( \sum_{i=0}^3 \dot u_i(\frac{a}{b}\epsilon\eta) \epsilon^i +
       \mathcal{O}(\epsilon^4) \right)   \epsilon^3, \\
   \end{aligned} 
   \right.
\end{equation}
with $u_i(i=1,2,3)$ is given by \cref{eq:u_orbital}, while for the smooth
predictor we will use \cref{eq:third_order_predictor_smooth_RPM_tau} instead.

Following the procedure as outlined in
\cref{subsec:center_manifold_tranformation_orbital}, we obtain that the coefficients
of the transformations $H$ and $\theta$ for the smooth orbital normal form
\cref{eq:normal_form_orbital} are given by
\begin{equation*}
\begin{gathered}
   q_{0} = \begin{pmatrix} 1 \\  0 \end{pmatrix}\!, \quad
   q_{1} = \begin{pmatrix} 0 \\  1 \end{pmatrix}\!, \quad
H_{2000} = \begin{pmatrix} -\frac{d}{2 a} \\ 0 \end{pmatrix}\!, \quad 
H_{1100} = \begin{pmatrix} \frac{-3 b d + 4 a e}{12 a^2} \\  0 \end{pmatrix}\!, \\
H_{0200} = \begin{pmatrix} 0 \\  \frac{-3 b d + 4 a e}{6 a^2} \end{pmatrix}\!, \quad
H_{3000} = \begin{pmatrix} 0 \\  \frac{-3 b d}{2 a} + 2 e \end{pmatrix}\!, \quad
H_{2100} = \begin{pmatrix} 0 \\  \frac{b (-3 b d + 4 a e)}{6 a^2} \end{pmatrix}\!, \\
H_{0010} = \begin{pmatrix} \frac{d}{4 a^2} \\  0 \end{pmatrix}\!, \quad
H_{1001} = \begin{pmatrix} \frac{-2 a b_1 + a_1 b + d}{a b} \\  0 \end{pmatrix}\!, \qquad
H_{0101} = \begin{pmatrix} 0 \\  \frac{-6 a b_1 + 4 a_1 b + 3 d}{2 a b} \end{pmatrix}\!, \\
H_{1101} = \begin{pmatrix} 0 \\  \frac{-3 (6 a b_1 - 4 a_1 b + b^2 - 3 d) d + 4 a b e}{12 a^2 b} \end{pmatrix}\!, \quad
H_{0102} = \begin{pmatrix} 0 \\  \frac{(6 a b_1 - 4 a_1 b - 3 d) (2 a b_1 - 2 a_1 b - d)}{2 a^2 b^2} \end{pmatrix}\!, \\
H_{1010} = \begin{pmatrix} 0 \\  \frac{-3 b d + 4 a e}{12 a^2} \end{pmatrix}\!, \quad
\theta_{1000} = -\frac{d}{2 a}, \quad
\theta_{0001} = -\frac{-2 a b_1 + 2 a_1 b + d}{2 a b},
\end{gathered}
\end{equation*}
while
\begin{align*}
H_{0001} = H_{2001} = H_{0002} = H_{1002} = H_{0003} = H_{0011} = H_{0110} = 
\begin{pmatrix} 0 \\ 0 \end{pmatrix}\!.
\end{align*}

Thus, the third-order homoclinic predictor using the smooth orbital normal form
in $\eta$ is given by 
\begin{equation*}
\begin{aligned}
    \bar x_\epsilon^o(\eta) =& \begin{pmatrix} 1 \\  0 \end{pmatrix} w_0(\eta) +
\begin{pmatrix} 0 \\ 1 \end{pmatrix} w_1(\eta) + 
\begin{pmatrix} -\frac{d}{2 a} \\ 0 \end{pmatrix} w_0^2(\eta) + 
\begin{pmatrix} \frac{-3 b d + 4 a e}{12 a^2} \\  0 \end{pmatrix} w_0(\eta) w_1(\eta) + \\
        & \begin{pmatrix} 0 \\ \frac{-3 b d + 4 a e}{6 a^2} \end{pmatrix} w_1^2(\eta) +
          \begin{pmatrix} 0 \\ \frac{-3 b d}{2 a} + 2 e \end{pmatrix} w_0^3(\eta) +
          \begin{pmatrix} 0 \\ \frac{b (-3 b d + 4 a e)}{6 a^2} \end{pmatrix} w_0^2(\eta) w_1(\eta) + \\
        & \begin{pmatrix} \frac{d}{4 a^2} \\  0 \end{pmatrix} \beta_1 +
          \begin{pmatrix} \frac{-2 a b_1 + a_1 b + d}{a b} \\  0 \end{pmatrix} w_0(\eta) \beta_2 +
          \begin{pmatrix} 0 \\ \frac{-6 a b_1 + 4 a_1 b + 3 d}{2 a b} \end{pmatrix} w_1(\eta) \beta_2 + \\
        & \begin{pmatrix} 0 \\ \frac{-3 (6 a b_1 - 4 a_1 b + b^2 - 3 d) d + 4 a b e}{12 a^2 b} \end{pmatrix} w_0(\eta) w_1(\eta) \beta_2 + \\
        & \begin{pmatrix} 0 \\ \frac{(6 a b_1 - 4 a_1 b - 3 d) (2 a b_1 - 2 a_1 b - d)}{2 a^2 b^2} \end{pmatrix} w_1(\eta) \beta_2^2 + 
          \begin{pmatrix} 0 \\ \frac{-3 b d + 4 a e}{12 a^2} \end{pmatrix} w_0(\eta) \beta_1,
\end{aligned}
\end{equation*}
where $w_{0,1}$ are given by \cref{eq:w0_w1_eta} and $\beta_{1,2}$ by
\cref{eq:third_order_predictor_RPM_tau}.
Since \cref{eq:compare:f_normal_form} is the smooth normal form
\cref{eq:BT_smooth_nf} we obtain from
\cref{eq:third_order_predictor_smooth_RPM_tau} the third-order homoclinic
approximation in $t$
\begin{equation*}
\bar x_\epsilon^s(t) =  \frac1{a} \sum_{i=0}^3
\begin{pmatrix}
 \left(      u_i(\epsilon t) \epsilon^i + \mathcal{O}(\epsilon^4) \right) \epsilon^2 \\
 \left( \dot u_i(\epsilon t) \epsilon^i + \mathcal{O}(\epsilon^4) \right) \epsilon^3
\end{pmatrix}.
\end{equation*}
To relate the smooth orbital predictor $\bar x_\epsilon^o(\eta)$ with the smooth
predictor $\bar x_\epsilon^s(t)$, we need to consider the time transformation
\begin{equation*}
\begin{aligned}
    t_\epsilon(\eta) = \eta \left( 1 + \theta_{0001}\frac{a}{b}\epsilon^2\left(\tau_0 + \tau_2 \epsilon^2 \right) \right) + 
        \theta_{1000} \frac{1}{b} \epsilon \int u(s) \, ds,
\end{aligned}
\end{equation*}
where
\[
    \theta_{1000} = -\frac{d}{2 a}, 
    \quad \theta_{0001} = -\frac{-2 a b_1 + 2 a_1 b + d}{2 a b}, 
    \quad s = \frac{a}{b} \eta \epsilon
\] 
and the integral is given by \cref{eq:int_u_s_regular_perturbation}. We
eliminate $\epsilon$ from $t_\epsilon(\eta)$ by substituting $\epsilon$ by
$\epsilon^o(\alpha_1)$ defined in \cref{eq:compare:epsilon0}. Subsequently, we
substitute $t_{\epsilon^o(\alpha_1)}(\eta)$ into $\bar x_{\epsilon^s(\alpha_1)}^s(t)$.
Thus, we now have two approximations, both parametrized by $\tau$. To compare these
approximations, we first rescale $\eta$ by $\frac{b}{a}\frac{\eta}{\epsilon}$,
otherwise the expansions in $\alpha_1$ become polynomial. We thus arrive at the
following equation which should be satisfied
\begin{equation*}
    \bar x_{\epsilon^s(\alpha_1)}^s\left(t_{\epsilon^o(\alpha_1)}\left(\frac{b}{a}\frac{\eta}{\epsilon}\right)\right) =
    \bar x_{\epsilon^o(\alpha_1)}^o\left(\frac{b}{a}\frac{\eta}{\epsilon} \right) +
    \begin{pmatrix}
        \mathcal{O}(\alpha_1^{3/2}) \\
        \mathcal{O}(\alpha_1^{7/4})
    \end{pmatrix}.
\end{equation*}

Expanding and simplifying the first component of
$\bar{x}_{\epsilon^o(\alpha_1)}^o(\frac{b}{a}\eta/\epsilon)$ in $\alpha_1$
gives
\begin{equation*}
    \begin{aligned}
        &\left(\bar x_{\epsilon^o(\alpha_1)}^o\left(\frac{b}{a}\frac{\eta}{\epsilon}\right)\right)_1 
        = \frac{3 \sech^2\eta-1}{\sqrt{-a}} \sqrt{\alpha_1} +\frac{18 \sqrt{2} b \tanh \eta \sech^2\eta \log (\cosh \eta)}{7 (-a)^{5/4}} \alpha_1^{3/4} \\
        &\qquad \frac{1}{196 a^2}\left[-6 \sech^2\eta \left(7 \left(5 a_1 b+6 b^2+7 d\right)-72 b^2 (\log (\cosh\eta)-1)\log (\cosh\eta) \right. \right. \\ 
        &\qquad \left. -36 b^2 \eta \tanh \eta\right)+70 a_1 b + 9 \sech^4\eta \left(24 b^2 (2-3 \log (\cosh\eta )) \log (\cosh\eta) \right. \\
        &\qquad \left. \left. +30 b^2+49 d\right)+98 d \right] \alpha_1 + \left[ \tanh \eta \left\{-18 b \log (\cosh \eta) \left(-980 a b_1+1225 a_1 b\right. \right. \right. \\ 
        &\qquad \left. -336 b^2(\log (\cosh \eta)-3) \log (\cosh \eta) +312 b^2+1176 d\right)-21 \sech^2\eta \left(-1372 a e \right. \\
        &\qquad +36 b \log (\cosh \eta) \left(6 b^2 \log (\cosh \eta) (4 \log (\cosh \eta)-7)-18 b^2-49 d\right)+234 b^3 \\
        &\qquad \left. \left. +1029 b d\right)-9604 a e+4914 b^3+7203 b d\right\} -4536 b^3 \eta  \left(-2 \log (\cosh \eta) \right. \\
        &\qquad \left. \left.+\sech^2\eta (3 \log (\cosh \eta)-1)+1\right)\right] \frac{\sech^2\eta}{4802 \sqrt{2} (-a)^{11/4}} \alpha_1^{5/4}
            + \mathcal{O}(\alpha_1^{3/2}).
    \end{aligned}
\end{equation*}
Similarly, for the second component of $\bar x_{\epsilon^o(\alpha_1)}^o(\eta/\epsilon)$ we obtain
\begin{equation*}
    \begin{aligned}
        & \left(\bar x_{\epsilon^o(\alpha_1)}^o\left(\frac{b}{a}\frac{\eta}{\epsilon}\right)\right)_2 
        = \frac{3 \sqrt{2} \tanh \eta \sech^2\eta}{\sqrt[4]{-a}} \alpha_1^{3/4} + \frac{9 b \sech^4\eta }{7 a}(\cosh (2 \eta)-2 (\cosh (2 \eta)-2) \\
        &\qquad \log (\cosh \eta)-1) \alpha_1 + -\frac{3 b \sech^4\eta}{196 \sqrt{2} (-a) ^{7/4}} (\sinh (2 \eta) (35 a_1-144 b (\log(\cosh \eta)-2) \\
        &\qquad \log (\cosh \eta)+48 b)+72 b (2 \eta-\eta \cosh (2 \eta)+\tanh \eta (2 \log (\cosh \eta) (6 \log (\cosh \eta)-7)\\
        &\qquad -3))) \alpha_1^{5/4} +\left(2 \left(36 b \log (\cosh \eta)  \left(490 a b_1-490 a_1 b+84 b^2 \log (\cosh \eta) (2 \log ( \right. \right. \right. \\
        &\qquad \left. \left. \cosh \eta)-9)+222 b^2-245 d\right)+90 b \left(-98 a b_1+98 a_1 b+111 b^2\right)-9604 a e \right. \\
        &\qquad \left. +11613 b d\right) + 3 \sech^2\eta \left(36 b \log (\cosh \eta) \left(245 (d-2 a b_1)+490 a_1 b+168 b^2 (14 \right. \right. \\
        &\qquad \left. -5 \log (\cosh \eta)) \log (\cosh \eta)-138 b^2\right)-3 b \left(-1960 a b_1+1960 a_1 b +6462 b^2 \right. \\
        &\qquad \left. +12985 d\right)+7 \sech^2\eta \left(-6860 a e+216 b^3 \log (\cosh \eta) (\log (\cosh\eta) (20 \log(\cosh \eta) \right. \\
        &\qquad \left. \left. -47)-1)+1818 b^3+5145 b d\right) + 48020 a e\right)-4536 b^3 \eta \tanh \eta \left(-4 \log (\cosh \eta) \right. \\
        &\qquad \left. \left. + \sech^2\eta (12 \log(\cosh \eta)-7)+4\right)\right) \frac{\sech^2\eta}{9604 (-a)^{5/2}}  \alpha_1^{3/2}
            + \mathcal{O}(\alpha_1^{7/4}).
    \end{aligned}
\end{equation*}
By expanding $\bar
x_{\epsilon^s(\alpha_1)}^s\left(t_{\epsilon^o(\alpha_1)}\left(\frac{b}{a}\frac{\eta}{\epsilon}\right)\right)$
and simplifying we obtain 
\begin{align*}
    \bar x_{\epsilon^s(\alpha_1)}^s\left(t_{\epsilon^o(\alpha_1)}\left(\frac{b}{a}\frac{\eta}{\epsilon}\right)\right) 
    &= \bar x_{\epsilon^o(\alpha_1)}^o\left(\frac{b}{a}\frac{\eta}{\epsilon} \right) +
        \begin{pmatrix}
            \frac{(3 b d - 4 a e) \sech^2(\eta) \tanh( \eta )}{\sqrt2 (-a)^{ 11/4 }} \alpha_1^{5/4} \\
            \frac{(3 b d - 4 a e)(\cosh (2 \eta )-2) \sech^4(\eta) }{2 (-a)^{5/2}} \alpha_1^{3/2}
        \end{pmatrix} \\
    &= \bar x_{\epsilon^o(\alpha_1)}^o\left(\frac{b}{a}\frac{\eta}{\epsilon} \right) +
        \begin{pmatrix}
            \frac{(3 b d - 4 a e) u_0'(\eta)}{24\sqrt2 (-a)^{ 11/4 }} \alpha_1^{5/4} \\
            \frac{(3 b d - 4 a e) u_0''(\eta)}{24 (-a)^{5/2}} \alpha_1^{3/2}
        \end{pmatrix},
\end{align*}
i.e., the predictors differ by a phase shift. In fact, by using the freedom in
the constants of integration in \cref{eq:u_int_s}, we can let 
\[
    t(\eta) \rightarrow t(\eta) + \theta_{1000} \frac{1}{b} \frac{2}{3}
    \left(\frac{4 a e}{b d}-3\right) \epsilon^2.
\] 
In this case, we will have equivalence between the predictors up to the desired
order. 

\begin{remark}
It is important to note here the phase condition used in the orbital
predictor isn't preserved under the transformation $H$. Therefore,
any improvements obtained in the approximation to the homoclinic solutions in a
normal form due to a different phase condition is, in general, not preserved
when lifting the approximations to the center manifold.
\end{remark}

