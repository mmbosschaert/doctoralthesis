\section[CMR and normalization]{Parameter-dependent center manifold reduction combined with normalization and time-reparametrization}
\label{sec:Center_manifold_reduction_ODE}

Without loss of generality, we may
assume that the system \cref{eq:ODE} has a codimension two bifurcation of the equilibrium $x_0 = 0$ at $\alpha_0 = 0$. Let the normal form on the
$n_c$-dimensional center manifold be given by
\begin{equation}
    \label{eq:G}
    \frac{d}{d\eta} w(\eta) = G(w(\eta), \beta),
                    \qquad G\colon\mathbb R^{n_c} \times \mathbb
                    R^2 \to \mathbb R^{n_c}.
\end{equation}
Here $G$ is assumed to be one of the (known) equivalent normal forms, i.e.,
it is known which terms are in $G$ are which are not there. Thus $G$ admits a series expansion of the form
\begin{equation}
    \label{eq:G_expansion}
    G(w(\eta), \beta) = \sum_{|\nu|+|\mu| \geq 1} \frac{1}{\nu!\mu!} g_{\nu\mu} w^\nu \beta^\mu,
\end{equation}
where the multi-indices $\nu$ and $\mu$ have lengths $n_0$ and 2,
respectively, and the coefficients $g_{\nu\mu}\in\mathbb R^{n_c}$.

Similarly, $f$ admits the series expansion given by
\begin{equation}
    \label{eq:f_expansion}
    f(x,\alpha) = A x + J_1 \alpha 
    + \sum_{k,l \geq 0, \; k+l>1} \frac{1}{k!l!} D_1^k D_2^l f_{(0,0)}( x^{(k)}, \alpha^{(l)}).
\end{equation}
Here $A:=D_1 f_{(0,0)}$ is the Jacobian matrix of \cref{eq:ODE} evaluated at
$(x,\alpha)=(0,0)$. Likewise, $J_1 := D_2 f_{(0,0)}$ and 
$D_1^k D_2^l f_{(0,0)} \colon \left(\mathbb R^n \right)^k \times \left(\mathbb R^2\right)^l \rightarrow \mathbb R^n$ 
are the mixed Fr\'echet derivatives of order $k+l$. Lastly, $x^{(k)} := (x,\dots,x) \in \left(\mathbb R^n\right)^k$,
with $x\in\mathbb R^n$ repeated $k$ times and similarly for $\alpha^{(l)}$.

Now suppose that a parameter-dependent approximation to an emanating
codimension one bifurcation curve in the normal form \cref{eq:G} is known and given by
\begin{equation}
    \label{eq:general_approximation} 
    \epsilon \mapsto (w_\epsilon(\eta), \beta_\epsilon).
\end{equation}
We assume here that $w_\epsilon(\eta)$ and $\beta_\epsilon$ have been derived
up to a certain (not necessarily similar) order in $\epsilon$. Let $\Phi$
be a basis for the center subspace of $A$ and $\Psi$ a basis for the center
subspace of the adjoint $A^T$, normalized such that $\Psi^T \Phi$ is the
identity matrix on $\mathbb R^n$. Then 
%by the reduction principle of Shoshitaishvili 
the solutions on the parameter-dependent center manifold can be
parametrized by 
\begin{align}
    \label{eq:H}
    x(t(\eta)) &= \Phi w(\eta) + H(w(\eta), \beta), 
               & H\colon\mathbb R^{n_c} \times \mathbb R^2 \to \mathbb R^n, \\
               \label{eq:K}
    \alpha &= K(\beta), 
           & K\colon\mathbb R^2 \to \mathbb R^2,
\end{align}
where $H$ describes a correction to the critical center subspace and $K$ provides a correspondence between the original parameters in \cref{eq:ODE} and those of the normal form \cref{eq:G}.
Next, let the time $t$ in the ODE \cref{eq:ODE} and the time $\eta$ in the normal form
\cref{eq:G} be related through the parameter-dependent time-rescaling
\begin{equation}
				\label{eq:theta}
				\frac{dt}{d\eta} = \theta(w, \beta), \qquad 
								\theta\colon \mathbb R^{n_c} \times \mathbb R^2 \to \mathbb R.
\end{equation}

Then the invariance of the center manifold implies the \emph{homological
equation}
\begin{equation}
    \label{eq:homological_equation}
    \tag{HOM}
    f(\Phi w + H(w,\beta), K(\beta)) \theta(w, \beta) = (\Phi + H_w(w, \beta)) G(w,\beta).
\end{equation}

The mappings $H$, $K$, and $\theta$ can be expanded as
\begin{equation}
    \label{eq:HK_expansion}
    H(w, \beta) = \sum_{\substack{|\nu|+|\mu| \geq 1, \\ |\nu|\neq 1 \wedge |\mu|=0}} \frac{1}{\nu!\mu!} H_{\nu\mu} w^\nu \beta^\mu,
    \qquad
    K(\beta) = \sum_{|\mu|\geq 1} \frac1{\mu!} K_\mu \beta^\mu,
\end{equation}
and
\begin{equation}
    \label{eq:theta_expansion}
    \theta(w, \beta) = \sum_{|\nu|+|\mu| \geq 0} \frac1{\nu!\mu!} \theta_{\nu\mu} w^\nu \beta^\mu.
\end{equation}

Substituting
\cref{eq:G_expansion,eq:f_expansion,eq:HK_expansion,eq:theta_expansion} into
\cref{eq:homological_equation} and collecting terms of equal power in $(w,\beta)$, we
obtain linear systems which can be solved at each order recursively. However,
these solutions may be non-unique.

The question now is which coefficients are needed to include in the expansion of
\cref{eq:f_expansion,eq:HK_expansion,eq:theta_expansion} in order to translate the 
approximation \cref{eq:general_approximation} obtained up to a certain order,
to the parameter-dependent center manifold of the codimension two point in such
a way that the approximation order of \cref{eq:general_approximation} is maintained.

The first task here is to determine which terms in the expansion of the 
system restricted to the center manifold of \cref{eq:ODE} at the bifurcation
point alter the approximation \cref{eq:general_approximation}.  It is
important here to not only take into account the terms that alter the
approximation that are present in the normal form $G$, but also terms that are
\emph{not present in the normal form}, as long as the approximation
\cref{eq:general_approximation} is affected by the terms. 

The coefficients $g_{\nu\mu}$ corresponding to these terms, i.e., the terms that
alter the predictor, which components consequently may vanish, can be computed
by reconstructing the normal form from \cref{eq:H,eq:K}, or equivalently from
\cref{eq:homological_equation}. Indeed, by multiplying the homological equation
\cref{eq:homological_equation} by $\Psi$, using \cref{eq:G_expansion}, and
rearranging terms we obtain
\begin{align*}
\frac{d}{d\eta} w(\eta) &=  \sum_{|\nu|+|\mu| \geq 1} \frac{1}{\nu!\mu!} g_{\nu\mu} v^\nu \beta^\mu \\ 
                        &= \Psi (f(\Phi w + H(w,\beta), K(\beta)) \theta(w, \beta) - D_wH(w, \beta)G(w,\beta)).
\end{align*}

% The terms of the
% second type, i.e., not in the normal form, but do alter the approximation
% \cref{eq:general_approximation} must be transformed into higher order terms.

The next observation is that the coefficient $g_{\mu\nu}$ first appears in the
system obtained from the homological equation \cref{eq:homological_equation} by
considering the coefficient corresponding to the term $w^\mu\beta^\nu$.
Using the expansion \cref{eq:f_expansion} we see that this system has the
following form
\begin{equation}
    \label{eq:general_form_system}
    \theta_{00} (A H_{\nu\mu} + J K_\mu) = \Phi g_{\nu\mu} + \Pi_{\nu\mu}.
\end{equation}
Here $\Pi_{\nu\mu}$ contains the remaining terms. The exact representation
$\Pi_{\nu\mu}$  is not important here. What is important, however, is that it
only depends on coefficients $g_{rs}, H_{rs}$, $K_{rs}$, and $\theta_{\nu\mu}$ such
that $|r| \leq |\nu|$ and $|s| \leq |\mu|$, except for the
coefficient $g_{\nu\mu}$ itself. Note that, depending on the normal form under
consideration, the coefficient $H_{\nu\mu}$ and $K_{\nu\mu}$ may also be
present in $\Pi_{\nu\mu}$. 

Thus, in order to compute the coefficient $g_{\nu\mu}$, we need to solve the
system \cref{eq:general_form_system}. Only then we are able to reconstruct the
normal form on the center manifold at the bifurcation point such that the terms
$g_{\nu\mu}$, which alter the predictor, are either present, or transformed
away into higher order terms. Therefore, we conclude with the following
statement.

\begin{proposition}
\label{thm:coefficients}
Consider the normal form \cref{eq:G} for a codimension two bifurcation point at
$(x,\alpha) \equiv (0,0)$ of \cref{eq:ODE}. Let the asymptotic approximation 
\cref{eq:general_approximation} for a codimension one bifurcation curve
emanating from the bifurcation point be determined up to a certain order.
Suppose that the asymptotics are affected by terms with the coefficients
$g_{\nu\mu}$ of the ODE restricted to the center manifold of the bifurcation
point, which may or may not be present in the normal form. 

Then in order to
lift the available asymptotics in the normal form to the full system, such that
the approximation order of \cref{eq:general_approximation} in maintained, it is
generically necessary and sufficient to include the coefficients $H_{rs}$ and $K_{rs}$,
with $|r| \leq |\nu|$ and $|s| \leq |\mu|$, into the expansion of $H$ and $K$
in \cref{eq:HK_expansion}, in such a way that the systems
\cref{eq:general_form_system} are solvable. The systems to be solved reveal
simultaneously which higher-order derivatives must be included in the Taylor
expansion of $f$ in \cref{eq:f_expansion}.
\end{proposition}

Although the above statement may look obvious, it has not been understood correctly
in all previous attempts to derive a higher order approximation of the homoclinic
bifurcation curve emanating from the generic codimension two Bogdanov--Takens in
$n$-dimensional systems, see \cite{Al-Hdaibat2016,Kuznetsov2014improved,Gray-Scott2015}.

The word \emph{generically} in \cref{thm:coefficients} refers to the exclusion
of special cases in which the system \cref{eq:ODE} contains certain symmetry or
other restrictions, see, for example, \cite[2.3.3 Symetries and
Reversibility]{Iooss@2011}. In such situations, certain coefficients and
higher-order derivatives are known to vanish beforehand.

In \cref{subsec:center_manifold_tranformation_orbital} we will show how to
determine which coefficients in the ODE restricted to the center manifold of a
generic codimension two Bogdanov--Takens bifurcation point in \cref{eq:ODE}
alter the homoclinic asymptotics up to order three. Based on this, and
using \cref{thm:coefficients}, we are able to rigorously determine which
coefficients to include in the parameter-dependent center manifold
transformation $H$ and parameter transformation $K$.

\begin{remark}
The construction outlined above is applicable not only to codimension two
bifurcations and emanating codimension one bifurcation curves. The same
construction also holds when considering, e.g., codimension three
bifurcation points and emanating sheets of codimension two, and the natural
extensions of this.
\end{remark}

\begin{remark}
We have not discussed yet which coefficients of the time-reparametrization $\theta$ are
needed to include. These terms are determined by the requirement that the systems in
\cref{eq:general_form_system} are solvable for the by now known, multi-indices
$\nu$ and $\mu$. We know that this is indeed possible by the existence of the
considered normal form. Furthermore, the determined coefficients $H_{\nu\mu}$ and
$K_{\mu}$ are left unchanged if a reparametrization of time is not allowed.
Indeed, the time-reparametrization $\theta$ only simplifies the normal form,
but does not change which terms, in, and not in, the normal form affected
by the asymptotic approximation.
\end{remark}

\begin{remark}
Note that the in practice when studying truncated normal forms up to a
relatively low order, it is common not to include the factorials in the normal
form expansion \cref{eq:G_expansion}.
\end{remark}

\subsection{Parameter-dependent normal form}%
%
Suppose that the ordinary differential equation \cref{eq:ODE} undergoes a
generic codimension two Bogdanov--Takens bifurcation at $(x,\alpha) \equiv
(x_0,\alpha_0)$. That is, the linearization of \cref{eq:ODE} has a double, but
not semisimple, zero eigenvalue, while all other eigenvalues are away from the
imaginary axis. The critical smooth normal form on the two-dimensional center
manifold of $(x_0, \alpha_0)$ is given
by~\cite{Arnold_1983,guckenheimer1983nonlinear}
\begin{equation*}
\begin{aligned}
\begin{cases}
\dot w_0 = w_1, \\
\dot w_1 = a w_0^2 + b w_0 w_1 + \mathcal O(\|w\|^3),
\end{cases}
\end{aligned}
\end{equation*}
where 
\begin{equation*}
ab \neq 0,
\end{equation*}
$w_i$ is a shorthand notation for $w_i(\eta)$ for $i=0,1$, and the dot is
the derivative with respect to $\eta$.

Under these non-degeneracy and certain transversality conditions, the
\emph{topological normal form} for the codimension two Bogdanov--Takens
bifurcation is given by  
\begin{equation}
\label{eq:universal_unfolding}
\begin{aligned}
\begin{cases}
\dot w_0 = w_1, \\
\dot w_1 = \beta_1 + \beta_2 w_1 + a w_0^2 + b w_0 w_1,
\end{cases}
\end{aligned}
\end{equation}
see~\cite{bogdanov1975zeroEigenvalues,Bogdanov1976,Takens1974,guckenheimer1983nonlinear,Kuznetsov2004}. 
It is well-known that in system (\ref{eq:universal_unfolding}) three codimension one
bifurcation curves emanate from $(\beta_1,\beta_2)=(0,0)$: a saddle-node, a Hopf,
and a saddle-homoclinic bifurcation curve. 

By using either the regular perturbation or the Lindstedt-Poincar\'e method,
an approximation to the homoclinic bifurcation curve and the corresponding
solution can, theoretically, be obtained up to any order in the
singular-rescaling parameter $\epsilon$,
see~\cite{Kuznetsov2014improved,Gray-Scott2015, Al-Hdaibat2016, Algaba_2019}.

To obtain the second-order homoclinic approximation to the homoclinic solutions
on the center manifold in \cref{eq:ODE}, it is, in general, insufficient to
only consider the topological normal form \cref{eq:universal_unfolding},
see~\cite{Broer1991}.  One way to deal with this problem is to consider the
smooth parameter-dependent normal form 
\begin{equation}
\label{eq:BT_smooth_nf}
\begin{cases}
\begin{aligned}
\dot{w}_0 = & w_1,\\
\dot{w}_1 = & \beta_1+\beta_2 w_1+\left(a+a_1\beta_2\right)w_0^{2}
 +\left(b+b_1\beta_2\right) w_0w_1+ew_0^{2}w_1+dw_0^{3} + g(w,\beta),
\end{aligned}
\end{cases}
\end{equation}
with
\[
    g(w,\beta) = \mathcal O(|\beta_1|\|w\|^2 + |\beta_2| w_1^2 + \|\beta\|^2\|w\|^2
					 + \|\beta\|\|w\|^3 + \|w\|^4)
\] 
as in~\cite{Gray-Scott2015,Al-Hdaibat2016}. Here $w_i=w_i(t)(i=0,1)$ now
depends explicitly on $t$ as in the original ODE \cref{eq:ODE}.

However, in this paper, we will allow for a time-reparametrization and use the
$C^\infty$-equivalent normal form 
\begin{equation}
\label{eq:normal_form_orbital}
\begin{cases}
\begin{aligned}
	\dot w_0 &= w_1, \\
	\dot w_1 &= \beta_1 + \beta_2 w_1 + aw_0^2 + b w_0 w_1 + w_0^2 w_1
								h(w_0,\beta) + w_1^2 Q(w_0,w_1,\beta),
\end{aligned}
\end{cases}
\end{equation}
where $h$ is $C^\infty$ and $Q$ is $N$-flat for an a priori given $N$,
see~\cite{Broer1991}. Here, the dot represents the derivative with respect to
the new time $\eta$ of $w_i(\eta)(i=0,1)$.  Furthermore, we will show that we
can assume $h(0,0)=0$. Note that we do not impose the coefficients to be $a=1$
and $b=\pm 1$ as in~\cite{Broer1991}. This simplifies the systems to be solved
in the next section without complicating the solutions for the homoclinic
corrections. Indeed, we can scale away the coefficients $a$ and $b$ in the
singular-rescaling. Also note that the normal form
\cref{eq:normal_form_orbital} was used in~\cite{Broer1991} to study degenerate
(codimension 3) Bogdanov--Takens bifurcations, while we found it to be essential
for constructing homoclinic predictors in the case of generic codimension two
Bogdanov--Takens bifurcations.

To approximate the homoclinic solutions emanating from the Bogdanov--Takens point
we apply the singular rescaling
\begin{equation}
\label{eq:blowup}				
w_0 = \frac a{b^2} u \epsilon^2, \quad
w_1 = \frac{a^2}{b^3} v \epsilon^3, \quad
\beta_1 = -4 \frac{a^3}{b^4} \epsilon^4, \quad 
\beta_2 = \frac a b \tau \epsilon^2, \quad 
s = \frac ab \epsilon \eta, \quad (\epsilon \neq 0),
\end{equation}
to \cref{eq:normal_form_orbital} with $h(0,0)=0$ to obtain the second-order
nonlinear oscillator
\begin{equation}
\label{eq:second_order_nonlinear_oscillator}
				\ddot u = -4 + u^2 + \dot u \left( u + \tau \right)\epsilon + \mathcal
								O(\epsilon^4).
\end{equation}
Here the dot represents the derivative with respect to $s$.

\subsection{Center manifold reduction for smooth orbital normal form}
\label{subsec:center_manifold_tranformation_orbital}
We want to relate the third-order homoclinic approximation in the smooth
orbital normal form \cref{eq:normal_form_orbital} to the homoclinic solutions
of \cref{eq:ODE} near $(x_0,\alpha_0)$. The third-order $\epsilon$-expansion
of the homoclinic solution depends on the coefficients in $\epsilon$ up to order
three in the perturbed Hamiltonian system
\cref{eq:second_order_nonlinear_oscillator}, see
\cref{sec:PolynomailLindstedtPoincare}.  By inspecting the blowup
transformation \cref{eq:blowup}, we can determine exactly which coefficients
must be included in the expansion of $H$ and $K$, and which multilinear forms
to include in the expansion of $f$. Indeed, we search for those terms 
\begin{equation}
\label{eq:bt_general_term_ode_cm}
\beta_1^i\beta_2^jw_0^kw_1^l, \qquad \text{with }  i,j,k,l \in \mathbb N_0,
\end{equation}
in the expansion of the reduced system restricted to the center manifold of
$(x_0,\alpha_0)$ in \cref{eq:ODE} that alter the coefficients in $\epsilon$ up
to order three in the nonlinear oscillator
\cref{eq:second_order_nonlinear_oscillator}. Thus, by substituting
\cref{eq:blowup} into \cref{eq:bt_general_term_ode_cm}, inspecting the
resulting exponents, and taking into account
\cref{eq:second_order_nonlinear_oscillator}, we obtain that these terms are
determined by solving the linear Diophantine equation
\begin{equation}
    \label{eq:diophantine}
    4i + 2j + 2k + 3l - 4 = n, \qquad n\in\{-2,-1,0,1,2,3\},
\end{equation}
for $i,j,k,l \in \mathbb N_0$. From \cref{eq:diophantine} we obtain the
solutions listed in \cref{fig:terms_affecting_predicor}.
%
\begin{table}
\begin{center}
\begingroup
\renewcommand*{\arraystretch}{1.4}
\begin{tabular}{rl}
\hline
order in $\epsilon$ & affected terms \\
\hline%
\(\epsilon^{-2}\) & \(w_0,\; \beta_2\)\\
\(\epsilon^{-1}\) & \(w_1\)\\
\(\epsilon^0\)  & \(w_0^2,\; w_0 \beta_2,\; \beta_2^2,\; \beta_1\)\\
\(\epsilon^1\)  & \(w_0w_1,\;  w_1\beta_2\) \\
\(\epsilon^2\)  & \(w_1^2,\; w_0^3,\; w_0^2 \beta_2 ,\; w_0 \beta_2^2 ,\; \beta_2^3,\; 
									w_0\beta_1 ,\; \beta_1 \beta_2\)\\
\(\epsilon^3\)  & \(w_0^2 w_1,\; w_0 w_1 \beta_2,\;  w_1 \beta_2^2,\;  w_1 \beta_1\) \\[0.2cm]
\hline
\end{tabular}
\endgroup
\caption{\label{fig:terms_affecting_predicor}
        Terms in the reduced system restricted to the parameter-dependent center manifold of the
        codimension two Bogdanov--Takens point in \cref{eq:ODE} that alter the third-order homoclinic predictor.} 
\end{center}
\end{table}
%
To be concrete, suppose the term $\alpha_2^3$ is present in the system
restricted to the center manifold of the Bogdanov--Takens point obtain by a
standard center manifold reduction, i.e., without simultaneously transforming
the reduced system into normal form. Then, by applying the blowup
transformation \cref{eq:blowup}, the coefficient of the term $\alpha_2^3$ will
show up in the coefficient of $\epsilon^2$ in the perturbed Hamiltonian
system \cref{eq:second_order_nonlinear_oscillator}. Since we will derive a
third-order approximation for \cref{eq:second_order_nonlinear_oscillator} in
which the corresponding coefficient of the term $\alpha_2^3$ is not present,
this term needs to be transformed away. In \cref{app:incorrect_predictor} this
example, is workout in detail to show that the transformation
in~\cite{Al-Hdaibat2016} leads to an incorrect predictor for the parameters.

Thus, we expand the mappings $H$, $K$, and $\theta$, including precisely those
coefficients needed to transfer the homoclinic predictor in the normal form to
the center manifold maintaining the approximation order. Using
\cref{fig:terms_affecting_predicor} we write:
%
\begin{align}
f(x,\alpha) ={}&
Ax+J_1\alpha+\frac12 B(x,x)+A_1(x,\alpha)+\frac12 J_2(\alpha,\alpha) 
+\frac16 C(x,x,x) \label{eq:f_expansion_bt} \\
& +\frac12 B_1(x,x,\alpha)+\frac12 A_2( x,\alpha, \alpha)
	+ \frac16 J_3(\alpha, \alpha, \alpha)  
  +\mathcal{O}\left(\|(x,\alpha)\|^4\right), \nonumber \\
H(w,\beta)={}& H_{0010}\beta_1 + H_{0001} \beta_2 
  + \frac12 H_{2000}w_0^2 + H_{1100}w_0w_1 + \frac12 H_{0200}w_1^2 
  \label{eq:H_expansion} \\
  & + H_{1010}w_0\beta_1 + H_{1001}w_0\beta_2 + H_{0110}w_1\beta_1 
  + H_{0101}w_1\beta_2 + \frac12 H_{0002}\beta_2^2\nonumber \\
  & + H_{0011}\beta_1\beta_2 + \frac16 H_{3000}w_0^3 + \frac12 H_{2100}w_0^2w_1 
  + H_{1101}w_0w_1\beta_2 + \frac12 H_{2001}w_0^2\beta_2\nonumber \\
  & + \frac{1}{6}H_{0003}\beta_2^3 + \frac12 H_{1002}w_0\beta_2^2 
  + \frac12 H_{0102}w_1\beta_2^2 \nonumber \\
  & + \mathcal{O}(|w_1|^3+|w_0w_1^2|+|\beta_2w_1^2|+|\beta_1|\|w\|^2
  +|\beta_1^2|\|w\| + |\beta_1^2| + \|(w,\beta)\|^4), \nonumber \\
K(\beta)={}& K_{10}\beta_1 + K_{01}\beta_2 + \frac{1}{2}K_{02}\beta_2^{2} 
	+ K_{11}\beta_1\beta_2 + \frac16 K_{03} \beta_2^3
  \label{eq:K_expansion} \\
  &+ \mathcal{O}(|\beta_1|^2+|\beta_1||\beta_2|^2 + |\beta_1|^2|\beta_2|  
	+ |\|\beta\|^4), \nonumber \\
\label{eq:theta_expansion_bt}
\theta(w,\beta) ={}& 1 + \theta_{1000}w_0 + \theta_{0001} \beta_2 
    + \mathcal O\left(|w_1| + |\beta_2| + \|(w,\beta)\|^2\right).
\end{align}
%
where $A=D_1 f(x_0,\alpha_0)$, $J_1=D_2 f_{\alpha}(x_0,\alpha_0)$, and
$B,J_2,J_3,C,A_1,A_2$ and $B_1$ are the standard multilinear forms, introduced
for readability.
\begin{figure}
\renewcommand{\qtreeunaryht}{4ex}
\Tree%
[.{$q_0, q_1, p_1, p_0$} 
[.{$\theta_{1000}, \gamma_1, \gamma_2, H_{2000}, H_{1100}, H_{0200}, H_{3000}, H_{2100}$} 
		[.{$\gamma_3, \delta_1, \gamma_4, \delta_2, H_{0010}, H_{0001}, K_{10}, K_{01}$}
			{$H_{1010}, H_{0110}$} 
			[.{$(\gamma_5, \theta_{0001}), H_{1001}, H_{0101}, H_{2001}, H_{1101}$}
				{$K_{11}, H_{0011}$} 
				[.{$\gamma_6, \delta_3, K_{02}, H_{0002}, H_{1002}, H_{0102}$} 
					{$K_{03},H_{0003}$} ] ] ] ] ]
\caption{\label{fig:coefficients} Schematic overview is which order the coefficients in the expansion of
$H$, $K$ and $\theta$ are derived.}
\end{figure}
%
In \crefrange{subsubsection:generalized_eigenvectors}{subsubsection:K03_H0003},
we will solve the unknowns in
\cref{eq:H_expansion,eq:K_expansion,eq:theta_expansion_bt}. For this it might
be useful to consult \cref{fig:coefficients}. There a schematic overview is
given in which order the coefficients are solved. Notice, in particular, that
of course the critical normal form coefficients can be solved independently
of the parameter-dependent normal form coefficients. Also, due to the
interdependence of the normal form coefficients, it is convenient not to solve
order by order. 

\begin{remark}
Notice that compared with~\cite{Al-Hdaibat2016} there are four additional terms
in the expansion of $H$, i.e. with  coefficients $H_{0011},H_{1002},H_{0102}$,
and $H_{0003}$, and two additional terms in the expansion of $K$, with
coefficients $K_{11}$ and $K_{03}$. Also, there are two additional higher-order
derivatives, namely $A_2$ and $J_3$, included in the Taylor expansion of $f$
previously not taking into account. By \cref{thm:coefficients}, we know that
the coefficients \cref{eq:H_expansion,eq:K_expansion,eq:theta_expansion_bt} and
the higher-order derivatives in \cref{eq:f_expansion_bt}, are both necessary
and sufficient. The coefficients in the time-reparametrization \cref{eq:theta_expansion_bt}
are determined by the requirement that the systems obtained from the homological
equation are solvable.
\end{remark}

\subsubsection{(Generalized) eigenvectors}
\label{subsubsection:generalized_eigenvectors}

We assume that the equilibrium $(x_0, \alpha_0)$ has a double (but not
semisimple) zero eigenvalue, while all other eigenvalues are away from the
imaginary axis. Thus, there exist two real linearly independent (generalized)
eigenvectors, $q_0, q_1 \in \mathbb{R}^{n}$, of $A$, such that 
\begin{equation}
\label{eq:eigenvectors}
Aq_0=0,\qquad Aq_1=q_0,
\end{equation}
and two left (generalized) eigenvectors $p_1^T, p_0^T \in\mathbb{R}^{n}$,
of $A$, such that
\[
p_1 A=0,\qquad p_0 A=p_1.
\]
These vectors can be normalized to satisfy
\[
p_i q_j=\delta_{ij},\qquad i=0,1,\,j=0,1.
\]
%
As in~\cite{Kuznetsov2005practical}, we impose the condition
\begin{equation}
\label{eq:q0} 
q_0^T q_0=1,\qquad q_1^T q_0=0,
\end{equation}
to uniquely define the vectors $\{q_0,q_1,p_1,p_0\}$ up to a plus or minus sign.

Note that collecting the coefficients of the linear terms in $w$ in the
homological \cref{eq:homological_equation} equation are precisely the systems
defining the (generalized) eigenvectors \cref{eq:eigenvectors}. 


\subsubsection{Critical coefficients}%

Collecting the quadratic coefficients in $w$ from the homological
\cref{eq:homological_equation} yields the systems
%
\begin{align}
-AH_{2000} &= B(q_{0,}q_0) - 2aq_1, \label{eq:AH2000} \\
-AH_{1100} &= B(q_0,q_1) - bq_1 + \theta_{1000}q_0 - H_{2000},  \label{eq:AH1100} \\
-AH_{0200} &= B(q_1,q_1) - 2H_{1100}. \label{eq:AH0200} 
\end{align}

The Fredholm solvability condition for the first two systems yields the well-known expressions
\begin{align*}
a &= \frac12 p_1 B(q_0,q_0), \\
b &= p_1 B(q_0, q_1) + p_0 B(q_0, q_0),
\end{align*}
for the critical coefficients, see for example~\cite{Kuznetsov1999}. By the
non-degeneracy conditions, we have that $ab \neq 0$.

\begin{remark}
Since we assume that $p_1 B(q_0,q_0) \neq 0$ we can use the freedom in the
eigenvectors,
\[
(q_0, q_1) \to c_1 (q_0,q_1), \qquad
(p_1, p_0) \to \frac1{c_1} (p_1, p_0),
\]	
to normalize the critical coefficient
\[
a= \frac12 p_1 B(q_0, q_0),
\]
to one. Solving for $c_1$ then gives
\[
c_1=\frac2{p_1 B(q_0, q_0)}.
\]
Alternatively, the freedom could have been used to set $b=1$. To have the
situation $a=1$ and $b=\pm1$, as in~\cite{Broer1991}, the coefficient in front
of the constant term in the expansion of $\theta$, i.e., $\theta_{0000}$, should
be used. For convenience, we fixed the constant $\theta_{0000}$ to $1$.
\end{remark}

Now that \cref{eq:AH2000,eq:AH1100} are solvable, we can define
\begin{align*}
\hat H_{2000} &= -\AINV \left( B(q_0,q_0) - 2aq_1 \right), \\
\hat H_{1100} &= -\AINV \left( B(q_0,q_1) - bq_1 - \hat H_{2000} \right).
\end{align*}
The expression $x=\AINV y$ is defined by solving the non-singular
bordered system
\begin{equation*}
\begin{pmatrix}
A & p_1^T \\ q_0^T & 0
\end{pmatrix}
\begin{pmatrix}
x\\ s
\end{pmatrix}
=
\begin{pmatrix}
y\\ 0
\end{pmatrix},
\end{equation*}
for the unknown $(x,s) \in \mathbb R^{n+1}$ that necessarily satisfies $s = 0$.
While this uniquely defines $x$, the solutions to system $Ax=y$, are of course
still non-unique. The properties of bordered linear systems and their role in
numerical bifurcation analysis are discussed in~\cite{Keller1987Numerical}
and~\cite[Chapter 3]{govaerts2000numerical}.

It follows that the general solutions of the systems \cref{eq:AH2000,eq:AH1100}
are given by
\begin{align*}
H_{2000} &= \hat H_{2000} + \gamma_1 q_0, \\
H_{1100} &= \hat H_{1100} + \gamma_1 q_1 - \theta_{1000} q_1 + \gamma_2 q_0.
\end{align*}
The constant $\gamma_1$ is determined by the solvability condition from
\cref{eq:AH0200}, which gives
\begin{equation*}
\gamma_1 = p_0  \left( B(q_0,q_1) -\hat H_{2000} \right) 
								+ \frac12 p_1 B(q_1,q_1) + \theta_{1000}.
\end{equation*}

To determine the constant $\gamma_2$ and the coefficient $\theta_{1000}$ we
consider the $w_0^3$ and $w_0^2w_1$ terms in the homological equation. After
some simplification, we obtain the systems
\begin{align}
\label{eq:AH3000}
-A H_{3000} ={}& 3 B(H_{2000},q_0) + C(q_0,q_0,q_0) + 6a \theta_{1000} q_1
									- 6a H_{1100}, \\
\label{eq:AH2100}
-A H_{2100} ={}& 2 B(H_{1100},q_0) + B(H_{2000},q_1) + C(q_0,q_0,q_1) - 2 a H_{0200} - 2 b H_{1100}  \\
				& -H_{3000} + 2 \theta_{1000} (bq_1 - \theta_{1000}q_0 +H_{2000}). \nonumber
\end{align}
The solvability condition of the first equation determines $\theta_{1000}$ as
\begin{equation}
\label{eq:theta1000}
\theta_{1000} = -\frac1{12a} p_1 \left\{ 
			3B(\hat H_{2000},q_0) + C(q_0,q_0,q_0)
			\right\} + \frac12 p_1 \hat H_{1100}.
\end{equation}
The solvability condition for the system in \cref{eq:AH2100} yields, after a
rather lengthy calculation, that $\gamma_2$ is determined by 
\begin{align}
\label{eq:gamma_2}
\gamma_2 &= \frac1{6a} 
\bigg[ p_1\left\{ 2 B(\hat H_{1100},q_0)+B(\hat H_{2000},q_1)
+ C(q_0,q_0,q_1) \right\} \\
  & \qquad +2a p_0 B(q_1,q_1) + 2b p_0 \left( B(q_0,q_1) 
    - \hat H_{2000} \right) \nonumber \\
	& \qquad + p_0 \left( 3B(\hat H_{2000},q_0) 
	  + C(q_0,q_0,q_0) \right) \nonumber \\
	& \qquad + \gamma_1 b - 10 a p_0 \hat H_{1100} + 2 b \theta_{1000} \bigg].
				\nonumber 
\end{align}
Since the systems in \cref{eq:AH0200,eq:AH2100,eq:AH3000} are now all consistent,
we are allowed to take the bordered inverses to obtain
\begin{align}
	H_{0200} ={} & -\AINV \left[B(q_1,q_1) - 2H_{1100} \right], \nonumber          \\
  \label{eq:H3000}
	H_{3000} ={} & -\AINV \left[ 3B(H_{2000},q_0) + C(q_0,q_0,q_0) +
		6a\theta_{1000}q_1 - 6a H_{1100} \right],                            \\
  \label{eq:H2100}
	H_{2100} ={} & -\AINV \left[ -2 a H_{0200}-2 b H_{1100}-H_{3000} + 2
	B(H_{1100},q_0) \right.                                              \\
	             & \left. \quad + B(H_{2000},q_1) + 2 \theta_{1000}
	(bq_1 - \theta_{1000}q_0 +H_{2000})+C(q_0,q_0,q_1) \right]. \nonumber
\end{align}

\subsubsection{Parameter-dependent linear coefficients}
The coefficients of the linear terms in $\beta$ give the
systems
%
\begin{equation}
\label{eq:AH0010}
\begin{aligned}
-AH_{0001} &= J_1K_{01}, \\
-AH_{0010} &= J_1K_{10} - q_1.
\end{aligned}
\end{equation}
Since $p_1$ and $J_1$ are known, we can calculate 
\begin{equation*}
    \nu := (p_1 J_1)^T.
\end{equation*}
By the transversality condition, the vector $\nu$ is nonzero. It then follows
from the Fredholm alternative that
\begin{equation*}
\begin{aligned}
K_{01}   &= \delta_1\hat K_{01}, \\
H_{0001} &= \delta_1 \left( \hat H_{0001} + \gamma_3 q_0 \right), \\
K_{10}   &= \hat K_{10} + \delta_2 K_{01}, \\
H_{0010} &= \hat H_{0010} + \delta_2 H_{0001} + \gamma_4 q_0,
\end{aligned}
\end{equation*}
where
\begin{equation*}
\begin{aligned}
\hat K_{10} &= \frac1{\|\nu\|^2}\nu, \\
\hat H_{0010} &= -\AINV \left(J_1\hat K_{10} - q_1 \right), \\
\hat K_{01} &=
\begin{pmatrix}
	0 & -1 \\ 1 & 0
\end{pmatrix} \hat K_{10}, \\
\hat H_{0001} &= -\AINV J_1\hat K_{01},
\end{aligned}
\end{equation*}
and $\delta_{1,2}$, $\gamma_{3,4}$ are real constants determined by the
solvability condition of the $w\beta$ terms in the homological equation.
Collecting the corresponding systems in the homological equation yields 
\begin{equation*}
\begin{aligned}
-AH_{1001} &= B(H_{0001},q_0)+A_1(q_0,K_{01}), \\
-AH_{0101} &= B(H_{0001},q_1)+A_1(q_1,K_{01})-H_{1001}-q_1+\theta_{0001}q_0, \\
-AH_{1010} &= B(H_{0010},q_0)+A_1(q_0,K_{10})-H_{1100} +\theta_{1000}q_1,\\
-AH_{0110} &= B(H_{0010},q_1)+A_1(q_1,K_{10})-H_{0200}-H_{1010}.
\end{aligned}
\end{equation*}
The solvability condition for the first two systems yields
\begin{equation*}
\begin{aligned}
\gamma_3 &= -\frac{p_1 \left( 
				B(\hat H_{0001},q_0)+A_1(q_0,\hat K_{01}) \right)}{2a}, \\
\delta_1 &= \frac{1}{p_1 \left(
			B(\hat H_{0001},q_1)+A_1(q_1,\hat K_{01}) \right) + p_0
      \left( B(\hat H_{0001},q_0)+A_1(q_0,\hat K_{01}) \right) + \gamma_3 b},
\end{aligned}
\end{equation*}
while the solvability condition for the latter two systems yields
\begin{equation*}
\begin{aligned}
\gamma_4 &= \frac{p_1H_{1100} - \theta_{1000} - p_1 \left( B(\hat
				H_{0010},q_0)+A_1(q_0,\hat K_{10})\right)}{2a} ,\\
\delta_2 &= -p_1 \left( B(\hat H_{0010},q_1) + A_1(q_1,\hat K_{10}) \right)
				- \gamma_4 b + p_1 H_{0200} \\
	& \quad - p_0 \left( B(\hat H_{0010},q_0)+A_1(q_0,\hat K_{10})-H_{1100}
	\right). \\
\end{aligned}
\end{equation*}
Note that the denominator in $\delta_1$ is nonzero by the transversality
condition.


\subsubsection{Coefficients \texorpdfstring{$H_{1010} \text{ and }
								H_{0110}$}{H1010 and H0110}}

Since we do not need to use the non-uniqueness in the systems for the
coefficients $H_{1010}$ and $H_{0110}$ to simplify higher-order systems we
let
\begin{equation*}
\begin{aligned}
H_{1010} &= -\AINV \left[ B(H_{0010},q_0)+A_1(q_0,K_{10})-H_{1100}
								+\theta_{1000}q_1 \right],\\
H_{0110} &= -\AINV \left[B(H_{0010},q_1)+A_1(q_1,K_{10})-H_{0200}-H_{1010}
\right].
\end{aligned}
\end{equation*}

\subsubsection{Coefficients \texorpdfstring{$(\theta_{0001},\gamma_5),H_{1001},
H_{0101}, H_{2001}, H_{1101}$}{(theta0001,gamma5),H1001,H0101,H2001,H1101}}

Define
\begin{equation*}
\begin{aligned}
\hat H_{1001} &= -\AINV \left[ B(H_{0001},q_0)+A_1(q_0,K_{01}) \right], \\
\hat H_{0101} &= -\AINV \left[
				B(H_{0001},q_1)+A_1(q_1,K_{01})-\hat H_{1001}-q_1 \right], \\
\end{aligned}
\end{equation*}
so that
\begin{equation*}
\begin{aligned}
H_{1001} &= \hat H_{1001} + \gamma_5 q_0, \\
H_{0101} &= \hat H_{0101} + \gamma_5 q_1 - \theta_{0001} q_1. \\
\end{aligned}
\end{equation*}

In order to determine $\gamma_5$ and $\theta_{0001}$, we consider the systems
corresponding to the $w_0^2\beta_2$ and $w_0w_1\beta_2$ terms in the homological
equation. These are given by
\begin{equation}
\label{eq:AH2001_AH1101}
\begin{aligned}
-AH_{2001} &= -2 a H_{0101} + A_1(H_{2000},K_{01}) +
	B(H_{0001},H_{2000}) + 2 B(H_{1001},q_0) \\
				& \qquad + 2 a \theta_{0001} q_1 +
				B_1(q_0,q_0,K_{01}) + C(H_{0001},q_0,q_0), \\
-AH_{1101} &= -b H_{0101} - H_{1100} - H_{2001} + A_1(H_{1100},K_{01}) + \\
				& \qquad \theta_{1000} (H_{1001} + q_1 - \theta_{0001} q_0) + 
				B(H_{0001},H_{1100}) + B(H_{0101},q_0) + \\
				& \qquad B(H_{1001},q_1) + \theta_{0001} (H_{2000} + b q_1 -
				\theta_{1000} q_0) + B_1(q_0,q_1,K_{01}) \\
				& \qquad + C(H_{0001},q_0,q_1).
\end{aligned}
\end{equation}
The Fredholm solvability condition leads to the following system to be solved
\begin{equation}
\label{eq:gamma_5_theta0001}
\begin{pmatrix}
				 2a &  4a \\
				  b &   b 
\end{pmatrix}
\begin{pmatrix}
				\gamma_5 \\
				\theta_{0001}
\end{pmatrix}
=
\begin{pmatrix}
				\zeta_1 \\
				\zeta_2 
\end{pmatrix},
\end{equation}
where
\begin{equation}
\label{eq:zeta1_zeta2}
\begin{aligned}
\zeta_1 &=  -p_1 \left[ -2 a\hat h_{0101}  
			+ A_1(h_{2000},K_{01}) + B(h_{0001},h_{2000}) \right.  \\
			& \left. \qquad + 2 B(\hat h_{1001},q_0)
			+ B_1(q_0,q_0,K_{01}) + C(h_{0001},q_0,q_0) \right], \\
\zeta_2 &= -p_1 \left[ -b \hat h_{0101} - h_{1100}  +
				A_1(h_{1100},K_{01}) \right. \\
				& \qquad + \theta_{1000} (\hat h_{1001} + q_1) + 
				B(h_{0001},h_{1100}) + B(\hat h_{0101},q_0) \\
				& \qquad \left. + B(\hat h_{1001},q_1) + B_1(q_0,q_1,K_{01})
				+ C(h_{0001},q_0,q_1) \right]  \\
				& \quad -p_0 \left[ -2 a \hat h_{0101} + A_1(h_{2000},K_{01}) +
				B(h_{0001},h_{2000}) \right. \\
				& \qquad + \left. 2 B( \hat h_{1001},q_0) +
				B_1(q_0,q_0,K_{01}) + C(h_{0001},q_0,q_0) \right] .
\end{aligned}
\end{equation}	
Notice that the matrix is invertible by the non-degeneracy condition. Now that
the systems in \cref{eq:AH2001_AH1101} are solvable, we obtain
\begin{equation}
\label{eq:H2001_H1101}
\begin{aligned}
H_{2001} &= -\AINV \left[ -2 a H_{0101} + A_1(H_{2000},K_{01}) +
	      B(H_{0001},H_{2000})  \right. \\
				& \qquad \left. + 2 B(H_{1001},q_0) + 2 a \theta_{0001} q_1 +
				B_1(q_0,q_0,K_{01}) + C(H_{0001},q_0,q_0) \right], \\
H_{1101} &= -\AINV \left[ -b H_{0101} - H_{1100} - H_{2001} +
				A_1(H_{1100},K_{01}) + \right. \\
				& \qquad \theta_{1000} (H_{1001} + q_1 - \theta_{0001} q_0) + 
				B(H_{0001},H_{1100}) + B(H_{0101},q_0) + \\
				& \qquad B(H_{1001},q_1) + \theta_{0001} (H_{2000} + b q_1 -
								\theta_{1000} q_0) + B_1(q_0,q_1,K_{01}) \\
				& \qquad  \left. + C(H_{0001},q_0,q_1) \right].
\end{aligned}
\end{equation}

\subsubsection{Coefficients \texorpdfstring{$K_{11} \text{ and } H_{0011}$}{K11
				and H0011}}

Collecting the systems corresponding to the $\beta_1 \beta_2$ term in the
homological equation yields
\begin{equation}
\label{eq:AH0011}
\begin{aligned}
-A H_{0011} &= J_1 K_{11} + A_1(H_{0001},K_{10}) + A_1(H_{0010},K_{01}) \\
						& \qquad +
						B(H_{0001},H_{0010})+J_2(K_{01},K_{10})+\theta_{0001}q_1-H_{0101}.
\end{aligned}
\end{equation}
Using the identity 
%
\begin{equation*}
   p_1J_1K_{10}=1 
\end{equation*}
from the second system in \cref{eq:AH0010} combined with the solvability
condition yields
\begin{equation}
\begin{aligned}
K_{11}={}& -p_1\left[A_1(H_{0001},K_{10})+A_1(H_{0010},K_{01}) \right. \nonumber\\
				 &\left. + B(H_{0010},H_{0001}) +
				 J_2(K_{10},K_{01}) + \theta_{0001} q_1 - H_{0101} \right]K_{10}.
\end{aligned}
\end{equation}
It follows that 
\begin{equation}
\begin{aligned}
H_{0011} &= -\AINV \left[ J_1 K_{11} + A_1(H_{0001},K_{10}) +
				A_1(H_{0010},K_{01}) \right. \\
				 & \qquad \left. +
                    B(H_{0001},H_{0010})+J_2(K_{01},K_{10})+\theta_{0001}q_1-H_{0101}
				 \right].
\end{aligned}
\end{equation}

\subsubsection{Coefficients
				\texorpdfstring{$K_{02},H_{0002},H_{1002},H_{0102}$}
				{H0002,K02,H1002,H0102}}

The systems corresponding to the $\beta_2^2, w_0\beta_2^2$ and $w_1\beta_2^2$,
terms in the homological equation yields
\begin{equation}
\label{eq:AH0002_AH1002_AH0102}
\begin{aligned}
-A H_{0002}={}&  J_1K_{02} + 2A_1(H_{0001},K_{01})
							 + B(H_{0001},H_{0001}) + J_2(K_{01},K_{01}), \\
-A H_{1002}={}& 2A_1(H_{1001},K_{01}) + A_1(q_0,K_{02}) + A_2(q_0,K_{01},K_{01})
					 \\ & + B(q_0,H_{0002}) + 2B(H_{0001},H_{1001}) +
					 2B_1(q_0,H_{0001},K_{01}) \\
  & + C(q_0,H_{0001},H_{0001}), \\
-A H_{0102}={}& 2A_1(H_{0101},K_{01}) + A_1(q_1,K_{02}) + A_2(q_1,K_{01},K_{01}) \\
  & + B(q_1,H_{0002}) + 2B(H_{0001},H_{0101}) + 2B_1(q_1,H_{0001},K_{01}) \\
	& + C(q_1,H_{0001},H_{0001}) +2\theta_{0001} (H_{1001} + q_1 -
	  \theta_{0001}q_0)  \\
  & - 2H_{0101} - H_{1002}.
\end{aligned}
\end{equation}
The first system is solved similarly as \cref{eq:AH0011}. In order to make the second and third systems
consistent, define
\begin{equation*}
\begin{aligned}
\hat K_{02}={}&-p_1\left[2A_1(H_{0001},K_{01})+B(H_{0001},H_{0001})
				+J_2(K_{01},K_{01})\right]K_{10}, \\
\hat H_{0002}={}& -\AINV\left[ J_1 \hat K_{02} + 2A_1(H_{0001},K_{01})
							 + B(H_{0001},H_{0001}) + J_2(K_{01},K_{01})\right].
\end{aligned}
\end{equation*}
Then the general solution to the first system in \cref{eq:AH0002_AH1002_AH0102}
can be written as
\begin{equation*}
\begin{aligned}
				K_{02}={}& \hat K_{02} + \delta_3 K_{01}, \\
				H_{0002}={}& \hat H_{0002} +  \delta_3 H_{0001} + \gamma_6 q_0.
\end{aligned}
\end{equation*}
Substituting these two expressions into the last two system of \cref{eq:AH0002_AH1002_AH0102} and
using the solvability condition yields
\begin{equation*}
\begin{aligned}
				\gamma_6 ={}& -\frac1{2a} p_1 \left[ 2A_1(H_{1001},K_{01}) +
				A_1(q_0,\hat K_{02}) + A_2(q_0,K_{01},K_{01}) \right. \\
	& \qquad + B(q_0,\hat H_{0002}) + 2B(H_{0001},H_{1001}) +
				2B_1(q_0,H_{0001},K_{01}) \\
  & \left. \qquad + C(q_0,H_{0001},H_{0001}) \right],  \\
\delta_3 ={}& -p_1 \left[ 2A_1(H_{0101},K_{01}) + A_1(q_1,\hat K_{02}) +
    A_2(q_1,K_{01},K_{01}) \right. \\
	& \qquad + B(q_1,\hat H_{0002}) + 2B(H_{0001},H_{0101}) +
	    2B_1(q_1,H_{0001},K_{01}) \\
	& \left. \qquad +  C(q_1,H_{0001},H_{0001}) +2\theta_{0001} (H_{1001} + q_1) -
	  2H_{0101} \right] \\
	& \qquad - p_0 \left[ 2A_1(H_{1001},K_{01}) + A_1(q_0,\hat K_{02}) +
    A_2(q_0,K_{01},K_{01}) \right. \\ 
	& \qquad + B(q_0,\hat H_{0002}) + 2B(H_{0001},H_{1001}) +
    2B_1(q_0,H_{0001},K_{01}) \\
  & \left. \qquad + C(q_0,H_{0001},H_{0001}) \right] - \gamma_6 b.
\end{aligned}
\end{equation*}
Now that the last two systems in \cref{eq:AH0002_AH1002_AH0102} are consistent,
we obtain
\begin{equation*}
\begin{aligned}
H_{1002}={}& -\AINV \left[ 2A_1(H_{1001},K_{01}) + A_1(q_0,K_{02}) +
				A_2(q_0,K_{01},K_{01}) \right. \\ 
			  & + B(q_0,H_{0002}) + 2B(H_{0001},H_{1001}) +
				2B_1(q_0,H_{0001},K_{01})  \\
				& + \left. C(q_0,H_{0001},H_{0001}) \right], \\
H_{0102}={}& -\AINV \left[ 2A_1(H_{0101},K_{01}) + A_1(q_1,K_{02}) +
				A_2(q_1,K_{01},K_{01}) \right. \\
  & + B(q_1,H_{0002}) + 2B(H_{0001},H_{0101}) + 2B_1(q_1,H_{0001},K_{01}) \\
	& + C(q_1,H_{0001},H_{0001}) +2\theta_{0001} (H_{1001} + q_1 -
	  \theta_{0001}q_0)  \\
	& \left. - 2H_{0101} - H_{1002} \right].
\end{aligned}
\end{equation*}

\subsubsection{Coefficients \texorpdfstring{$K_{03} \text{ and } H_{0003}$}{K03
				and H0003}}
\label{subsubsection:K03_H0003}

Collecting the systems corresponding to the $\beta_2^3$ term in the
homological equation yields
\begin{equation*}
\begin{aligned}
-A H_{0003} ={}& J_1 K_{03} + A_1(H_{0001},K_{02}) + A_1(H_{0002},K_{01})
				+ 2 (A_1(H_{0001},K_{02}) \\
				& + A_1(H_{0002},K_{01}) + 3 B(H_{0001},H_{0002}) + 3 J_2(K_{01},K_{02})
				\\
				& + 3 A_2(H_{0001},K_{01},K_{01}) + 3 B_1(H_{0001},H_{0001},K_{01}) \\
				& + C(H_{0001},H_{0001},H_{0001}) + J_3(K_{01},K_{01},K_{01}).
\end{aligned}
\end{equation*}
This equation is solved similarly as equation \cref{eq:AH0011}. We obtain
\begin{equation*}
\begin{aligned}
K_{03}={}& -p_1 \left[ A_1(H_{0001},K_{02}) + A_1(H_{0002},K_{01})
				+ 2 A_1(H_{0001},K_{02}) \right. \\
				& + 2 A_1(H_{0002},K_{01}) + 3 B(H_{0001},H_{0002}) + 3 J_2(K_{01},K_{02})
				\\
				& + 3 A_2(H_{0001},K_{01},K_{01}) + 3 B_1(H_{0001},H_{0001},K_{01}) \\
				& \left. + C(H_{0001},H_{0001},H_{0001}) +
				J_3(K_{01},K_{01},K_{01}) \right] K_{10}, \\
H_{0003} ={}& -\AINV \left[ J_1 K_{03} + A_1(H_{0001},K_{02}) +
				A_1(H_{0002},K_{01}) + 2 A_1(H_{0001},K_{02}) \right. \\
				& + 2 A_1(H_{0002},K_{01}) + 3 B(H_{0001},H_{0002}) + 3 J_2(K_{01},K_{02})
				\\
				& + 3 A_2(H_{0001},K_{01},K_{01}) + 3 B_1(H_{0001},H_{0001},K_{01}) \\
				& \left. + C(H_{0001},H_{0001},H_{0001}) +
				J_3(K_{01},K_{01},K_{01})\right] .
\end{aligned}
\end{equation*}

\subsection{Center manifold reduction for smooth normal form} 
\label{sec:center-manifold-reduction-without-time-reparametrization}
If we do not allow for a reparametrization of time, we can no longer consider the
normal form \cref{eq:normal_form_orbital}. Instead, we need to use the smooth
normal form as introduced in~\cite{Gray-Scott2015}, i.e., equation
\cref{eq:BT_smooth_nf}. Applying the blowup transformation
\begin{equation}
\label{eq:blowup_smooth}				
\beta_1 = - \frac4a \epsilon^4, \quad 
\beta_2 = \frac b a \tau \epsilon^2, \quad 
w_0 = \frac1a u \epsilon^2, \quad
w_1 = \frac1a v \epsilon^3, \quad
s = \epsilon t, \quad (\epsilon \neq 0),
\end{equation}
to the smooth normal form, we obtain the second-order nonlinear differential
equation 
\begin{equation}
\label{eq:second_order_nonlinear_oscillator_smooth_normalform}
				\ddot u = -4 + u^2 + \frac ba \dot u \left( u + \tau \right)\epsilon 
				+ \frac1{a^2} u^2\left(\tau b a_1 + d u \right) \epsilon^2
        + \frac1{a^2} u\dot u \left( \tau b b_1 + e u \right) \epsilon^3
				+ \mathcal{O}(\epsilon^4).
\end{equation}
Here the dot represents the derivative with respect to $s$. 

Note that, by using hypernormalization, we can still simplify the smooth normal form.
Indeed, as already remarked in~\cite{Kuznetsov2005practical} the coefficient
$e$ can be set to zero.  Furthermore, it can be seen from the system in
\cref{eq:gamma_5_theta0001} that either the coefficient $a_1$ or $b_1$ can also
be removed. The natural choice here is for the coefficient $b_1$ to
be set to zero in the normal form. The parameter-dependent center manifold
transformation in this situation is obtained by first setting the coefficients
$\theta_{1000}$ and $\theta_{0001}$ to zero in
\cref{subsec:center_manifold_tranformation_orbital}. Equation \cref{eq:AH3000}
becomes
\begin{equation}
    \label{eq:AH3000_smooth}
    -A H_{3000} = 3 B(H_{2000},q_0) + C(q_0,q_0,q_0) - 6d q_1 - 6a H_{1100},
\end{equation}
and equation \cref{eq:theta1000} becomes redundant. After $\gamma_2$ in
\cref{eq:gamma_2} has been calculated, the Fredholm solvability condition yields
that
\begin{equation*}
    d = \frac16 p_1 \left[ 3B(H_{2000},q_0) + C(q_0,q_0,q_0)  - 6 a H_{1100}\right].
\end{equation*}
Now that \cref{eq:AH3000_smooth} is consistent, we can replace \cref{eq:H3000}
with
\begin{equation*}
	H_{3000} = -\AINV \left[ 3B(H_{2000},q_0) + C(q_0,q_0,q_0) - 6dq_1 - 6a H_{1100} \right].
\end{equation*}

Next, we replace the first equation in \cref{eq:AH2001_AH1101} with
\begin{equation}
\label{eq:AH2001}
\begin{aligned}
-AH_{2001} ={}& - 2 a_1 q_1 -2 a H_{0101} + A_1(H_{2000},K_{01}) + B(H_{0001},H_{2000}) \\
              & + 2 B(H_{1001},q_0) + B_1(q_0,q_0,K_{01}) + C(H_{0001},q_0,q_0)
\end{aligned}
\end{equation}
and the system given in \cref{eq:gamma_5_theta0001} becomes the single equation
\[
\gamma_5 = -\frac{\zeta_2}{b}.
\]
Here $\zeta_2$ is still given by the second equation in \cref{eq:zeta1_zeta2}
(with $\theta_{1000}$ still set to zero), while $\zeta_1$ is no longer needed.
Applying the Fredholm solvability condition to \cref{eq:AH2001} yields 
\begin{equation*}
\begin{aligned}
    a_1 &= \frac{1}{2} p_1 \left[-2 a H_{0101} + A_1(H_{2000},K_{01}) +
	B(H_{0001},H_{2000}) + 2 B(H_{1001},q_0) \right. \\
        & \left. \qquad + B_1(q_0,q_0,K_{01}) + C(H_{0001},q_0,q_0) \right].
\end{aligned}
\end{equation*}
Since \cref{eq:AH2001} is now consistent, we can replace the first equation in
\cref{eq:H2001_H1101} with
\begin{equation*}
\begin{aligned}
    H_{2001} &{}= -\AINV \left[ -2 a H_{0101} + A_1(H_{2000},K_{01}) + B(H_{0001},H_{2000})  \right. \\
				& \left. + 2 B(H_{1001},q_0) - 2 a_1 q_1 + B_1(q_0,q_0,K_{01}) + C(H_{0001},q_0,q_0) \right].
\end{aligned}
\end{equation*}
The remaining systems and equations are unchanged.

To compare the homoclinic predictors corresponding to different normal forms, we also
provide the parameter-dependent center manifold transformation for the smooth
normal form \cref{eq:BT_smooth_nf} without transforming away the coefficients
$e$ and $b_1$. In this case, in addition to the modification given above, we also set
$\gamma_2$ and $\gamma_5$ to zero. Then the system in \cref{eq:AH2100} becomes
\begin{equation}
\label{eq:AH2100smooth}
\begin{aligned}
-A H_{2100} ={}& - 2 e q_1 -2 a H_{0200}-2 b H_{1100}-H_{3000} + 2 B(H_{1100},q_0) \\
				& + B(H_{2000},q_1) + C(q_0,q_0,q_1),
\end{aligned}
\end{equation}
while the second systems in \cref{eq:AH2001_AH1101} should be replaced with
\begin{equation}
\label{eq:AH1101smooth}
\begin{aligned}
-AH_{1101} ={}& -b_1 q_1 -b H_{0101} - H_{1100} - H_{2001} + A_1(H_{1100},K_{01}) + \\
              &+ B(H_{0001},H_{1100}) + B(H_{0101},q_0) + B(H_{1001},q_1) \\
              &+ B_1(q_0,q_1,K_{01}) + C(H_{0001},q_0,q_1).
\end{aligned}
\end{equation}
Applying the Fredholm solvability condition to these equations gives
\begin{equation*}
\begin{aligned}
e ={}& \frac{1}{2} p_1 \left[ -2 a H_{0200}-2 b H_{1100}-H_{3000} + 2 B(H_{1100},q_0) \right.  \\
     & \left. + B(H_{2000},q_1) + C(q_0,q_0,q_1) \right], \\ 
b_1 ={}& p_1 \left[ -b H_{0101} - H_{1100} - H_{2001} + A_1(H_{1100},K_{01}) + B(H_{0001},H_{1100})  \right. \\
      & \left. + B(H_{0101},q_0) + B(H_{1001},q_1) + B_1(q_0,q_1,K_{01}) + C(H_{0001},q_0,q_1) \right].
\end{aligned}
\end{equation*}
Now that \cref{eq:AH2100smooth,eq:AH1101smooth} are consistent, we can replace
\cref{eq:H2100} and the second system in \cref{eq:H2001_H1101} with
\begin{align*}
	H_{2100} ={} & -\AINV \left[ -2e q_1 - 2 a H_{0200}-2 b H_{1100}-H_{3000} + 2 B(H_{1100},q_0) \right. \\
	             & \left. \quad + B(H_{2000},q_1) + C(q_0,q_0,q_1) \right]. \nonumber
\end{align*}
and
\begin{equation*}
\begin{aligned}
H_{1101} ={}& -\AINV \left[ -b_1 q_1 -b H_{0101} - H_{1100} - H_{2001} + A_1(H_{1100},K_{01}) + \right. \\
				& + B(H_{0001},H_{1100}) + B(H_{0101},q_0) + B(H_{1001},q_1) \\
				& \left. + B_1(q_0,q_1,K_{01}) + C(H_{0001},q_0,q_1) \right],
\end{aligned}
\end{equation*}
respectively. The remaining systems and equations are unchanged.



\begin{remark}
The derivation in~\cite{Al-Hdaibat2016,Kuznetsov2014improved} leads to a `big'
system in which equations need to be solved simultaneously. The derivation
presented here does not involve a `big' system, making the expressions also
suitable for infinitely-dimensional ODEs generated by partial and
delay differential equations, to which the (parameter-dependent) center
manifold theorem applies.
\end{remark}
