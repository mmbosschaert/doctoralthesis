\section{Discussion}%
We have derived third-order predictors for the homoclinic curve emanating from
the generic codimension two Bogdanov--Takens bifurcation in general
$n$-dimensional autonomous ODEs. By considering the smooth orbital normal form
\cref{eq:normal_form_orbital} and incorporating the time-re\-pa\-ram\-e\-triza\-tion in the
homological equation \cref{eq:homological_equation}, we were able to derive
the third-order asymptotic of the homoclinic curve independent of any undetermined
normal form coefficients. However, for this simplification, there is a price to
pay. Firstly, the systems to be solved to obtain the coefficients for the
parameter-dependent center manifold become more difficult, see
\cref{subsec:center_manifold_tranformation_orbital}. Ideally, there should be
an automatic algorithm in line with~\cite{Murdock@2003}. However, to the best
of our knowledge, such algorithms do not exist yet.  Secondly, the translation
of time in the homological equation needs to be inverted numerically. This,
however, can be done relatively cheap and is very accurate, as shown by the
examples.

In \cref{thm:coefficients} we have shown how to obtain a correct transformation
to the parameter-dependent center manifold by inspecting which terms are in,
\emph{and are not in}, the normal form that alters the homoclinic asymptotic up
to certain order. The examples in \cref{sec:examples}%
\unless\ifthesis%
, the \hyperref[mysupplement]{online Supplement},
\fi
and the
\href{https://mmbosschaert.github.io/MatCont7p2NewInitBTHom-/}{online Jupyter
Notebook}, but also the comparison in
\cref{sec:comparison_homoclinic_predictors}, confirm that we indeed have
obtained the correct transformation.

The additional non-linear transformation
\cref{eq:second_time_transformation} greatly simplifies the computation
of the coefficients in the Lindstedt-Poincar\'e method since all calculations
become essentially polynomial, which is ideal for computers to work with.
Nonetheless, the algorithmic complexity grows exponentially as the order
increases linearly. Also, the radius of convergence is clearly finite, as shown
in \cref{sec:case_study_BT2}. One way to increase the convergence radius is by
using transformations as in~\cite{Milton@1974}. However, we didn't include any
results in this direction since it would distract too much from our main
objectives.

Using different phase conditions can improve the accuracy of the homoclinic
approximation. However, this only holds true when applied directly to the system
considered. Indeed, the phase condition isn't invariant under the
parameter-dependent center manifold transformation. Thus, its applicability is
very limited. Furthermore, using a different phase condition may, somewhat
unexpectedly, result in difficult integrals to be solved, see
\cref{sec:RPM_norm_minimizing_phase condition}.

The higher-order approximation to the non-linear time transformation in the
Lindstedt-Poincar\'e method turns out to be essential to obtain higher-order
approximations to the homoclinic solutions. This is clearly seen by inspecting
the profiles of the homoclinic solution in
\cref{fig:RP_vs_LP2016_vs_LP_profiles} and in the convergence plot in
\cref{fig:RP_vs_LP2016_vs_LP}. Without the higher-order approximation the same
convergence order as the unperturbed Hamiltonian solution, i.e., the
zeroth-order solution. It should be noted that the higher-order approximation
of the non-linear time transformation is more difficult to obtain. Therefore,
we conclude that there seem to be \emph{no benefits} of the
Lindstedt-Poincar\'e method over the regular perturbation method for starting
continuation of homoclinic orbits. Indeed, the numerical comparisons in
\cref{sec:examples} show similar accuracy of convergence at each order.

By comparing the convergence order of Lindstedt-Poincar\'e the with the regular
perturbation method, we see that contrary to what one might expect, the
regular perturbation method may result in better accuracy at the same order. A
possible explanation for this might be that although the Lindstedt-Poincar\'e
method provides a uniform approximation in time, the numerical solution is
truncated to a finite interval in which the `parasitic turn' doesn't give a
significant contribution. After all, it then simply depends on the higher-order
non-linear terms in the system which favor one method over the other. 

