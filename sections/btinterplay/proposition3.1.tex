\section{Proof of \texorpdfstring{\Cref{proposition:symmetry}}{Proposition 3.1}}
\label{sec:proof:symmetry_proposition}
\begin{proposition}
    Assume that the perturbed Hamiltonian system
    \cref{eq:second_order_nonlinear_oscillator} is obtained from the normal form
    \cref{eq:normal_form_orbital} by the singular rescaling
    \cref{eq:blowup}. Then the non-trivial branch of homoclinic solutions 
    \[
        (u(s,\epsilon), \dot u(s, \epsilon), \epsilon, \tau(\epsilon)),
            \qquad \epsilon<|\epsilon_0|,
    \]
    for some $\epsilon_0>0$ satisfies
    \begin{equation}
        \tau(\epsilon) = \tau(-\epsilon).
    \end{equation}
    Furthermore, if the solutions $u_i$ are even functions for $i$ even, then 
    the solution $u$ contains the additional symmetry
    \[
        u(s,-\epsilon) = u(-s,\epsilon)
    .\] 
\end{proposition}
\begin{proof}
    The proof follows almost entirely~\cite[Proposition 4.2]{Beyn_1994}. Thus,
    it can be shown that the transformation \cref{eq:blowup} induces the
    symmetry
    \[
        \varphi(s, D_0 (u^0, \dot u^0)^T, -\epsilon, \tau)
        =
        D_0 \varphi(-s, (u^0, \dot u^0)^T, \epsilon, \tau)
    \] 
    on the flow $\varphi$ of \cref{eq:second_order_nonlinear_oscillator}, where 
    \[
        D_0 = \begin{pmatrix} 1 & ~0 \\  0 & -1 \end{pmatrix}.
    \] 
    From this relation one can then conclude that if $(u(s,\epsilon), \dot
    u(s,\epsilon), \epsilon, \tau(\epsilon))$ is a homoclinic solution to
    \cref{eq:second_order_nonlinear_oscillator} then so is $(\tilde
    u(-s,\epsilon), -\dot{\tilde{u}}(-s,\epsilon), -\epsilon, \tau(\epsilon))$.
    The proof in~\cite{Beyn_1994} then finishes with the remark that these two
    homoclinic solutions must be equal by the uniqueness of the non-trivial
    branch. However, by the \emph{non-uniqueness} of the non-trivial branch, we
    obtain the relation
    \[
        (u(s,\epsilon), \epsilon, \tau(\epsilon))
        =
        (u(-s,-\epsilon), \epsilon, \tau(-\epsilon))
        +
        (\gamma(\epsilon) \dot u_0(s), \epsilon, 0),
    \]
    where
    \begin{equation}
        \label{eq:gamma}
        \gamma(\epsilon) = \sum_{i\geq 1} \gamma_i \epsilon^i,
        \qquad
        \gamma_i \in \mathbb R.
    \end{equation}
    Thus, $\tau$ is indeed an even function of $\epsilon$. Using the expansion
    for $u$ from \cref{eq:u_i_tau_i_RPM} we see that by inspecting the
    coefficients of equal powers in $\epsilon$ we only need to impose that
    $\gamma_i=0$ for $i$ even, then the assertion follows.
\end{proof}
