\section{Homoclinic asymptotics in the orbital and smooth normal form}
\label{sec:asymptotics}

In this section, we derive asymptotics to the homoclinic solution near the
generic codimension two Bogdanov--Takens point for the smooth orbital normal
form \cref{eq:normal_form_orbital} and smooth normal form
\cref{eq:BT_smooth_nf}. 

We start by revisiting the standard regular perturbation method with the smooth
orbital normal form, but with a different phase condition. In
\cref{sec:third_order_homoclinic_approximation_RP} we provide the explicit
third-order asymptotics obtained using this method and phase condition. It will
be shown in \cref{sec:examples} that this improves the accuracy of the
homoclinic approximation in the normal form. Then, in
\cref{sec:PolynomailLindstedtPoincare} we revisited the Lindstedt-Poincar\'e
method. By an additional non-linear time transformation, we obtain a very
simple algorithm to approximate the homoclinic solution. In
\cref{sec:third_order_homoclinic_approximation_LP} we provide an explicit
third-order homoclinic approximation in the perturbed Hamiltonian system,
obtained from the smooth orbital normal form, using the algorithm described in
\cref{sec:PolynomailLindstedtPoincareMethod}. Then, by expanding these
asymptotics, we obtain the third-order approximation using the regular
perturbation method to \cref{eq:second_order_nonlinear_oscillator} with phase
condition $\dot u(0)=0$. In \cref{sec:phase_condition} we show that the
solutions obtained using Lindstedt-Poincar\'e method are non-unique as well. By
exploiting the non-uniqueness, we show how to obtain a more accurate
third-order predictor for the normal form \cref{eq:universal_unfolding}. Next,
using again the algorithm described in
\cref{sec:PolynomailLindstedtPoincareMethod}, we combine the
Lindstedt-Poincar\'e method with the smooth normal form to obtain a third-order
homoclinic predictor. Note that, here, in contrast to \cite{Al-Hdaibat2016}, we
also provide higher-order approximations to the reparametrization of time. The
profiles of the homoclinic solution will only then be approximated accurately
as, demonstrated in \cref{sec:examples}. Lastly, by expanding these
asymptotics, we recover the third-order homoclinic predictor obtained in
\cite{Al-Hdaibat2016} from combining the regular perturbation method with the
smooth normal form \cref{eq:BT_smooth_nf} and phase condition $\dot u(0)=0$.

Thus, in total there are six different third-order homoclinic asymptotic
presented in this section. A detailed comparison is between these asymptotics
is given in \cref{sec:examples}, and the \hyperref[mysupplement]{online
Supplement}.

\subsection{The Regular Perturbation Method with norm minimizing phase condition}
\label{sec:RPM_norm_minimizing_phase condition}
For $\epsilon=0$, \cref{eq:second_order_nonlinear_oscillator} is a Hamiltonian
system with the first integral
\begin{equation*}
    H(u,\dot u) = \frac12 \dot u^2+4u-\frac13 u^3.
\end{equation*}
The Hamiltonian system has the well-known explicit homoclinic solution
$(u_0(s),\dot u_0(s))$ given by
\[
    u_{0}(s) = 6 \tanh^2(s) - 4.
\]
Thus, for $(u, \epsilon, \tau) = (u_0, 0, \tau)$ there exists a trivial branch
of homoclinic orbits in \cref{eq:second_order_nonlinear_oscillator}.
In~\cite{Beyn_1994} it is shown that there exists a bifurcation point at
$\tau=\frac{10}{7}$ from which a smooth non-trivial branch emanates
transversally. Parametrizing this branch by $\epsilon$, we formally have
\begin{equation}
    \label{eq:u_i_tau_i_RPM}
    u(s,\epsilon) = \sum_{i\geq 0} u_i(s) \epsilon^i, 
    \qquad 
    \tau(\epsilon) = \sum_{i\geq 0} \tau_i \epsilon^i.
\end{equation}
Substituting \cref{eq:u_i_tau_i_RPM} into
\cref{eq:second_order_nonlinear_oscillator} and collecting equal terms in
$\epsilon$ yields the following differential equations to be solved:
\begin{align}
  &\ddot u_0 - u_0^2 +4 = 0, \label{eq:z0} \\
  &\ddot u_i - 2 u_0 u_i
    =
    z_i, 
        \label{eq:regular_perturbation_method_equation_epsilon_i} \\
  &\dot u_i (\pm \infty) = \ddot u_i (\pm \infty) = 0,
        \qquad i\in \mathbb N. \nonumber
\end{align}
Here $z_i$ dependents on the sums and products of $u_j, \dot u_j$ and $\tau_{j-1}$
for $0\leq j < i$.  Multiplying equation
\cref{eq:regular_perturbation_method_equation_epsilon_i} by $\dot u$ and
integrating from $s_0$ to $s$ yields,
\[
  \int_{s_0}^s \dot u_0 \ddot u_i - 2 \dot u_0 u_0 u_i \;dx = \int_{s_0}^s \dot
  u_0 z_i \; dx.
\]
Using integration by parts twice then gives
\begin{equation}
  \label{eq:regular_perturbation_method_integration_by_parts}
  \left. \left( \dot u_0 \dot u_i - u_i \ddot u_0 \right)\right|_{s_0}^s = \int_{s_0}^s \dot
  u_0 z_i \; dx.
\end{equation}
Notice that solutions $(\dot u_i(s),\ddot u_i(s))$ must vanish at plus and minus
infinity. We obtain that the coefficients $\tau_{i-1}$ are given by the condition
\begin{equation*}
    0 = \int_{-\infty}^\infty \dot u_0 z_i \; dx.
\end{equation*}
To simplify the equations that follow below we would like to
use~\cite[Proposition 4.2]{Beyn_1994}. 

However, we noticed that the proposition is not precise enough for the
conclusion to hold. Indeed, the proof relies on the uniqueness of the
non-trivial branch of homoclinic orbits. However, we see that the left-hand
side \cref{eq:regular_perturbation_method_equation_epsilon_i} is invariant
under the transformation
\[
u_i \to u_i + \gamma \dot u_0, \qquad \gamma \in\mathbb R, \; i\in\mathbb N.
\]
We, therefore, slightly modify the proposition with an additional assumption.

\begin{proposition}
    \label{proposition:symmetry}
    Assume that the perturbed Hamiltonian system
    \cref{eq:second_order_nonlinear_oscillator} is obtained from the normal form
    \cref{eq:normal_form_orbital} by the singular rescaling
    \cref{eq:blowup}. Then the non-trivial branch of homoclinic solutions 
    \[
        (u(s,\epsilon), \dot u(s, \epsilon), \epsilon, \tau(\epsilon)),
            \qquad \epsilon<|\epsilon_0|,
    \]
    for some $\epsilon_0>0$ satisfies
    \begin{equation}
        \label{eq:tau}
        \tau(\epsilon) = \tau(-\epsilon).
    \end{equation}
    Furthermore, if the solutions $u_i$ are even functions for $i$ even, then 
    the solution $u$ contains the additional symmetry
    \[
        u(s,-\epsilon) = u(-s,\epsilon)
    .\] 
\end{proposition}
\begin{proof}
    The proof follows almost entirely~\cite[Proposition 4.2]{Beyn_1994}. For
    details, see \cref{sec:proof:symmetry_proposition}.
\end{proof}

It, therefore, follows from \cref{proposition:symmetry} together with equality
\cref{eq:regular_perturbation_method_integration_by_parts} that the condition for
solving $\tau_{i-1}$ in \cref{eq:tau_condition_RP} simplifies to
\begin{equation}
    \label{eq:tau_condition_RP}
    0 = \int_0^\infty \dot u_0 z_i \; dx.
\end{equation}
for $i$ even, whereas $\tau_i=0$ for $i$ odd.

From \cref{eq:regular_perturbation_method_integration_by_parts} we obtain the
solution
\begin{equation}
    \label{eq:u_i_RP}
    u_i = \dot u_0 \int \frac{1}{\dot u_0^2} \int \dot u_0 z_i \; dx \; dx,
        \qquad i \in \mathbb N,
\end{equation}
or
\begin{equation*}
    u_i = \left( \dot u_0 \int \frac{1}{\dot u_0^2} \, dx \right) \int \dot u_0 z_i(u,\dot u,\tau) \, dx
-  \dot u_0 \int \left( \int \frac{1}{\dot u_0^2} \, dx \right) \dot u_0 z_i(u,\dot u,\tau) \, dx. 
\end{equation*}
From \cref{eq:u_i_RP} we see that there are two integration constants involved.
The first integration constant, originating from the inner integral, is needed
to ensure the boundedness of the solution. The second integration constant
introduces precisely the freedom 
\begin{equation*}
    u_i \rightarrow u_i + \gamma_i \dot u_0,  \qquad i\in\mathbb N,
\end{equation*}
with arbitrary constants $\gamma_i\in\mathbb{R}$. In~\cite{Kuznetsov2014improved} the
condition
\begin{equation}
    \label{eq:RP_phase_condtion_unnatural}
    \dot u_i(0) = 0
\end{equation}
is imposed to ensure the uniqueness of the solution.  This phase condition is also
used in~\cite{Beyn_1994} in a theoretical setting. However, a more natural phase
condition would be to minimize the $L^2$-distance between the current and
previous solution obtained from the regular perturbation method. This
phase condition is also used in~\cite{Doedel1986auto, Champneys1996,
DeWitte2012,Doedel@1989} for numerical continuation of heteroclinic and
homoclinic orbits. Using the $L^2$ phase condition yields
\begin{equation*}
  \int_{-\infty}^\infty 
        \langle 
            (\dot u_0(s), \ddot u_0(s)) (u_{i}(s) + \gamma_i \dot u_0(s),
            \dot u_{i}(s) + \gamma_i \ddot u_0(s)) 
        \rangle
        \, ds = 0,
    \qquad i \in \mathbb{N}.
\end{equation*}
By \cref{proposition:symmetry} this phase condition is
equivalent to the condition that
\begin{equation}
     \label{eq:u_i_L2_phase_condition}
  \int_0^\infty 
        \langle
        (\dot u_0(s), \ddot u_0(s)) (u_{i}(s) + \gamma_i \dot u_0(s),
        \dot u_{i}(s) + \gamma_i \ddot u_0(s)) 
        \rangle
        \, ds = 0,
    \qquad i \in \mathbb{N},
\end{equation}
for $i$ odd and $\gamma_i=0$, for $i$ even, if we ensure that $u_i$ is even for $i$
even. By using integration by parts together with \cref{eq:z0} and subsequently
solving \cref{eq:u_i_L2_phase_condition} for $\gamma_i$ we obtain
\begin{equation}
    \label{eq:c_i}
    \gamma_i = -\frac{35}{2592} 
            \int_0^\infty \dot u_0(1 - 2 u_0)u_{i}(s) \, ds,
        \qquad i \in \mathbb{N}.
\end{equation}
In~\cite{Kuznetsov2014improved}, the phase condition
\begin{equation}
     \label{eq:u_i_L2_phase_condition_u_only}
     \int_{-\infty}^\infty \left( u(s) - u_0(s) \right) \dot u_0(s) \, ds
     = 0,
    \qquad i \in \mathbb N,
\end{equation}
was also tested. This phase condition only minimizes the $L_2$-distance of the
$u$-component between the current and the zeroth-order solution obtained from
the regular perturbation method. It is reported in~\cite{Kuznetsov2014improved}
that for \cref{eq:u_i_L2_phase_condition_u_only} no substantial superiority
over using phase condition \cref{eq:RP_phase_condtion_unnatural} was found.
Our findings show that, at least for \cref{eq:u_i_L2_phase_condition}, this is
only partially true.  Indeed, the numerical simulations in
\cref{sec:topological_normal_form} show that, as one would expect, using the
phase condition \cref{eq:u_i_L2_phase_condition} does indeed improve the
approximation to the homoclinic orbit. However, when the homoclinic
approximations are lifted from the normal form to the center manifold of the
Bogdanov--Takens point, the phase conditions are, in general, not preserved, and
the improvements are no longer observed.

As we will see below, the $L_2$ phase condition
\cref{eq:u_i_L2_phase_condition} is more difficult to solve. It is, therefore,
more efficient to use the orbital normal form \cref{eq:universal_unfolding}
instead of the smooth normal form \cref{eq:BT_smooth_nf}.

\subsubsection{Third-order homoclinic approximation}
\label{sec:third_order_homoclinic_approximation_RP}
For $i=1$ we obtain the equation
\begin{equation*}
    z_1(s) = (u_0(s)+\tau_0) \dot u_0(s).
\end{equation*}
Condition \cref{eq:tau_condition_RP} yields
\begin{equation*}
    \tau_0 = \frac{10}{7}.
\end{equation*}
Then from \cref{eq:u_i_RP} we obtain the solution
\begin{equation*}
    u_1(s) = -\frac{6}{7} \dot u_0(s) \log (\cosh (s)).
\end{equation*}
The $L_2$ phase condition then yields that
\[
\gamma_1 = -\frac{3}{245} (70 \log (2)-59)
.\] 
Note that the integral to be evaluated in \cref{eq:c_i} is labor-intensive and
prone to error.  Therefore, we used the (freely available) Wolfram
Engine~\cite{WolframEngine} (although not open source).  Correcting the
previous solution $u_1$ leads to the solution
\begin{equation*}
    u_1(s) = \frac{3}{245} (59-70 \log (2 \cosh (s))) \dot u_0(s).
\end{equation*}

Continuing with the second-order system we have the equation
\[
    z_2 = (u_0+\tau_0) \dot u_1+u_1 \dot u_0+u_1^2.
\] 
Here we directly used that $\tau_1=0$ by the symmetry as explained above.
From \cref{eq:u_i_RP} we obtain
\begin{align*}
    u_2(s) ={}& 
    \frac{1}{60025} 36 \sech^2 s  \left[3 \sech^2 s  \left\{70 \log (2 \cosh  s )
        (105 \log (2 \cosh s )-247) \right.\right. \\ 
              &{} \left.\left. +6289\right\}-2(3675 s \tanh  s +210 \log (2 \cosh  s )
              (35 \log (2 \cosh  s )-94)+7129)\right].
\end{align*}
Notice that, since $u_2$ is an even function, we automatically have that
$\gamma_2$ vanishes.

For $i=3$ we have
\begin{equation*}
    z_3 = (u_2+\tau _2) \dot u_0+(u_0+\tau _0) \dot u_2+u_1 \left(\dot u_1+2 u_2\right).
\end{equation*}
Condition \cref{eq:tau_condition_RP} yields
\begin{equation*}
    \tau_2 = \frac{288}{2401}.
\end{equation*}

Then from \cref{eq:u_i_RP} we obtain the solution
\begin{align*}
    u_3(s) ={}&
    \frac{216 \sech^2 s}{14706125}  
    \left[\sech^2 s  \left\{3675 s (210 \log (2 \cosh  s )-247) \right. \right. \\
              & +\tanh  s  \left(-171500 (\cosh (2 s)- 5) \log ^3(2 \cosh  s )+7350 (129 \cosh (2 s)-470) \right. \\
              & \left.\left. \log^2(2 \cosh  s )+4456830 \log (2 \cosh s)-966242\right)\right\} \\
              & \left.-70 \{210 s (35 \log (2 \cosh  s )-47)+30673 \tanh  s \log (\cosh  s )\}\right]
.\end{align*}

Trying to solve the integral in \cref{eq:c_i} with the Wolfram Engine yields
\begin{align}
    \label{eq:c_i_integral}
    \int_0^\infty \dot u_0(1 - 2 u_0)u_3(s) \, ds ={}& 
    \frac{16 \left(-5234558923+331676100 \pi^2+6260972760 \log (2)\right)}{514714375} \\
    &{} -\frac{155520}{343}\int_0^\infty 
                \log^3(2\cosh s )\sech^6 s \tanh^2 s  \, ds \nonumber \\ 
    &{}  \frac{31104}{343}
                \int_0^\infty \log^3(2\cosh s )\sech^6 s \tanh^2 s \cosh(2s) 
                \, ds \nonumber \\
    &{}  \frac{622080}{343}
            \int_0^\infty \log^3(2\cosh s )\sech^8 s \tanh^2 s  
            \, ds \nonumber \\ 
    &{} -\frac{124416}{343}
            \int_0^\infty \log^3(2\cosh s )\sech^8 s \tanh^2 s \cosh(2s) 
            \, ds, \nonumber
\end{align}
i.e., the Wolfram Engine was unable to solve the integral. We observe that, in
order to solve \cref{eq:c_i_integral}, it is sufficient to solve integrals of
the form
\begin{equation}
    \label{eq:I_n}
    I_n := \int_0^{\infty} \log^3 (2 \cosh s) \sech^n s \, ds,
\end{equation}
with $n=4,6,8$ and $n=10$. After a lengthy calculation, see \cref{sec:I_n}, we
obtain the closed form expression
\begin{multline*}
I_n = 2^{n-3} 3 \sum_{k=0}^{\frac{n}{2}-1} \binom{\frac{n}{2}-1}k 
            (-1)^k  \\
        \left[\frac1{(\frac{n}{2}+k)^4} +
            \frac{8}{2k+n} \left(\frac{H_{\frac{n}{2} + k}}{(2k+n)^2}
                + \frac{H^{(2)}_{\frac{n}{2}+k} - \zeta(2)}{2(2k+n)}
            + \frac{H^{(3)}_{\frac{n}{2}+k} - \zeta(3)}{4}
        \right)\right].
\end{multline*}
where $\zeta$ is the Riemann zeta function and $H_n^{(m)}$ is the $n$th
generalized harmonic number of order $m$. Explicitly we obtain
\begin{align*}
    I_4 ={}& \frac{82}{27}-\frac{5 \pi ^2}{36}-\zeta (3), \\
    I_6 ={}& \frac{38342}{16875}-\frac{47 \pi ^2}{450}-\frac{4 \zeta (3)}{5}, \\
    I_8 ={}& \frac{25545482}{13505625}-\frac{319 \pi ^2}{3675}-\frac{24 \zeta (3)}{35}, \\
    I_{10} ={}& \frac{5428830032}{3281866875}-\frac{7516 \pi ^2}{99225}-\frac{64 \zeta (3)}{105}.
\end{align*}
From the above integrals, we deduce that
\begin{equation*}
    \gamma_3 = \frac{264 \zeta(3)}{343}-\frac{884895199}{7147176750}
            -\frac{100 \pi^2}{3087}-\frac{1104228 \log 2}{420175}.
\end{equation*}
Since, by the symmetry, $\tau_3=0$, we obtain the third-order homoclinic predictor 
\begin{equation}
\label{eq:third_order_predictor_RPM_tau}
\begin{cases}
\begin{aligned}
w_0(\eta)  &= \frac{a}{b^2} \left( \sum_{i=0}^3 u_i(\frac{a}{b}\epsilon\eta) \epsilon^i +
\mathcal{O}(\epsilon^4) \right)   \epsilon^2, \\
w_1(\eta)  &= \frac{a^2}{b^3} \left( \sum_{i=0}^3 \dot u_i(\frac{a}{b}\epsilon\eta) \epsilon^i +
\mathcal{O}(\epsilon^4) \right)   \epsilon^3, \\
\beta_1    &= -4 \frac{a^3}{b^4}\epsilon^4, \\
\beta_2    &= \frac{a}{b}\epsilon^2 \left( \frac{10}{7} + \frac{288}{2401} \epsilon^2 + \mathcal{O}(\epsilon^4) \right),
\end{aligned}
\end{cases}
\end{equation}
for the smooth orbital normal form \cref{eq:normal_form_orbital}.


\begin{remark}
In~\cite{Kuznetsov2014improved} there is the remark that the
author~\cite{Beyn_1994} was unable to find a tangent predictor due to the
(normalized) form. The system in~\cite[Equation (4.5)]{Beyn_1994} to be solved
is given by
\begin{equation}
  \label{eq:Beyn_tangent_system}
  \begin{cases}
  \begin{aligned}
  \dot x  - y &= \frac12 a_1 \hat x^2 + a_2 \tau_0 \hat x + a_3 \tau_0^2, \\
  \dot y -2\hat x x &=  b_1 \hat x \hat y + b_2 \tau_0 \hat y,
  \end{aligned}
  \end{cases}
\end{equation}
here $(\hat x, \hat y)$ is the zeroth-order solution $(u_0, \dot u_0)$. The
coefficients $a_1,a_2,b_1,b_2$ are different normal form coefficients than used
in this \paper{}, but $\tau_0$ is identical to the $\tau_0$ used in this \paper.
Using the same technique as above it is easy to derive that
\begin{align*}
x(s) ={}& \left(a -\frac{1}{16} a_3 s  \tau _0^2-\frac{1}{96} a_3 \tau _0^2
	\sinh (2 s ) -\frac{1}{48} a_2 \tau _0 \sinh (2 s )+\frac{1}{24} a_3 \tau _0^2
	\coth (s ) \right . \\ 
					&-\frac{1}{24} a_2 \tau _0 \coth (s )+\frac{a_1 s
	}{8}-\frac{1}{48} a_1 \sinh (2 s ) +\frac{1}{12} a_1 \coth (s )+\frac{1}{15}
	b_2 \tau_0 \cosh (2 s ) \\ 
	&+\frac{1}{240} b_2 \tau _0 \cosh (4 s
	)-\frac{2}{21} b_1 \cosh (2 s )-\frac{1}{168} b_1 \cosh (4 s ) \\ 
	&\left.  -\frac{6}{7} b_1 \log (\cosh (s )) \right) \hat y(s)
\end{align*}
is a solution to \cref{eq:Beyn_tangent_system}. Therefore,
by~\cite{Keller1977} the convergence of the homoclinic solution in
the perturbed Hamiltonian system follows. Although the author
in~\cite{Beyn_1994} was unable to provide a tangent approximation in
phase-space, the author did prove, by refining the convergence cones
from~\cite{JepsonDecker1986}, the convergence of the zeroth-order approximation in
phase-space in the perturbed Hamiltonian system.
\end{remark}


\subsection{A polynomial Lindstedt-Poincar\'e method}
\label{sec:PolynomailLindstedtPoincare}

The Lindstedt-Poincar\'e method considers a nonlinear time transformation
defined implicitly through the relation 
\begin{equation}
  \label{eq:first_non_linear_time_transformation}
  \frac{d\xi}{ds}=\omega(\xi),
\end{equation}
which can be used to remove the so-called \emph{secular terms}, i.e., terms
growing without bound, appearing in the process of approximating periodic
orbits in weakly nonlinear oscillators using the regular perturbation approach.

The Lindstedt-Poincar\'e method is also used to approximate homoclinic
solutions in nonlinear oscillators, referred to as the generalized
Lindstedt-Poincar\'e method,
see~\cite{Chen2009,Chen@2009A,Chen@2009B,Chen@2010,Chen@2012}. In this case,
there are no terms growing without bound when applying the regular perturbation
approach. Instead, there are so-called \emph{parasitic turns}, see~\cite[Figure
1]{Kuznetsov2014improved}.  The nonlinear transformation
\cref{eq:first_non_linear_time_transformation} can then be used to remove the
parasitic turns. In fact, using the nonlinear transformation, one can obtain a
very simple form for the solution of the homoclinic orbit in phase-space,
see~\cite[Equation 35]{Chen2009} and~\cite{Algaba_2019}.

In both cases, i.e., when approximating periodic orbits or homoclinic orbits,
we do the same: a nonlinear time transformation is used to obtain a uniform
approximation of the orbit in time. 

\subsubsection{General polynomial method}
\label{sec:PolynomailLindstedtPoincareMethod}
Substituting the parameterization of time $\omega$
\cref{eq:first_non_linear_time_transformation} into
\cref{eq:second_order_nonlinear_oscillator} yields
\begin{equation}
\label{eq:secondorder_gBT_omega}
\omega\dfrac{d}{d\xi}\left(\omega \hat u^\prime\right)-\hat u^2+4
  =\epsilon\omega \hat u^\prime (\hat u+\tau) + \mathcal O(\epsilon^4),
\end{equation}
were $\hat u(\xi(s)) = u(s)$.

We now introduce a new time $\zeta$ via the transformation
\begin{equation}
  \label{eq:second_time_transformation}
  \frac{d\zeta}{d\xi} = 1 - \zeta^2
\end{equation}
to simplify the solutions obtained below. Note that this transformation implies
that $\zeta = \tanh(\xi + c_1)$, where $c_1$ is some constant. Without loss of
generality, we can assume that $c_1=0$ since $c_1$ just shifts the homoclinic
solution in time. Substituting \cref{eq:second_time_transformation} into
\cref{eq:secondorder_gBT_omega} yields
\begin{equation}
\label{eq:secondorder_gBT_zeta}
(1 - \zeta^2)\tilde\omega\dfrac{d}{d\zeta}\left((1-\zeta^2)\tilde\omega \tilde
u^\prime\right)-\tilde u^2+4
  =\epsilon\tilde\omega (1-\zeta^2)\tilde u^\prime(\tilde u+\tau) +\mathcal{O}(\epsilon^4),
\end{equation}
where the prime ${}^\prime$ now represents the derivative with respect to the
variable $\zeta$, $\tilde u(\zeta(\xi(s))) = u(s)$, and $\tilde \omega
(\zeta) = \omega(\xi(\zeta))$.

Expanding $\tilde u$, $\tilde \omega$ and $\tau$ in $\epsilon$ 
\begin{equation}
  \label{eq:u_tilde_expansion_tau_expansion}
  \tilde u(\zeta) = \sum_{i=0} \tilde u_i(\zeta) \epsilon^i,  \qquad
  \tilde \omega(\zeta) = \sum_{i=0} \tilde \omega_i(\zeta) \epsilon^i,  \qquad
  \tau = \sum_{i=0} \tau_i \epsilon^i,
\end{equation}
substituting into \cref{eq:secondorder_gBT_omega}, and collecting terms of equal
power in $\epsilon$, we obtain the following systems to be solved:
\begin{align}
  &(1-\zeta^2)\tilde\omega_0
    \left((1-\zeta^2)\tilde\omega_0 \tilde u_0^\prime\right)^\prime-\tilde u_0^2 +4 
        {}= 0, \label{eq:0th_order_equation} \\
  &(1-\zeta^2)\left((1-\zeta^2) \tilde u_i^\prime\right)^\prime-2 \tilde u_0
  \tilde u_i   
    + 2 (1 - \zeta^2)\tilde\omega_i \left( (1 - \zeta^2) \tilde u_0^\prime \right)^\prime 
    + (1 - \zeta^2)^2 \tilde u_0^\prime \tilde\omega_i^{\prime}
    \label{eq:ith_order_equation} \\ 
  &\qquad = \tau_{i-1} \left(1- \zeta^2\right) \tilde u_0^\prime 
         + z_i, \qquad i \in \mathbb{N}. \nonumber
\end{align}
Here $z_i$ contains the sums and products of terms in $\tilde u_j,\tilde
\omega_j$ and $\tau_{j-1}$ with $0 \leq j \leq i-1$, with $\tau_{-1}$ is
defined to be zero.

\begin{theorem}
  \label{thm:i_order_solution}
  Equations \cref{eq:0th_order_equation,eq:ith_order_equation} are
  solvable for every $i\in\mathbb N_{0}$, with
  \begin{equation}
    \label{eq:u_i_form}
    \tilde u_i(\zeta) = \sigma_i \zeta^2 + \delta_i,
  \end{equation}
	where $\sigma_i$ and $\delta_i$ are constants to be determined.
\end{theorem}
\begin{proof}
It is easy to see that equation \cref{eq:0th_order_equation} is solvable with
$\sigma_0=6$, $\delta_0=-4$, and $\tilde\omega_0(\zeta) = 1$.

Assume that for $i=1,\dots,n-1$, the systems given by
\cref{eq:ith_order_equation} are solvable for $\tilde\omega_i$ and $\tilde u_i$.
Furthermore, also assume that for $i=1,\dots,n$, $\tilde u_i$ is of the form
\cref{eq:u_i_form}.  We will show that the system \cref{eq:ith_order_equation}
with $i=n$ is solvable for $\tilde \omega_n$.

First notice that \cref{eq:ith_order_equation} is just a first order ordinary
differential equation in $\tilde \omega_i$:
\begin{equation}
\label{eq:omegap_i_as_ode}
  \tilde\omega_i^{\prime} + \frac{2 \left( (1 - \zeta^2) \tilde u_0^\prime \right)^\prime}
				{(1 - \zeta^2) \tilde u_0^\prime} \tilde\omega_i
  = \frac{2 \tilde u_0\tilde u_i - (1-\zeta^2)\left((1-\zeta^2)\tilde u_i^\prime\right)^\prime
    + \tau_{i-1} \left(1- \zeta^2\right) \tilde u_0^\prime + z_i}
				{(1 - \zeta^2)^2 \tilde u_0^\prime}.
\end{equation}
Multiplying by the integrating factor 
\begin{equation}
    \label{eq:integrating_factor}
    (1-\zeta^2)^2\left( \tilde u_0^\prime \right)^2
\end{equation}
and subsequently integrating with respect to $\zeta$ yields the
identity
\begin{align}
  \tilde\omega_i  &=  \frac{ (1-\zeta^2)\left((1-\zeta^2)\tilde u_0^\prime\right)^\prime \tilde u_i 
      - (1-\zeta^2)^2\tilde u_0^\prime \tilde u_i^\prime + (g_i(\zeta)-g_i(1)) }
          { \left( \left(1 -\zeta^2\right) \tilde u_0^\prime \right)^2} \nonumber \\
        &=  -\frac{\sigma_i}{12} 
            - \frac{ (1-\zeta^2)\left((1-\zeta^2)\tilde u_0^\prime\right)^\prime \tilde u_i 
            + (g_i(\zeta)-g_i(1)) }
          { \left( \left(1 -\zeta^2\right) \tilde u_0^\prime \right)^2}
          \label{eq:omega_i_explicit}
\end{align}
where
\begin{align*}
  g_i(\zeta) &= \tau_{i-1} \int \left(1- \zeta^2\right) \left(
               \tilde u_0^\prime \right)^2 \; d\zeta + \int \tilde u_0^\prime z_i \; d\zeta.
\end{align*}
Here we used the identity 
\[
\left((1-\zeta^2)\left((1-\zeta^2)\tilde u_0^\prime\right)^\prime \right)' 
=2 \tilde u_0 \tilde u_0^\prime,
\]
obtained from differentiating equation \cref{eq:0th_order_equation} and then
using integrating by parts.  Furthermore, we have chosen the integration
constant $g_i(1)$ such that numerator in \cref{eq:omega_i_explicit} vanishes
for $\zeta=1$. Indeed, for $\tilde\omega_i$ to be well-defined, the numerator in
\cref{eq:omega_i_explicit} must have roots of at least multiplicity two at
$\zeta=0$ and $\zeta=\pm 1$. By setting $\zeta=-1,0$ in the numerator of
\cref{eq:omega_i_explicit}, we obtain the equations
\begin{align}
  0 & = g_i(-1) - g_i(1), \\
  0 & = 12 \delta_i - (g_i(0) - g_i(1)),
\end{align}
respectively. The first equation can be solved explicitly for $\tau_i$. Since
$g_i(0)=0$, it follows that $\delta_i = \frac{g_i(1)}{12}$. To show that the
roots $\zeta=0$ and $\zeta=\pm1$ have multiplicity two, we notice that
differentiation of the numerator in \cref{eq:omega_i_explicit} with respect to
$\zeta$ is equal to multiplying the right-hand side of \cref{eq:omegap_i_as_ode}
with the integrating factor
\cref{eq:integrating_factor}, i.e.,
\begin{equation}
    \label{eq:vanishing_condition_for_double_root}
    \tilde u_0' \left( 2 \tilde u_0\tilde u_i - (1-\zeta^2)\left((1-\zeta^2)\tilde u_i^\prime\right)^\prime
    + \tau_{i-1} \left(1- \zeta^2\right) \tilde u_0^\prime - z_i \right).
\end{equation}
Since $\tilde u_0^\prime = 12 \zeta$, we can factor out $\zeta=0$. Then substituting
$\zeta=\pm1$ into \cref{eq:vanishing_condition_for_double_root}, the following
equation needs to be satisfied
\begin{equation*}
    2 \tilde u_0(\pm1)\tilde u_i(\pm1) + z_i(\pm1) = 0.
\end{equation*}
Notice that this condition is equivalent to the condition obtained by
substituting $\zeta=\pm1$ into \cref{eq:ith_order_equation}. Therefore, by
solving the above equation for either $\pm 1$, yields
\begin{equation}
    \label{eq:sigma_i}
    \sigma_i = -\delta_i - \frac{z_i(1)}{4}.
\end{equation}

Lastly, notice that for $i=1$ we have the solution
\[
\tau_0 = \frac{10}{7}, \quad \sigma_1=0, \quad \delta_1=0,
				\quad \tilde\omega_1(\zeta) = \frac{6}{7} \zeta.
\]
\end{proof}

\begin{corollary}
    \label{corllary:rational_coefficients}
    For $i\in \mathbb N_0$ the polynomials $\tilde \omega_i$ 
    \cref{eq:omega_i_explicit} have rational
    coefficients. Also, the $\tau_i, \sigma_i$ and  $\delta_i$ are rational.
\end{corollary}

\begin{proof}
    The proof follows from a simple induction argument taking into account to
    structure of $z_i$, $i\in\mathbb N$, in \cref{eq:ith_order_equation}.
\end{proof}

\begin{corollary}
\label{corollary:delta_i_sigma_i}
The following relation holds
\[
\sigma_i = \delta_i = \tau_i = 0, \qquad \text{for $i$ odd}.
\]
\end{corollary}
\begin{proof}
From \cref{proposition:symmetry} we have that the branch of non-trivial
homoclinic orbits has the following symmetry
\[
u(-s, \epsilon)=u(s, -\epsilon)
    + \gamma \dot u_0(s),
\quad \tau(\epsilon) = \tau(-\epsilon),
\]
for $s \in \mathbb{R}$ and some open neighborhood of $\epsilon=0$.  Since
$u(s, \epsilon) = \tilde u(\zeta(\xi(s))), \epsilon)$ and
\begin{equation*}
    \tilde u(\zeta(\xi(s)), \epsilon) = \sigma(\epsilon) \zeta^2(\xi(s)) +
    \delta(\epsilon),
\end{equation*}
where $\sigma(\epsilon) = \sum_i \sigma_i \epsilon^i$ and $\delta(\epsilon) =
\sum_i \delta_i \epsilon^i$, it follows that
\begin{multline*}
    \sigma(\epsilon) \zeta^2(\xi(-s)) + \delta(\epsilon) = u(-s, \epsilon)
    = u(s, -\epsilon) + \gamma \dot u_0(s) \\
    = \sigma(-\epsilon) \zeta^2(\xi(s)) + \delta(-\epsilon)
    + \gamma(\epsilon) \left[1-\zeta^2(\xi(s))\right] 12 \zeta(\xi(s)).
\end{multline*}
Therefore, $\sigma, \delta$, and $\tau$ are even functions in $\epsilon$, from
which the assertion follows.
\end{proof}


\begin{corollary}
    \label{corollary:quadraticBTsigma_delta_relation}
    For the quadratic Bogdanov--Takens normal form \cref{eq:universal_unfolding}
    we have the relation that 
    \begin{equation}
        \label{eq:relation_sigma_delta}
        \sigma_i = -\delta_i, \qquad \text{for } i\geq 1.
    \end{equation}
\end{corollary}

\begin{proof}
    Applying the singular rescaling \cref{eq:blowup} to the normal form
    \cref{eq:universal_unfolding}, and consecutive applying the nonlinear time
    transformations \cref{eq:first_non_linear_time_transformation} and
    \cref{eq:second_time_transformation}, we obtain
    \cref{eq:secondorder_gBT_zeta} without the higher-order terms in $\epsilon$.
    After some calculations we obtain the explicit expression for $z_i$ with
    $i\geq 1$ in \cref{eq:ith_order_equation}, namely
    \begin{align*}
    z_i(\zeta) = 
         \sum_{k=1}^{i-1} u_k u_{i-k} 
         + (1-\zeta^2) \left\{ \sum_{l=1}^{i-1} 
          u_l^\prime \tau_{i-1-l}
         + \sum_{k=1}^{i-1} \sum_{l=0}^{i-1-k} 
           \omega_k u_l^\prime \tau_{i-1-l-k} + \right. \\
           \left. 
           \sum_{k=0}^{i-1} \sum_{l=0}^{i-1-k}
           \omega_k u_l^\prime u_{i-1-l-k} 
         - \sum_{l=1}^{i-1}
           \omega_l \left((1-\zeta^2)
           u_{i-l}^\prime \right)^\prime
         - \sum_{k=1}^{i-1} \sum_{l=0}^{i-k}
           \omega_l \left((1-\zeta^2)\omega_k u_{i-l-k}^\prime
           \right)^\prime \right\}. \nonumber \\
    \end{align*}
    From \cref{corollary:delta_i_sigma_i}, we have that $z_1(1)=0$. By
    assuming that the relation \cref{eq:relation_sigma_delta} holds for
    $i=1,2,\dots,n-1$,  $n\in\mathbb N$, we see directly that  $z_i(1)=0$. The
    assertion now follows by \cref{eq:sigma_i}, with $i=n$.
\end{proof}

\begin{remark}
    From \cref{corollary:delta_i_sigma_i} it follows that the solution $\tilde
    u$ for the quadratic Bogdanov--Takens normal form
    \cref{eq:universal_unfolding} can be represented by the single parameter
    $\sigma$
    \begin{equation*}
        \tilde u(\zeta) = 2 - (1-\zeta^2)\sum_{i\geq 0} \sigma_i \epsilon^i.
    \end{equation*}
    Consequently, $\hat u$ becomes
    \begin{equation*}
        \hat u(\xi) = 2 - \sech^2(\xi) \sum_{i\geq 0} \sigma_i \epsilon^i.
    \end{equation*}
\end{remark}


\subsubsection{Third-order orbital homoclinic approximation}
\label{sec:third_order_homoclinic_approximation_LP}
For the third-order homoclinic predictor we obtain
\begin{align}
				\sigma &= 6 + \frac{18}{49}\epsilon^2 + \mathcal{O}(\epsilon^4),
				\nonumber \\
				\delta &=-4 - \frac{18}{49}\epsilon^2 + \mathcal{O}(\epsilon^4),
				\nonumber \\
				\label{eq:tau_orbital}
				\tau   &= \frac{10}{7} + \frac{288}{2401} \epsilon^2 +
								\mathcal{O}(\epsilon^4), \\
				\tilde \omega(\zeta) &= 1 - \frac{6}{7}\zeta \epsilon	+ 
				\left(\frac{9}{98} + \frac{27}{98}\zeta^2 \right) \epsilon^2 +
				\left( -\frac{198}{2401}\zeta + \frac{18}{343}\zeta^3 \right)\epsilon^3+
								\mathcal{O}(\epsilon^4). \nonumber
\end{align}
From which it follows that
\begin{align}
    \label{eq:third_order_uhat}
    \tilde {u}(\zeta) 
    ={}& 2 - \left(1-\zeta^2\right) \left(6 + \frac{18}{49}\epsilon^2 \right) 
        + \mathcal{O}(\epsilon^4), \\
    \label{eq:third_order_vhat}
	\tilde  v(\zeta) 
  ={}& -2 \tilde\omega(\zeta) \sigma (1-\zeta^2)\zeta
	= -\left[ -12 + \frac{72}{7} \zeta \epsilon
			 - \left( \frac{90}{49} + \frac{162 }{49} \zeta^2 \right)\epsilon^2
		 \right. \\
		 & + \left. \left( \frac{3888}{2401} \zeta -
     \frac{216}{343}\zeta^3 \right) \epsilon^3 \right]
     (1-\zeta^2) \zeta + \mathcal{O}(\epsilon^4) \nonumber.
\end{align}

The relation $\xi(s)$ is obtained by solving the ODE
\begin{equation}
				\label{eq:third_order_dxi_ds}
				\frac{d\xi}{ds}(s) = \tilde \omega(\tanh(\xi(s))).
\end{equation}
Thus, we substitute 
\begin{equation*}
				\xi(s) = s + \xi_1(s)\epsilon + \xi_2(s)\epsilon^2
				+ \xi_3(s)\epsilon^3 + \mathcal{O}(\epsilon^4),
\end{equation*}
into \cref{eq:third_order_dxi_ds} and expand the resulting equation in
$\epsilon$ to obtain
\begin{align}
				\frac{d\xi_1}{ds}(s) &= -\frac{6\tanh(s)}{7}, 
            \label{eq:xi_1} \\
				\frac{d\xi_2}{ds}(s) &= \frac{18+54\tanh^2(s)
				-168\xi_1(s)+168\tanh^2(s)\xi_1(s)}{196},
            \label{eq:xi_2} \\
				\frac{d\xi_3}{ds}(s) &= -\frac{198 \tanh(s)}{2401}+
				\frac{18 \tanh^3(s)}{343}-\frac{27}{49}
				(-\tanh(s) \xi_1(s)+\tanh^3(s) \xi_1(s)) \nonumber \\
						& \qquad-\frac{6}{7} (-\tanh(s)
				\xi_1^2(s)+\tanh^3(s) \xi_1^2(s)+\xi_2(s)-\tanh^2(s) \xi_2(s)).
            \nonumber
\end{align}
Here we directly used that $\xi_0(s)=s$. By solving these equations recursively,
we obtain
\begin{equation*}
\begin{aligned}
    \xi_1(s) ={}& c_1 - \frac{6}{7}\log(\cosh(s)), \\
    \xi_2(s) ={}& c_2 -\frac{18 s}{49}+\frac{45 \tanh (s)}{98}-\frac{6}{7} c_1
             \tanh (s)+\frac{36}{49} \tanh (s) \log (\cosh(s)), \\
    \xi_3(s) ={}& c_3 + \frac{1}{4802}\left( 3 \sech^2(s) \left(-504 \log^2(\cosh
            (s))-276 \cosh (2 s) \log (\cosh (s)) \right.\right. \\
            {}& + 102 \log (\cosh (s))+14 (18 s-49 c_2) \sinh (2 s)+1176
                c_1 \log (\cosh (s)) \\
            {}& \left. \left. + 546-686 c_1{}^2-441 c_1\right) \right).
\end{aligned}
\end{equation*}
The constants $c_i (i=1,2,3)$ lead to different phase conditions.  A
computationally simple phase condition is given by
\begin{equation}
    \label{eq:xi_i}     
    \xi_i(0)=0,\qquad \mbox{for } i=1,2,3.
\end{equation}
These results in the constraint $v(0)=0$, i.e., the phase condition used
in~\cite{Kuznetsov2014improved}. Solving \cref{eq:xi_i} leads to the solution
\begin{equation}
    \label{eq:c_i_first_phase_condition}
     c_1=0, \qquad c_2=0, \qquad c_3=-\frac{117}{343}.
\end{equation}
Substituting the above expression for $\xi$ into \cref{eq:blowup} we obtain the
third-order predictor
\begin{equation}
\label{eq:third_order_predictor_LP_tau}
\begin{cases}
\begin{aligned}
w_0(\eta)  &= \frac{a}{b^2} 
\tilde {u}\left(\tanh\left(\xi\left(\frac{a}{b}\epsilon\eta\right)\right)\right) \epsilon^2, \\
w_1(\eta)  &= \frac{a^2}{b^3}
\tilde {v}\left(\tanh\left(\xi\left(\frac{a}{b}\epsilon\eta\right)\right)\right) \epsilon^3, \\
\beta_1    &= -4 \frac{a^3}{b^4}\epsilon^4, \\
\beta_2    &= \frac{a}{b}\epsilon^2\tau,
\end{aligned}
\end{cases}
\end{equation}
where $\tau,\tilde{u}$ and $\tilde{v}$ are given by
\cref{eq:tau_orbital,eq:third_order_uhat,eq:third_order_vhat}, respectively.

\begin{remark}
By expanding $\tilde {u}\left(\tanh\left(\xi(s)\right)\right)$ in $\epsilon$ up
to third-order we obtain
\[
    u(s) = u_0(s) + u_1(s) \epsilon + u_2(s) \epsilon^2 + u_3(s) \epsilon^3,
\]
where
\begin{equation}
\label{eq:u_orbital}    
\begin{aligned}
    u_0(s) ={}& 6 \tanh^2(s) - 4, \quad
    u_1(s) = -\frac{72 b \tanh(s) \sech^2(s) \log(\cosh(s))}{7 a}, \\
    u_2(s) ={}& \frac{18}{49} \sech^2(s) \left(-12 s \tanh (s)-24 (\log(\cosh(s))-1) \log(\cosh(s)) \right. \\
              &  \left. +3 \sech^2(s) (32 \log(\sech(s))+12 \log(\cosh(s)) (\log(\cosh(s))+2)-5)+14\right), \\
    u_3(s) ={}& -\frac{27 \sech^5(s)}{2401}  \left(-273 \sinh(s)+91 \sinh(3 s)+84 s \cosh(3 s) (2 \log(\cosh(s))-1) \right. \\
              & -84 s \cosh(s) (6 \log(\cosh(s))-1)-1232 \sinh(s) \log^3(\cosh(s)) \\
              & +112 \sinh(3 s) \log^3(\cosh(s))+2016 \sinh(s) \log^2(\cosh(s)) \\
              & -336 \sinh(3s) \log^2(\cosh(s))+904 \sinh(s) \log(\cosh(s)) \\
              & \left. -104 \sinh(3 s) \log(\cosh(s))\right).
\end{aligned}
\end{equation}
Together with \cref{eq:tau_orbital}, this is precisely the solution obtained
by using the regular perturbation method to
\cref{eq:second_order_nonlinear_oscillator} with phase condition $\dot u(0)=0$.

Note that for the conjecture in~\cite[Section 7]{Al-Hdaibat2016} to
hold, the phase condition \cref{eq:xi_i} must be satisfied.
\end{remark}

\subsubsection{Non-uniqueness of homoclinic solution}
\label{sec:phase_condition}
Note that in \cref{thm:i_order_solution} we could have assumed the solutions of
\cref{eq:ith_order_equation} to be of the form
\begin{equation}
    \label{eq:u_zeta_second_form}
    \tilde u(\zeta)  = \sum_{i\geq 0} \left(\sigma_i \zeta^2 + \delta_i + \gamma_i (1-\zeta^2)
    \tilde u_0^\prime(\zeta) \right) \epsilon^i,
\end{equation}
where $\gamma_0=0$ and $\gamma_i\in\mathbb{R}$ are constants to
be determined by some phase condition. Thus, we have freedom in \cref{eq:xi_i}
and in \cref{eq:u_zeta_second_form} both originating from the non-uniqueness
of the homoclinic orbit. The solutions \cref{eq:xi_1,eq:xi_2} together with
\begin{align*}
    \xi_3(s) =&{}
    \frac{1}{4802} \left(18 \left[49 \gamma_1 (7 \gamma_1+4)+84 s \tanh (s)+92 \log
        (\sech(s))-105\right] \right. \\
        & -7 \sech^2(s) \left[-7 \gamma_1 (7 \gamma_1-3) (35 \gamma_1+9)+18 (7 \gamma_1 \
(7 \gamma_1+4)+9) \log (\sech(s)) \right. \\
        & \left.\left. +216 \log ^2(\cosh (s))-234\right] \right)
\end{align*}
and
\begin{equation*}
    \gamma_1 = \frac{1}{35} \left(-4-\frac{59}{\sqrt[3]{836+15
                \sqrt{4019}}}+\sqrt[3]{836+15 \sqrt{4019}}\right)
\end{equation*}
also leads to the phase condition $\xi_i(0)=0$ for $i=1,2,3$. However,
$v_3(0)=0$ no longer holds. The solutions $\tilde u$ and $\tilde v$ are now
given by
\begin{align*}
\tilde{u}(\zeta) 
={}& 2 + \left(1-\zeta^2\right) \left( -6 + 12 \gamma_1 \zeta  \epsilon
    + \left(6 \gamma_1^2-\frac{18}{49}\right)\epsilon^2 
    + \mathcal{O}(\epsilon^4) \right), \\
\tilde v(\zeta) 
    ={}& (1-\zeta^2) \tilde \omega (\zeta) u'(\zeta) 
    = (1-\zeta^2) \tilde \omega(\zeta) \sum_{i=0} \left( \sigma_i \zeta   +
         12 \gamma_i \left( 1 - 3 \zeta^2 \right) \right) \epsilon^i \\
	  ={}& (1-\zeta^2)\left[12\zeta + 
        \left(12\gamma_1 - 36\frac{2 + 7 \gamma_1\zeta^2}{7}\right) \epsilon +
        \nonumber \right.\\
       & 6\zeta\frac{15 - 168\gamma_1 - 245\gamma_1^2 + 
         3(9 + 7 \gamma_1(16 + 7 \gamma_1))\zeta^2}{49} \epsilon^2 + \nonumber \\
     {}& \left( 216\zeta^2\frac{-18 + 7\zeta^2}{2401} 
        + 6\gamma_1^3(-3 + 2\zeta^2 + \zeta^4) -
      72\gamma_1^2\frac{1 - 6\zeta^2 + 4\zeta^4}7  \right. \nonumber \\
     {}& \left. \left. - 54 \gamma_1\frac{-1 - 6\zeta^2 + 15\zeta^4}{49}
        \right)\epsilon^3 
        + \mathcal{O}(\epsilon^4) \right] \nonumber.
\end{align*}
The numerical simulations in \cref{sec:topological_normal_form} show that for
the normal form \cref{eq:universal_unfolding} these asymptotics are more
accurate than the asymptotics derived in
\cref{sec:third_order_homoclinic_approximation_LP}.

\ifthesis
\else
\ifthesis
\section{Asymptotics for homoclinic solution in the smooth normal form}
\else
\subsubsection{Asymptotics for homoclinic solution in the smooth normal form}
\fi
\label{sec:asymptotics-for-homoclinic-solution-for-the-smooth-normal-form}
Following the procedure outlined in \cref{sec:PolynomailLindstedtPoincare} to
the second-order nonlinear differential equation
\cref{eq:second_order_nonlinear_oscillator_smooth_normalform} obtained from the
smooth normal form \cref{eq:BT_smooth_nf}. For the third-order homoclinic
predictor we obtain
\begin{align}
				\sigma ={}& 6 + \frac{3 \left(-70 a_1 b+6 b^2+49 d\right)}{49 a^2} \epsilon^2
                    + \mathcal{O}(\epsilon^4),
				\nonumber \\
				\delta ={}& -4 + \frac{140 a_1 b-18 b^2-245 d}{49 a^2} \epsilon^2
                    + \mathcal{O}(\epsilon^4),
				\nonumber \\
				\label{eq:tau_smooth}
				\tau   ={}& \frac{10}{7} + \frac{98 b (50 a b_1+73 d)-9604 a e-2450 a_1 b^2+288 b^3}{2401 a^2 b} \epsilon^2
								+ \mathcal{O}(\epsilon^4), \\
				\tilde \omega(\zeta) ={}& 1 - \frac{6b}{7a}\zeta  \epsilon	+ 
				\frac{70 a_1 b+18 b^2 \left(3 \zeta ^2+1\right)+49 d \left(9 \zeta ^2-5\right)}{196 a^2} \epsilon^2 +
				\nonumber \\
        & \frac{\zeta}{2401 a^3} \left( \left(-147 b \left(20 a b_1-7 d
        \zeta ^2+11 d\right)-9604 a e \left(\zeta ^2-1\right) \right. \right. \nonumber \\
        & \left. \left. +1470 a_1 b^2+18 b^3 \left(7 \zeta ^2-11\right)\right) \right)\epsilon^3 + \mathcal{O}(\epsilon^4)
        \nonumber.
\end{align}
It follows that
\begin{align}
    \label{eq:third_order_uhat_smooth}
    \tilde {u}(\zeta) 
    ={}& 6 \zeta ^2-4 + \frac{-70 a_1 b \left(3 \zeta ^2-2\right)+18 b^2 \left(\zeta
    ^2-1\right)+49 d \left(3 \zeta ^2-5\right)}{49 a^2} \epsilon^2 
        + \mathcal{O}(\epsilon^4), \\
    \label{eq:third_order_vhat_smooth}
	\tilde  v(\zeta) 
  ={}& -2 \tilde\omega(\zeta) \sigma (1-\zeta^2)\zeta
	= -\left[ -12 + \frac{72b}{7a} \zeta \epsilon
			 - \frac{3}{49 a^2} \left(70 a_1 b-6 b^2 \left(9 \zeta ^2+5\right)
       \right. \right. \\
     & \left. -147 d \left(3 \zeta ^2-1\right)\right)\epsilon^2 + \frac{12
         \zeta}{2401 a^3}  \left(147 b \left(20 a b_1-7 d \zeta ^2+18
         d\right) \right. \nonumber \\
    & \left. \left. + 9604 a e \left(\zeta ^2-1\right)-2940 a_1 b^2-18 b^3 \
\left(7 \zeta ^2-18\right)\right) \epsilon^3 \right]
     (1-\zeta^2) \zeta + \mathcal{O}(\epsilon^4) \nonumber.
\end{align}

The relation $\xi(s)$ can be obtained by solving the ODE
\begin{equation}
				\label{eq:third_order_dxi_ds_smooth}
				\frac{d\xi}{ds}(s) = \tilde \omega(\tanh(\xi(s))).
\end{equation}
Thus, we substitute 
\begin{equation*}
				\xi(s) = s + \xi_1(s)\epsilon + \xi_2(s)\epsilon^2
				+ \xi_3(s)\epsilon^3 + \mathcal{O}(\epsilon^4),
\end{equation*}
into \cref{eq:third_order_dxi_ds_smooth} and expand the resulting equation in
$\epsilon$ to obtain
\begin{align*}
\frac{d\xi_1}{ds}(s) ={}& -\frac{6b\tanh(s)}{7a}, \\
\frac{d\xi_2}{ds}(s) ={}& \frac{-168 a b \sech^2(s) \xi_1(s)+70 a_1 b+9 \left(6 b^2+49 d\right) \tanh ^2(s)+18 b^2-245 d}{196 a^2} , \\
\frac{d\xi_3}{ds}(s) ={}& \frac{\sech^3(s)}{4802 a^3} \left(4116 a^2 b \sinh (s) (\xi_1(s))^2-4116 a^2 b \cosh (s) \xi_2(s) \right. \\
                        & +441 a \left(6 b^2+49 d\right) \sinh (s)\xi_1(s)+2 \sinh (s) \left(-3 b \cosh (2 s) \right. \\
                        & \left(98 (5 a b_1+d)-245 a_1 b+12 b^2\right)-1470 a b b_1+9604 a e+735 a_1 b^2 \\
                        & \left. \left. -162 b^3-1323 b d\right)\right).
\end{align*}
Here we directly used that $\xi_0(s)=s$. By solving these equations recursively,
we obtain
\begin{equation*}
\begin{aligned}
\xi_1(s) ={}& - \frac{6b}{7a}\log(\cosh(s)), \\
\xi_2(s) ={}& \frac{2 s \left(35 a_1 b-36 b^2+98 d\right)+9 \tanh (s) \left(16 b^2 \log (\cosh (s))+10 b^2-49 d\right)}{196 a^2}, \\
\xi_3(s) ={}& \frac{1}{4802 a^3}\left(-7 \sech^2(s) \left(1372 a e-27 b \left(6 b^2+49 d\right) \log (\cosh (s)) \right. \right. \\
            & \left. +216 b^3 \log ^2(\cosh(s))-234 b^3-147 b d\right)-5880 a b b_1 \log (\cosh (s)) \\
            & + 9604 a e+42 b s \tanh (s) \left(-35 a_1 b+36 b^2-98 d\right)+4410 a_1 b^2 \log (\cosh (s)) \\
            & \left. -1656 b^3 \log (\cosh (s))-1638 b^3+2940 b d \log (\cosh (s))-1029 b d\right).
\end{aligned}
\end{equation*}
Here we used the phase condition that $\xi_i(0)=0,i=1,2,3$. This results in the
constraint $v(0)=0$. Substituting the above expression for $\xi$ into
\cref{eq:blowup_smooth} we obtain the third-order predictor
\begin{equation}
\label{eq:third_order_predictor_LP_tau_smooth}
\ifthesis
\begin{cases}
\begin{aligned}
w_0(t)  &= \frac{1}{a} \tilde{u}\left(\tanh\left(\xi(\epsilon t)\right)\right) \epsilon^2, \\
w_1(t)  &= \frac{1}{a} \tilde{v}\left(\tanh\left(\xi(\epsilon t)\right)\right) \epsilon^3, \\
\beta_1    &= -\frac{4}{a}\epsilon^4, \\
\beta_2    &= \frac{b}{a}\tau\epsilon^2,
\end{aligned}
\end{cases}
\else
\left( w_0(t), w_1(t), \beta_1, \beta_2 \right) = 
\left( 
\frac{1}{a} \tilde{u}\left(\tanh\left(\xi(\epsilon t)\right)\right) \epsilon^2, \\
\frac{1}{a} \tilde{v}\left(\tanh\left(\xi(\epsilon t)\right)\right) \epsilon^3, \\
-\frac{4}{a}\epsilon^4, \\
\frac{b}{a}\tau\epsilon^2 \right),
\fi
\end{equation}
where $\tau,\tilde{u}$ and $\tilde{v}$ are given by
\cref{eq:tau_smooth,eq:third_order_uhat_smooth,eq:third_order_vhat_smooth},
respectively.

Note that by expanding $\tilde{u}\left(\tanh\left(\xi(s)\right)\right)$ in
\cref{eq:third_order_predictor_LP_tau_smooth} up to order three in $\epsilon$ 
we obtain
\begin{equation}
\label{eq:u_i_RP_smooth}
\begin{aligned}
    u_0(s) &= 6 \tanh ^2(s)-4, \\
    u_1(s) &= -\frac{72 b \tanh (s) \sech^2(s) \log (\cosh (s))}{7 a}, \\
    u_2(s) &= \left(12 s \sinh (2 s) \left(35 a_1 b-36 b^2+98 d\right)+8
        \cosh (2 s) \left(7 \left(5 a_1 b+9 b^2-56 d\right) \right. \right. \\
              & \; \left. -108 b^2 \log ^2(\cosh (s))+108 b^2 \log (\cosh (s))\right)
              +9 \left(35 a_1 b+192 b^2 \log^2(\cosh (s)) \right. \\
              & \; \left.\left. -96 b^2 \log (\cosh(s))-64 b^2+245 d\right)-7
              \cosh (4 s) (5 a_1 b+7 d)\right)\frac{\sech^4(s)}{196 a^2}, \\
        u_3(s) &=  \left(-2 \sinh (s) \left(\cosh (2 s) \left(-6 b \log (\cosh (s))
                        \left(-980 (a b_1+3 d)+1225 a_1 b+312 b^2\right)
                        \right.\right.\right. \\ 
                  & \; \left. +7 \left(-1372 a e+234 b^3+147 b d\right)+2016 b^3 \log^3(\cosh (s)) -6048 b^3 \log ^2(\cosh (s))\right)\\
                  & \; +6 b \log(\cosh(s)) \left(980 a b_1-1225 a_1 b+1200 b^2 -9408 d\right)+7 \left(1372 a e-234 b^3 \right. \\
                  & \; \left. \left. - 147 b d\right)-10080 b^3 \log ^3(\cosh (s)) +15120
                  b^3 \log ^2(\cosh (s))\right)  \\
                  & \; +42 b s \cosh ( 3 s)\left(35 a_1 b -36 b^2+98 d\right) (2 \log (\cosh (s)) -1) \\
                  & \; \left. +42 b s \cosh (s) \left(-35 a_1 b+36 b^2-98 d\right) (6 \log (\cosh (s))-1)\right)\frac{3 \sech^5(s)}{4802 a^3}.
\end{aligned}
\end{equation}
Together with \cref{eq:tau_smooth}, this is the solution obtained by using the
regular perturbation method to the second-order nonlinear oscillator
\cref{eq:second_order_nonlinear_oscillator_smooth_normalform} obtained from the
smooth normal form with phase condition $\dot u(0)=0$. This gives us the third-order
homoclinic predictor
\begin{equation}
\label{eq:third_order_predictor_smooth_RPM_tau}
\ifthesis
\begin{cases}
\begin{aligned}
w_0(t)  &= \frac1{a} \left( \sum_{i=0}^3 u_i(\epsilon t) \epsilon^i\right) \epsilon^2, \\
w_1(t)  &= \frac1{a} \left( \sum_{i=0}^3 \dot u_i(\epsilon t) \epsilon^i\right) \epsilon^3, \\
\beta_1    &= -\frac{4}{a} \epsilon^4, \\
\beta_2    &= \frac{b}{a}\epsilon^2 \tau,
\end{aligned}
\end{cases}
\else
\left( w_0(t), w_1(t), \beta_1, \beta_2 \right) = 
\left( \frac1{a} \left( \sum_{i=0}^3 u_i(\epsilon t) \epsilon^i \right) \epsilon^2,
        \frac1{a} \left( \sum_{i=0}^3 \dot u_i(\epsilon t) \epsilon^i \right)   \epsilon^3,
-\frac{4}{a} \epsilon^4,
\frac{b}{a}\epsilon^2 \tau \right)
\fi
\end{equation}
where $\tau$ is given by \cref{eq:tau_smooth} and $u_i(i=0,\dots,3)$ are given
by \cref{eq:u_i_RP_smooth}.

\fi

\ifthesis
\subsubsection{Comparison with the nonlinear periodic time-reparametrization}
In~\cite{Algaba_2019} a different approach is used to approximate the homoclinic
solution near the quadratic normal form of a generic codimension 2
Bogdanov--Takens bifurcation. The approach there is an application of the
so-called Perturbation-Incremental Method described in~\cite{Xu_1996}. Consider
strongly nonlinear oscillators of the form
\begin{equation}
  \label{eq:nonlinear_oscillator}
  \ddot x + g(x) = \lambda f(x,\dot x, \mu) \dot x, \qquad \
\end{equation}
where $g$ and $f$ are arbitrary nonlinear functions, and
$\lambda$ and $\mu$ are parameters.

The authors in~\cite{Xu_1996} perform a  nonlinear periodic time-reparametrization
of the form
\begin{equation}
\label{eq:nonlinear_periodic_reparametrization}  
\frac{d\phi}{dt} =\Phi(\phi), \qquad \Phi(\phi+2\pi)=\Phi(\phi)
\end{equation}
to the system \cref{eq:nonlinear_oscillator}. Then it is assumed that there is a
homoclinic orbit present which can be approximated by the solution
\[
u(\phi) = p \cos(2\phi) + q,
\]
where $p$ and $q$ are constants to be approximated. In~\cite[Theorem
1]{Algaba_2019} it is shown that the solutions $u(\phi)$ and $u(s)$ are related
to each other through
\[
    \Phi(\phi) = \frac{\sqrt 2}{2} \omega(\xi) \sin\phi
.\] 

It follows that one should be able to factor out the term
$\sin{\phi}$ in \cref{eq:nonlinear_periodic_reparametrization}. This is
indeed precisely what we see in the transformation $\Phi$
in~\cite[(41)]{Algaba_2019}. Thus, although the nonlinear periodic time- reparametrization
is analytically equivalent to the polynomial generalized
Lindstedt-Poincar\'e method, it is geometrically \emph{less intuitive} than
using hyperbolic functions and computationally \emph{more expensive} than using
polynomials.

We also would like to point out that the singular rescaling
\begin{equation*}
    w_0 = u \epsilon^2, \quad
    w_1 = v \epsilon^3, \quad
    \beta_1 = -\epsilon^4, \quad 
    \beta_2 = \tau \epsilon^2, \quad 
    s = \epsilon \eta, \quad (\epsilon \neq 0),
\end{equation*}
used in~\cite{Algaba_2019} applied to the quadratic Bogdanov--Takens normal form
with coefficient $a=1$ and $b=1$ results in the $\sqrt 2$ turning up in the
calculations of homoclinic approximation. From a computational point of view,
this is less ideal to work with.

\begin{remark}
    Note that the polynomial generalized Lindstedt-Poincar\'e method described
    above only depends on the existence of the zeroth-order solution.
    Therefore, this method can be applied in similar situations where
    homoclinic orbits emanate from codimension two Bogdanov--Takens bifurcation
    points. For example, the transcritical codimension two bifurcation treated
    in~\cite{Hirschberg_1991}, see also~\cite[Appendix C.2]{Bosschaert@2016}.
    Furthermore, under certain symmetry present in the ODE heteroclinic
    solutions can also emanate from codimension two Bogdanov--Takens points
    which can be approximated using the polynomial Lindstedt-Poincar\'e method
    as described above.
\end{remark}
\fi
