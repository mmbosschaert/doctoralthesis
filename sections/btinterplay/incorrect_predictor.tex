\section{Explicit example demonstrating incorrect predictor}
\label{app:incorrect_predictor}

Although the second-order homoclinic approximation derived
in~\cite{Al-Hdaibat2016} for the smooth normal form \cref{eq:BT_smooth_nf} is
correct, the parameter and center manifold transformation are incorrect. To see
this, we suppose \cref{eq:ODE} is given by
\begin{align}
\label{eq:btnormalform_with_alpha2^3}
\dot x = f(x,\alpha) = \begin{pmatrix}   x_1 \\
 \alpha_1 + \alpha_2 x_1 + x_0^2 + x_0 x_1 + c_1 \alpha_2^3
 \end{pmatrix},
\end{align}
for some arbitrary nonzero constant $c_1 \in \mathbb R$.  We will now compare
two different methods for obtaining a second-order approximation to the
homoclinic solution in \cref{eq:btnormalform_with_alpha2^3}. To keep the
exposition as clear as possible, we focus solely on the parameters. For the
first method we directly apply the singular rescaling 
\[
\alpha_1=-4\epsilon^4, \quad \alpha_2 = \eta \epsilon^2,
\quad x_0= \epsilon^2, \quad x_1= \epsilon^3, \quad
s=\epsilon t,
\]
to \cref{eq:btnormalform_with_alpha2^3}. This yields the system
\[
\begin{cases}
\begin{aligned}
  \dot u ={}& v, \\
  \dot v ={}& -4 + u^2 + v\left( u + \tau \right)\epsilon +  c_1 \tau^3
	\epsilon^2, \\
\end{aligned}
\end{cases}
\]
where the dot $\dot{}$ now represents the derivative with respect to $s$.
Then, using the generalized Lindstedt-Poincar\'e method we obtain the approximation
\begin{equation}
\label{eq:first_predictor}
				(\alpha_1, \alpha_2) = \left(-4 \epsilon^4, \frac{10}7 \epsilon^2 
+\frac{288-1250 c_1}{2401}\epsilon^4 + \mathcal O(\epsilon^5) \right)
\end{equation}
for the parameters. For the second method we use the predictor
from~\cite{Al-Hdaibat2016}. That is, we use the second-order homoclinic
predictor derived for the smooth normal form \cref{eq:BT_smooth_nf}. Then
calculate the center manifold transformation, which for the two-dimensional
systems reduces to a near-identity transformation, and parameter transformation
to transfer the predictor to the original system. We obtain that the near-identity
and parameter transformation are just the identities. Thus, we obtain the
predictor
\begin{equation}
				\label{eq:second_predictor}	
				(\alpha_1, \alpha_2) = \left(-4 \epsilon^4, \frac{10}7 \epsilon^2 
				+\frac{288}{2401}\epsilon^4 + \mathcal O(\epsilon^5) \right).
\end{equation}
Obviously, this result is wrong.  To see why the latter second predictor
doesn't contain the term $c_1$ we consider the near-identity transformation
\begin{align}
\label{eq:near_identity_tranformation}
  \begin{cases}
  \begin{aligned}
    x &{}= w, \\ 
    \alpha &{}= \beta+ \begin{pmatrix} -c_1 \\ 0 \end{pmatrix}\beta_2^3.
  \end{aligned}
  \end{cases}
\end{align}
Then system \cref{eq:btnormalform_with_alpha2^3} becomes
\begin{align*}
  \begin{cases}
  \begin{aligned}
     \dot w_0 &{}= w_1, \\
		 \dot w_1 &{}= \beta_1 + \beta_2 w_1 + w_0^2 + w_0 w_1.
  \end{aligned}
  \end{cases}
\end{align*}
Using the second-order predictor from~\cite{Al-Hdaibat2016} for the smooth
normal form we obtain
\[
    (\beta_1, \beta_2) = \left(-4 \epsilon^4, \frac{10}7 \epsilon^2 
    +\frac{288}{2401}\epsilon^4 + \mathcal O(\epsilon^5) \right).
\]
Then using the near-identity transformation
\cref{eq:near_identity_tranformation} yields the predictor
\begin{equation*}
				(\alpha_1, \alpha_2) = \left(-4 \epsilon^4 
								- c_1 \left(\frac{10}7 \epsilon^2 
								+\frac{288}{2401}\epsilon^4 + \mathcal O(\epsilon^5) \right)^3, 
								\frac{10}7 \epsilon^2 
				+\frac{288}{2401}\epsilon^4 + \mathcal O(\epsilon^5) \right).
\end{equation*}
To compare this predictor with \cref{eq:first_predictor} we eliminate $\epsilon$
in both equations. This yields
\begin{equation*}
				\alpha_2(\alpha_1) = - \frac57 \sqrt{-\alpha_1} +
				\frac{288-1250c_1}{2401} \alpha_1 + \mathcal O(\alpha_1^{\frac32}).
\end{equation*}

We conclude, as expected, that if the correct transformation is used between the
normal form and the original system, we keep the correct order of accuracy for
the approximation. In~\cite{Al-Hdaibat2016} the coefficients $H_{0003}$ and
$K_{03}$ (among other coefficients) are not incorporated into the
parameter-dependent center manifold transformation
\cref{eq:H_expansion,eq:K_expansion} leading to the incorrect
predictor \cref{eq:second_predictor}.
