\ifthesis
\subsection{Implementation}%
\else
\section{Implementation}%
\fi
\label{sec:implementation}
In this section, we first briefly review the method used in \MATCONT to continue
homoclinic solutions in autonomous ordinary differential equations 
\cref{eq:ODE} in two parameters presented in ~\cite{DeWitte2012}. Then we
describe our implementation in \MATCONT of the algorithm to start the continuation of
the homoclinic solutions emanating from a codimension two Bogdanov--Takens point
using the derived above homoclinic predictors and parameter-dependent center manifold.

\ifthesis
\subsubsection{Continuation of homoclinic solutions in \MATCONT}
\else
\subsection{Continuation of homoclinic solutions in \MATCONT}
\fi
\label{sec:cont_hom_matcont}
\MATCONT uses a correction-prediction continuation method applied to a discretized defining
system~\cite{matcont2}. The defining system for the continuation of homoclinic
solutions in ordinary differential equations of the form \cref{eq:ODE} in two
parameters in \MATCONT are given by
\begin{equation}
\begin{aligned}
    \label{eq:definingSystem}
    \dot{x}(t)-2 T f(x(t), \alpha)=0,  \\
    f\left(s_{0}, \alpha\right)=0, \\
    \int_{0}^{1} \widetilde{x}(t)[x(t)-\widetilde{x}(t)] d t=0, \\
    Q^{U^{\perp}, \mathrm{T}}\left(x(0)-s_{0}\right)=0, \\
    Q^{S^{\perp}, \mathrm{T}}\left(x(1)-s_{0}\right)=0, \\
    T_{22 U} Y_{U}-Y_{U} T_{11 U}+T_{21 U}-Y_{U} T_{12 U} Y_{U}=0, \\
    T_{22 S} Y_{S}-Y_{S} T_{11 S}+T_{21 S}-Y_{S} T_{12 S} Y_{S}=0, \\
    \left\|x(0)-s_{0}\right\|-\epsilon_{0}=0, \\
    \left\|x(1)-s_{0}\right\|-\epsilon_{1}=0,
\end{aligned}
\end{equation}
see~\cite{DeWitte2012}.  Here, the infinite time interval
$\interval[open]{-\infty}{ \infty}$ of the homoclinic orbit is truncated to a
finite interval $[-T,T]$, where $T>0$ is called the half-return time. The
truncated interval is rescaled to the interval $[0,1]$ and divided into
\mintinline{matlab}{ntst} mesh-intervals. Each mesh interval is further
subdivided by equidistant fine mesh points, where the solution is approximated
by a vector polynomial. Each mesh interval contains a number of
\mintinline{matlab}{ncol} collocation points where the first equation in
\cref{eq:definingSystem} must be satisfied. The second equation in
\cref{eq:definingSystem} locates the saddle point $s_0$ of the homoclinic
orbit. The last two equations in \cref{eq:definingSystem} define the distance
$\epsilon_0$ and  $\epsilon_1$ between the saddle and the homoclinic solution
at $t=0$ and $t=1$, respectively. The half-return time $T$, $\epsilon_0$,
$\epsilon_1$ are referred to as the \emph{homoclinic parameters}. Either one or
two of the homoclinic parameters must be allowed to vary. If two homoclinic
parameters are selected to vary, the third equation (the phase condition) in
\cref{eq:definingSystem} is added. Here $\tilde x$ is a reference solution,
usually obtained from the previous continuation step. The fourth and fifth
equations in \cref{eq:definingSystem} place the solution at the endpoints in
the unstable and stable eigenspace of linearization of the saddle point,
respectively. The matrices $Q^{U^{\perp}} \in \mathbb{R}^{n \times n_{S}}$ and
$Q^{S^{\perp}} \in \mathbb{R}^{n \times n_{U}}$ are not recalculated each
continuation step, but constructed from the lower dimensional matrices $Y_{U}
\in \mathbb{R}^{n_{S} \times n_{U}}$ and $Y_{S} \in \mathbb{R}^{n_{U} \times
n_{S}}$. The fifth and sixth equations in \cref{eq:definingSystem}, referred to
as algebraic Ricatti equations, keep track of the lower dimensional matrix
$Y_U$ and $Y_S$, see~\cite{Friedman2001Continuation}. The matrices $Y_U$ and
$Y_S$ are initially set to zero.

Thus, in order to start continuation of homoclinic solutions near a codimension
two Bogdanov--Takens point in \cref{eq:ODE}, one needs to provide an initial
approximation to
\begin{itemize}
    \item the discretized orbit on the rescaled and truncated interval $[0,1]$,
    \item the parameter values $\alpha$,
    \item the saddle point $s_0$,
    \item the half-return time $T$,
    \item the initial distances $\epsilon_0$ and $\epsilon_1$,
    \item and an initial tangent vector for the next prediction.
\end{itemize}

Since the homoclinic predictors depend on the coefficients of the normal form,
we first calculate the parameter-dependent center manifold transformation.

\ifthesis
\subsubsection{Multilinear forms}
\else
\subsection{Multilinear forms}
\fi
Unfortunately, not all multilinear forms in the expansion
\cref{eq:f_expansion_bt} needed for the derivation of the coefficients for the
transformations $H$, $K$, and $\theta$, were previously implemented in \MATCONT.  We,
therefore, developed new scripts which generate the necessary multilinear
forms symbolically if the symbolic toolbox for MATLAB or for GNU Octave is
installed.  The multilinear forms can be generated with the graphical user
interface of \MATCONT, or via the command-line interface.  Examples are given in
the Supplementary Materials.  If the symbolic toolbox is not
available, finite differences in combination with polarization identities are
used instead.  Note that symbolical derivatives generated with either GNU
Octave or \MATCONT can be used interchangeably.

\ifthesis
\subsubsection{Coefficients of the parameter-dependent center manifold}
\else
\subsection{Coefficients of the parameter-dependent center manifold}
\fi
Next, we compute the coefficients in the expansion $G, H, K$, and $\theta$ as
derived in \cref{sec:Center_manifold_reduction_ODE}. Since we allow
using finite differences, the results may become inaccurate. We, therefore,
provide a warning message if one or more of the systems are not satisfied with
a prescribed accuracy. The code for calculating the coefficients of the
orbital, smooth, and hyper normal form can be found in the scripts
\mintinline{julia}{BT_nmfm_orbital.m}, \mintinline{julia}{BT_nmfm.m}, and
\mintinline{julia}{BT_nmfm_without_e_b1.m}, respectively. These scripts can be
called independently once a Bogdanov--Takens point is located.

\ifthesis
\subsubsection{The perturbation parameter and the half-return time}
\else
\subsection{The perturbation parameter and the half-return time}
\fi
To provide the data listed at the end of \cref{sec:cont_hom_matcont}, one first
needs to select a suitable perturbation parameter $\epsilon$ in the homoclinic
predictors in \cref{sec:homoclinic_asymptotics_n_dimension}.
In~\cite{Al-Hdaibat2016} a geometrically motived approach is used to determine
the perturbation parameter $\epsilon$ and half-return time $T$. The user first
provides the amplitude $A_0$ of the homoclinic orbit, which can be approximated
by
\begin{equation*}
    A_0 = \|\bar x(0,\epsilon)-s_0\|.
\end{equation*}
From the amplitude $A_0$ the initial perturbation parameter $\epsilon$ can then
be estimated by truncating the homoclinic predictors in
\cref{sec:homoclinic_asymptotics_n_dimension} up to second-order in $\epsilon$.
Then, by using that the norm of the eigenvector $q_0$ is of unit length, see
\cref{eq:q0}, we obtain
 \[
     \epsilon^o  = |b|\sqrt{ \frac{A_0}{6|a|}},
     \quad \mbox{and} \quad 
     \epsilon^s  = \sqrt{ \frac{A_0|a|}{6}},
\] 
for the orbital and smooth homoclinic predictor, respectively.

Secondly, the user provides the distance $k$, also referred to as
\mintinline{matlab}{TTolerance} in \MATCONT, between the endpoints of the
truncated homoclinic orbit and the saddle point
\begin{equation}
    k = \|\bar x(\pm T,\epsilon)-s_0\|. \label{eq:TTolerance}
\end{equation}
From this equation the half-return time $T$ is solved by again truncating the
homoclinic predictors given in \cref{sec:homoclinic_asymptotics_n_dimension} up
to second-order in $\epsilon$. We obtain
\[
     \eta(T) = \left|\frac{b}{a} \frac{1}{\epsilon}
     \arcsech\left(\frac{|b|}{\epsilon} \sqrt{\frac{k}{6|a|}}\right)\right|,
     \quad \mbox{and} \quad 
     T  = \frac{1}{\epsilon} \arcsech\left(\sqrt{\frac{k}{A_0}}\right),
\] 
for the orbital and smooth homoclinic predictor, respectively. Thus, for the
orbital predictor, the half-return time $T$ has to be computed by numerical
inverting $\eta(T)$, either using \cref{eq:tOfTauRegular} or \cref{eq:tOfTau},
depending on the perturbation method used.

Note that if one truncates the homoclinic orbit up to the second-order, only
the zeroth-order solution of $u$ is used. Furthermore, since only the
eigenvector $q_0$ is used in the approximation of the amplitude $A_0$ and the
half-return time $T$, one should not expect to obtain an accurate approximation
of the amplitude in general $n$-dimensional systems. For the half-return time
$T$, there is an additional problem in the approximation. Namely, the homoclinic
predictors are not even functions in $t$, i.e., we have the inequality
\begin{equation*}
    \|\bar x(T,\epsilon)-s_0\| \neq \|\bar x(-T,\epsilon)-s_0\|.
\end{equation*}
Thus, the more accurate interpretation of the amplitude $A_0$ and half-return
time $T$ is that they represent approximations for the amplitude and
half-return time of $u_0$ on the center subspace and not for the homoclinic
orbit on the center manifold. This, however, does not influence the convergence
of the initial prediction for the homoclinic orbit, as long as the derived
perturbation parameter $\epsilon$ is within the radius of convergence.

Instead of requiring the user to provide the amplitude $A_0$ and distance $k$,
we determine the perturbation parameter $\epsilon$ automatically.  Motivated by
\cref{sec:case_study_BT2} an initial, good guess for the orbital homoclinic
predictor would be obtained with $\epsilon=0.1$. The higher-order terms, not
taken into account for the predictor, would have to behave very badly to not
lead to convergence. The distance $k$ we set to $k = \epsilon 10^{-4}$.  In
case that Newton does not converge, the perturbation $\epsilon$ is halved and
$k$ is updated.  This process is then repeated until convergence is obtained,
or by the maximum prescribed number of tries. In fact, for all but one model in
\cref{sec:examples} and the Supplementary Materials, setting the perturbation
parameter $\epsilon=0.1$ and the distance $k=10^{-4}$, while using the orbital
predictor with either the Lindstedt-Poincar\'e or the regular perturbation
method, leads to convergence to the true homoclinic solution.

\ifthesis
\subsubsection{Homoclinic solution}
\else
\subsection{Homoclinic solution}
\fi
After the perturbation parameter $\epsilon$ and the half-return time $T$ have
been determined and using the coefficients of $H,K$, and possibly $\theta$, the
homoclinic approximation is obtained by evaluating one of the homoclinic
approximations given in \cref{sec:homoclinic_asymptotics_n_dimension} on the
fine mesh points $f_i$ for $i=0,\dots,$ \mintinline{matlab}{ntst}$\times$\mintinline{matlab}{ncol} on the
rescaled discretized interval $[0,1]$.  Note that we need to translate the
approximation to the Bogdanov--Takens phase point $x_0$ under consideration.

\ifthesis
\subsubsection{Saddle point}
\else
\subsection{Saddle point}
\fi
To obtain an approximation for the saddle point of the homoclinic orbit, we
simply let $t$ go to infinity in the homoclinic approximation obtained in the
previous step. Note that $\tau$ goes to infinity as $t$ does. Thus, using
\cref{eq:x_eta_espilon} we obtain
\begin{equation*}
\begin{aligned}
    s_0 ={}& x_0 + q_0 w_0^{\infty} + \hat H_{0010} \beta_1 + \hat H_{0001} \beta_2 +
            \frac{1}{2} \hat H_{2000} \left(w_0^{\infty}\right)^2  \\
           & + \hat H_{1010} w_0^{\infty} \beta_1 + \hat H_{1001} w_0^{\infty} \beta_2 +
             \frac{1}{2} \hat H_{0002} \beta_2^2 + \hat H_{0011} \beta_1 \beta_2 \\
           & + \frac{1}{6} \hat H_{3000} \left(w_0^{\infty}\right)^3 +
             \frac{1}{2} \hat H_{2001} \left(w_0^{\infty}\right)^2 \beta_2 
             + \frac{1}{6} \hat H_{0003} \beta_2^3 + \frac{1}{2} \hat H_{1002} w_0^{\infty} \beta_2^2
\end{aligned}
\end{equation*}
where
\[
  w_0^\infty = 2\frac{a}{b^2} \epsilon^2,
\] 
for the orbital predictor the approximation and
\[
    w_0^\infty = \frac1a \left( 2 - 2 \frac{5 a_1 b + 7 d}{7a^2} \epsilon^2
    \right) \epsilon^2,
\] 
for the smooth predictor.

\ifthesis
\subsubsection{Tangent}
\else
\subsection{Tangent}
\fi
Once a good initial prediction for the homoclinic solution to
\cref{eq:definingSystem} has been obtained, a normalized tangent vector is
needed to start the continuation process. The simplest method to obtain the
normalized tangent vector is by calculating the one-dimensional null space of
the sparse rectangular Jacobian of the defining system \cref{eq:definingSystem}.
Although this is easy to implement numerically via a QR-decomposition of the
transpose, the drawback is that we do not have any control of the orientation
of the tangent vector. Thus, we do not know in which direction the continuation
starts.  Obviously, we want to avoid continuing the homoclinic curve towards
the Bogdanov--Takens point. One way to obtain the correct direction is by
obtaining two approximations with close, but different perturbation parameters,
from which the normalized tangent can be approximated.  Computationally cheaper
and more accurate is to first compute the null space $V$ of the Jacobian as
described above and subsequently inspect the sign of the $\alpha_1$ component
in the vector $V$. This sign should be equal to the sign of the derivative of
the first component of $\bar\alpha$ with respect to $\epsilon$, see
\cref{eq:alpha_espilon}. A simple calculation yields
\[
\bar\alpha'(\epsilon)  = K_{10} \beta_1'(\epsilon) + K_{01} \beta_2'(\epsilon) +
     K_{02} \beta_2 \beta_2'(\epsilon) + K_{11} (\beta_1'(\epsilon) \beta_2 +
     \beta_1 \beta_2'(\epsilon)) + \frac{1}{2} K_{03}\beta_2^2 \beta_2'(\epsilon),
\] 
where $\beta_1'(\epsilon)$ and $\beta_2'(\epsilon)$ are given by
\[
\left\{
\begin{aligned}
    \beta_1'(\epsilon) ={}& -16 \frac{a^3}{b^4} \epsilon^3, \\
    \beta_2'(\epsilon) ={}& \frac{a}{b} (2 \tau_0 + 4 \tau_2 \epsilon^2) \epsilon, \\
\end{aligned}
\right.
\] 
with $\tau_{0,1}$ are given in \cref{eq:tau_orbital} for the orbital
predictor, and by
\[
\left\{
\begin{aligned}
    \beta_1'(\epsilon) ={}&  -\frac{16}{a} \epsilon^3, \\
    \beta_2'(\epsilon) ={}& \frac{b}{a} (2 \tau_0 + 4 \tau_2 \epsilon^2) \epsilon, \\
\end{aligned}
\right.
\] 
where $\tau_{0,1}$ are given in \cref{eq:tau_smooth} for the smooth homoclinic
predictor. Thus, we simply change the sign of the vector $V$ if the product of
the $\alpha_1$ component of $V$ with the first component of $\bar\alpha'(\epsilon)$
is negative, otherwise, we leave the vector $V$ unchanged.

