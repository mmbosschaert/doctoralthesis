\section{Introduction} 
Let $f\colon\mathbb R^n \times \mathbb R^2 \to \mathbb R^n$ with $n\geq 2$, be
smooth and suppose that the autonomous ordinary differential equation (ODE)
\begin{equation}
    \label{eq:ODE}
    \dot x(t) = f(x(t),\alpha)
\end{equation}
has equilibrium  $x_0= 0$ that undergoes a codimension two local bifurcation at
the critical parameter value $\alpha_{0}=0$. Here, the dot means the derivative
with respect to the independent variable $t\in\mathbb R$. To understand the
dynamics near the bifurcation point $(x_0,\alpha_0)$ for nearby parameter
values, one typically first restricts the ODE to the center manifold. By
projecting the solutions on the center manifold onto the center subspace, one
then obtains a $n_c$-dimensional ODE that locally governs the restricted
dynamics.  Using the normal form theory, one further tries to transform the
restricted ODE into a simpler form, called the \emph{critical normal form}.

If the canonical unfolding of the critical normal form is known and only
qua\-li\-tative behavior near the equilibrium is of interest, one can stop
here. However, if one is interested in relating solutions of the unfolding to
those of the original system \cref{eq:ODE} near the bifurcation point, one
needs a relation between the \emph{parameter-dependent normal form} and the
restricted ODE, and also a relation between this restricted ODE on the
parameter-dependent center manifold and the original system \cref{eq:ODE}.
These two relations can be found simultaneously utilizing the \emph{homological
equation} approach, see~\cite{Beyn2002149}. 

The solutions of interest here are the codimension one bifurcation curves
emanating from the codimension two point and the corresponding orbits in phase
space. In general, the bifurcation curves in the 
parameter-dependent normal form are not known exactly, but only by an
approximation up to a certain order.  Si\-mi\-lar\-ly, the transformation
from the normal form to the (parameter-dependent) center manifold is
generally also only known up to a certain order. Then, by combining these two
transformations, an approximation to the codimension one bifurcation curve and
the corresponding phase orbits is obtained for the original system
\cref{eq:ODE}.

These approximations are particularly useful in numerical continuation software
to start the continuation of the codimension one bifurcation curves emanating
from the codimension two bifurcation points, where the defining systems for the
orbits of interest become degenerate. A codimension two bifurcation that has
attracted much attention is the \emph{Bogdanov--Takens bifurcation} at which the
cri\-ti\-cal equilibrium has a double zero eigenvalue. It is well-known that
under certain non-degeneracy and transversality conditions, three codimension
one bifurcation curves emanate from the Bogdanov--Takens point: a saddle-node,
an (Andronov-)Hopf, and a saddle-homoclinic bifurcation curve. Since the
standard defining systems for the equilibrium bifurcations are non-degenerate
at the Bogdanov--Takens point, one does not need an approximation to start
continuation there.  On the contrary, the standard defining system for the
homoclinic solution does become degenerate, which is easily seen since the
homoclinic orbit shrinks to the equilibrium point when we approach the Bogdanov--Takens
bifurcation.

Starting continuation of the homoclinic orbits from a Bogdanov--Takens point in
ODEs attracted much attention. In planar systems, Melnikov's method was first
applied to solve this problem in~\cite{Rodriguez-Luis1990}. A first attempt to
provide asymptotic approximations to the homoclinic bifurcation curve near a
generic codimension two Bogdanov--Takens bifurcation point in general
$n$-dimensional systems was made in~\cite{Beyn_1994}. By applying a singular
rescaling to the (one of the equivalent) parameter-dependent normal form on the
center manifold, a perturbed planar Hamiltonian system is obtained. The
unperturbed Hamiltonian system contains an explicit homoclinic solution. A
first-order correction in parameter-space can subsequently be obtained by
reformulating the problem as a branching problem in a suitable Banach space,
see~\cite{Beyn_1994}.  Then, by using the regular perturbation method,
higher-order approximations to the homoclinic bifurcation curve can be
obtained. Unfortunately, in~\cite{Beyn_1994}, even the first-order correction in
the phase-space was not derived.  Nonetheless, it was proven that the obtained
homoclinic predictor converges to the true solution under the Newton iterations
in the perturbed Hamiltonian systems.

In~\cite{Kuznetsov2014improved} the work continued by obtaining a second-order
correction in parameter \emph{and} phase-space to the homoclinic bifurcation
curve for the perturbed Hamiltonian system. However, a new problem was
overlooked.  The normal form used in~\cite{Kuznetsov2014improved} is a normal
form for $C^\infty$-equivalence (also called \emph{smooth orbital
equivalence}), i.e., besides a $C^\infty$-coordinate change, also a time-reparametrization
must be taken into account, which was not the case
in~\cite{Kuznetsov2014improved}. In the subsequent paper~\cite{Gray-Scott2015},
this problem was resolved by considering a smooth normal form for the
Bogdanov--Takens bifurcation point, which is a normal form for
$C^\infty$-conjugacy (\emph{smooth equivalence}). 

In the follow-up paper~\cite{Al-Hdaibat2016}, progress was made in obtaining a
uniform approximation in time of the homoclinic solution, using a
generalization of the Lindstedt-Poincar\'e method. This removes the so-called
parasitic turns near the saddle point, as observed
in~\cite{Kuznetsov2014improved}. Although, as pointed out
by~\cite{Algaba_2019}, there were mistakes in the third-order approximation
with the Lindstedt-Poincar\'e method, the asymptotics for the homoclinic predictor from
\cite{Kuznetsov2014improved} for the smooth normal form improved significantly
in the phase space.

Nonetheless, the task of correctly lifting the asymptotics in the normal
form to the parameter-dependent center manifold was not accomplished. 
Effectively, only the zeroth-order approximation to the homoclinic
solutions in the phase space, i.e., a transformed homoclinic solution of the
\emph{un}perturbed Hamiltonian system, was available for a general $n$-dimensional
system.

In this \paper{}, we will provide for the first time the third-order homoclinic
predictor for the homoclinic solutions emanating from a generic Bogdanov--Takens
point for a general $n$-dimensional system. For this, we first consider the general
setting of lifting asymptotics of a codimension one bifurcation curves in parameter and phase-space
derived in the normal form of a codimension two bifurcation point to the 
parameter-dependent center manifold of the bifurcation point under consideration.
In \cref{sec:Center_manifold_reduction_ODE} we will show how to systematically
determine which coefficients to include in the parameter-dependent center manifold
transformation and parameter transformation in order to maintain the approximation
order of the available asymptotics, see in particular \cref{thm:coefficients}.

Then we conclude that for the specific case of lifting the asymptotics of the
homoclinic bifurcation curve obtained in the normal form of the Bogdanov--Takens
bifurcation, we need to consider several additional systems to be solved in the
homological equation method that were previously not taken into account.
During the derivation of the coefficients of the normal form and the
transformations, we will show that there is no need to solve certain systems
simultaneously (see the so-called `big' system in
~\cite{Kuznetsov2014improved,Gray-Scott2015,Al-Hdaibat2016}), making the expressions also
suitable for infinitely-dimensional ODEs generated by partial and
delay differential equations to which the (parameter-dependent) center
manifold theorem applies. 

Furthermore, by
allowing a transformation of time between the normal form and the original
system, we can use the parameter-dependent \emph{smooth orbital} normal form of
the codimension two Bogdanov--Takens bifurcation point when approximating the
homoclinic solution up to order three. This normal form is considerably simpler
than previously employed smooth normal forms. The derivation of the
coefficients will be the subject of the remainder of
\cref{sec:Center_manifold_reduction_ODE}.

Having derived the parameter-dependent center manifold transformation suitable for
lifting the third-order homoclinic asymptotics present in different generic
Bogdanov--Takens normal forms, we turn our attention to obtain the asymptotics
in \cref{sec:asymptotics}. We will revisit both the regular perturbation method and
the generalized Lindstedt-Poincar\'e method considered previously.

The non-uniqueness of the homoclinic solution due to a time shift results in
non-uniqueness for the systems to be solved in the regular perturbation method.
To obtain uniqueness, a so-called \emph{phase condition} needs to be satisfied.
The phase condition used in~\cite{Kuznetsov2014improved} originates from a
theoretical setting in~\cite{Beyn_1994}. In \cref{sec:RPM_norm_minimizing_phase
condition} we use another geometrically motivated phase condition which slightly
improves the regular perturbation solution. Furthermore, by modifying
Proposition 4.3 from~\cite{Beyn_1994}, we use symmetry arguments to simplify
the calculations.

In \cref{sec:PolynomailLindstedtPoincare} the generalized Lindstedt-Poincar\'e
method for the approximation of homoclinic orbits is improved by introducing an
additional transformation of time after applying the usual nonlinear time
transformation. The resulting algorithm solitary relies on polynomial division
and does not involve any hyperbolic or trigonometric functions as
in~\cite{Algaba_2019,Al-Hdaibat2016}. We show that for the quadratic Bogdanov--Takens
normal form, we can represent the homoclinic solution in phase-space with
only one single parameter.

In \cref{sec:third_order_homoclinic_approximation_LP} we provide an explicit
third-order homoclinic approximation in the perturbed Hamiltonian system using
the algorithm described in \cref{sec:PolynomailLindstedtPoincareMethod}. Here, we also
provide a third-order approximation to the reparametrization of time. The
profiles of the homoclinic solution will only then be approximated accurately,
resulting in a robust initial predictor for starting continuation of the branch
of homoclinic orbits.  In~\cite{Al-Hdaibat2016} the importance of the
time-reparametrization was not recognized, and the zeroth-order approximation
was used. We will demonstrate in detail that it is essential to use the higher
time-reparametrization by comparing the Lindstedt-Poincar\'e method with and
without the higher-order time-reparametrization.  Effectively, using the
Lindstedt-Poincar\'e method without the higher-order time-reparametrization is
equivalent to the zeroth-order regular perturbation method as illustrated in
\cref{fig:RP_vs_LP2016_vs_LP_profiles}.

The algorithm given in \cref{sec:PolynomailLindstedtPoincareMethod} is implemented in
\cref{sec:case_study_BT2} in the programming language
Julia~\cite{bezanson2017julia} for the quadratic normal form for the
Bogdanov--Takens codimension two bifurcation. Here we gain some insight about the
finite convergence radius of the homoclinic asymptotics and the speed of the
algorithm.

By combining the homoclinic asymptotics derived in \cref{sec:asymptotics} with
the parameter-de\-pendent center manifold transformation obtained in
\cref{sec:Center_manifold_reduction_ODE}, we get a correct homoclinic
predictor for a general $n$-dimensional system.  It will be shown in
\cref{sec:homoclinic_asymptotics_n_dimension} how to incorporate the time
translation into the homoclinic predictor.
 
Then we compare the homoclinic predictor for the smooth
normal form and the smooth orbital normal form. In
\cref{sec:comparison_homoclinic_predictors}, it will be shown that these two
predictors are asymptotically equivalent, up to a phase shift. Then, by
choosing the constants of integration in the time translation in a specific
manner, we show equivalence between the predictors.

All the above methods are implemented in the open-source bifurcation and
continuation software \MATCONT ~\cite{matcont2}.  In
\cref{sec:implementation} we describe the new implementation of the homoclinic
predictor in the latest version of \MATCONT. We show how to use the obtained
predictors to construct an initial prediction for the defining system of the
homoclinic solutions. Besides an initial prediction also an initial tangent
vector is necessary to start continuation.  Our implementation prevents the
issue of possible continuation in the wrong direction, i.e., towards the
Bogdanov--Takens point. 

The effectiveness of the new predictors is demonstrated on the topological
normal form and on two four-dimensional models from Neuroscience and Quantum
Field Theory in \cref{sec:examples}. A comparison between the new homoclinic
predictor near a generic codimension 2 Bogdanov--Takens bifurcation and the
predictor from~\cite{Al-Hdaibat2016} is given. It will be shown that the order
of the higher-order approximations to the homoclinic solutions in the normal
form is preserved under the parameter-dependent center manifold transformation.
Complementary to \cref{sec:examples} an 
\href{https://mmbosschaert.github.io/MatCont7p2NewInitBTHom-/}{online Jupyter Notebook}
is provided in which 10 different models are considered using the new
homoclinic predictor and comparing different approximation methods in detail.
