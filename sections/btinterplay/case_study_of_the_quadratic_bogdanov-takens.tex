\begin{code}
\begin{minted}[breaklines,escapeinside=||,mathescape=true,
numbersep=3pt, gobble=2, frame=lines, fontsize=\small, framesep=2mm]{julia}
module BTQuadraticHomoclinic

using Polynomials, OffsetArrays

function z(i, τ, σ, ω)
    if i == 1
        p = 24*Polynomial([0,-2, 0,  3])
    else
        p =  Polynomial([0,2])*sum(σ[l]*τ[i-1-l] for l in 1:i-1)
        p += Polynomial([0,2])*sum(σ[l]*ω[k]*τ[i-1-l-k] 
                for k in 1:i-1 for l in 0:i-1-k)
        p -= Polynomial([2,0,-6])*sum(σ[i-l]*ω[l] for l in 1:i-1)
        p -= 2*sum(σ[i-l-k]*ω[l]*derivative(
                Polynomial([0,1,0,-1])*ω[k]) for k in 1:i-1 for l in 0:i-k)
        p += Polynomial([-1,0,1])*Polynomial([0,2])*sum(σ[l]*
               σ[i-1-l-k]*ω[k] for k in 0:i-1 for l in 0:i-2-k)
        p += Polynomial([-4,0,6])*Polynomial([0,2])*sum(σ[i-1-k]*ω[k]
                for k in 0:i-1)
        p += Polynomial([1,0,-1])*sum(σ[k]*σ[i-k] for k in 1:i-1)
    end
    Polynomial([1,0,-1])*p
end

function solve(;order = n)
    σ = OffsetArray(zeros(Rational{BigInt}, order), 0:order-2)
    τ = OffsetArray(zeros(Rational{BigInt}, order-1), 0:order-1)
    ω = OffsetArray(Array{Polynomial}(undef, order), 0:order-1)

    σ[0], ω[0]  = 6, 1
    for i=1:order-1
        gi = integrate(Polynomial([0,12//1])*z(i, τ, σ, ω))
        if i%2 == 1
            τ[i-1] = -10//192*gi(1)
            ω[i] = (τ[i-1]*144*Polynomial([2//15, 0, 0, 1//3, 0, -1//5]) 
                      + gi) ÷ (Polynomial([1,0,-1])*Polynomial([0,12]))^2
        else
            σ[i] = -gi(-1)//12
            ω[i] = -σ[i]//6 + (σ[i]*Polynomial([-1,0,1])*
                (Polynomial([-4,0,6])^2-4) + gi + 12*σ[i]) ÷ 
                        (Polynomial([1,0,-1])*Polynomial([0,12]))^2
        end
    end
    τ, σ, ω
end

end
\end{minted}
\caption{Implementation in Julia of the algorithm outlined in
\cref{sec:PolynomailLindstedtPoincare} for the quadratic codimension 2
Bogdanov--Takens normal form \cref{eq:universal_unfolding}}
\label{lst:BTQuadraticHomoclinic}
\end{code}
\begin{table}[h]
    \centering
    \begin{minted}[breaklines,escapeinside=||,mathescape=true,
    numbersep=3pt, gobble=2, frame=lines, fontsize=\small, framesep=2mm]{julia}
    julia> @benchmark BTQuadraticHomoclinic.solve(order=20)
    BenchmarkTools.Trial:
      memory estimate:  107.36 MiB
      allocs estimate:  5384379
      --------------
      minimum time:     382.111 ms (21.56% GC)
      median time:      414.436 ms (24.41% GC)
      mean time:        459.831 ms (26.81% GC)
      maximum time:     577.254 ms (32.91% GC)
      --------------
      samples:          11
      evals/sample:     1
    \end{minted}
    \caption{Benchmark to obtain a 20th-order approximation to $\tau$ in the
    quadratic normal form \cref{eq:universal_unfolding} using the algorithm
    outlined in \cref{sec:PolynomailLindstedtPoincare}. See in particular
    \cref{corollary:quadraticBTsigma_delta_relation} and the algorithm above.}
    \label{tab:benchmark}
\end{table}
\section{Case study of the quadratic codim 2 Bogdanov--Takens normal form}
\label{sec:case_study_BT2}
In this section, we will numerically study the algorithm outlined in
\cref{sec:PolynomailLindstedtPoincare} for the quadratic codimension 2
Bogdanov--Takens normal form \cref{eq:universal_unfolding}.  Since the algorithm
only relies on arithmetic and calculus on polynomials over the field $\mathbb
Q$, see \cref{corllary:rational_coefficients} there is no need to use propriety
software for the implementation. We choose the relative new programming
language Julia~\cite{bezanson2017julia}. Julia natively supports arbitrary
precision rational numbers. We use the package
\mintinline{julia}{Polynomials.jl}~\cite{Polynomials} to handle the
differentiation and integration of polynomials, as well as polynomial division.
Since the programming language Julia starts indexing arrays at $1$, we use the
package \mintinline{julia}{OffsetArrays}~\cite{OffsetArrays} to lower the index
to $0$ to keep the indexing identical. The code is given in \cref{lst:BTQuadraticHomoclinic}.

\begin{figure}
\centering
\includetikz{BTQuadraticNormalFormTimings}
\caption{Log-linear plot of the order $i$ and the time in seconds it took to
compute the coefficients.}
\label{fig:BTQuadraticNormalFormTimings}
\end{figure}

\begin{table}
\label{table:tauCoeffients}
\centering
\def\arraystretch{1.5}
\pgfplotstabletypeset[
    col sep=comma,
    string type,
    every head row/.style={%
        before row=\hline,
        after row=\hline
    },
    every last
    row/.style={after
    row=\hline},
    columns/i/.style={column
        name=$i$,
        column
    type=r},
    columns/tau/.style={column
        name=$\tau_i$,
        column
    type=l}
    ]{\datadir/coefficients_first10.csv}
    \caption{First 20 coefficients of $\tau$}
\end{table}

\begin{figure}[ht!]
\includetikzscaled{BTQuadraticNormalFormApproximations}
\caption{Comparison between the numerical obtained continued homoclinic
    bifurcation curve emanating from the Bogdanov--Takens point in
    \cref{eq:universal_unfolding} with $a=b=1$ and the parameter approximations
    with orders ranging from 10 to 200.  For $\beta_1 \lesssim -8$ the
    approximation starts diverging. This indicates that the radius of
    convergence in the perturbation parameter $\epsilon$ is less than or equal
    to $\sqrt[4]{2}$. Note that for $\beta_1 > -8$ the tenth order is
    already indistinguishable from the numerical obtained parameters.}
\label{fig:BTQuadraticNormalFormApproximations}
\end{figure}

In~\cite{Algaba_2019} it is claimed that one can obtain higher-order
approximations very fast using their algorithm. However, our algorithm, which
should be superior, shows that the order of approximation and the computational
cost are not linear related. We performed a benchmark to obtain a 20th-order
approximation to $\tau$, see \cref{tab:benchmark}.
These results were obtained on the mobile CPU Intel i5-6200U (4) @ 2.800GHz
with 11407MiB of memory. The coefficients for $\tau$ are shown in
\cref{table:tauCoeffients}. We see that the length of the numerator and
denominator increases at each (even) order. This also holds true for the
coefficients of $\sigma$ and the coefficients of the polynomials $\omega$.
Performing operations on these rational numbers become increasingly more
difficult for the computer to deal with. In
\cref{fig:BTQuadraticNormalFormTimings} a log-linear plot is shown, plotting
the time in seconds to solve the $ith$ order equation
\cref{eq:ith_order_equation}. It took nearly 10 hours to obtain the 200th order
approximation of $\tau$. A linear regression on the last 50 data points
indicates the time increases exponentially. Extrapolation yields that it would
take more than 17 years to solve the first 500 terms if memory doesn't become
an issue. Since the algorithm is embarrassingly parallelizable we could speed
up the process. However, exponential growth cannot be escaped.

Next, we would like to make some comments on the radius of convergence of the
asymptotic approximation to the homoclinic orbit in the quadratic
Bogdanov--Takens bifurcation. In~\cite{Algaba_2019} there is the remark that the
higher-order approximation can greatly improve the accuracy of the
approximation for large parameter values. However, this fully depends
on the radius of convergence of the series. To get a first impression, we
compare the parameters computed from the numerical continued solution to the
homoclinic solution in \cref{eq:universal_unfolding} using \MATCONT with the
predicted parameters. \Cref{fig:BTQuadraticNormalFormApproximations} reveals 
a typical situation  in perturbation series. Increasing the order of the
perturbation parameter improves the approximation for small parameters, but for
larger parameters, the approximation becomes much worse. Reproducing~\cite[Fig
3c]{Algaba_2019}, but increasing the order, shows that this solution is outside
the radius of convergence.

