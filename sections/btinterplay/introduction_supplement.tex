\section{Overview}
\label{sm:sec:overview}
This supplement is intended to complement the \hyperref[maintext]{main
text}. As a result, rather than reading it in its entirety, it is advisable to
read the sections as referred to by the main text.

In \cref{app:incorrect_predictor} an explicit example is given in which it is
demonstrated that the parameter-dependent center manifold transformation
provided in~\cite{Al-Hdaibat2016} is incorrect. Additionally, we show how it is
resolved.

A short proof of \Cref{proposition:symmetry} is provided in
\cref{sec:proof:symmetry_proposition}. For completeness, we also repeat the
statement of the proposition here.

Then, the identity in \cref{eq:I_n}, which we were unable to solve with symbolic
software, is worked out in detail in \cref{sec:I_n}.

Next, using the transformations from
\cref{sec:homoclinic_asymptotics_n_dimension}, we analytically show in
\cref{sec:comparison_homoclinic_predictors} that the homoclinic predictor for
the smooth normal form is, up to a phase shift, asymptotically equivalent to
the orbital predictor. Then, by choosing the constants of integration in the
time translation in a specific manner, we show equivalence between the
predictors.

Implementation details of the new homoclinic initializers in \MATCONT are given
in \cref{sec:implementation}. We describe how to extract the initial data from
the derived homoclinic predictors needed for the defining system used in \MATCONT to
continue homoclinic solutions in ordinary differential equations in two
parameters. Additionally, we show how to obtain the correct normalized tangent
vector in order to start continuation of the homoclinic orbits away from the
Bogdanov--Takens point.

In \cref{sec:Hodgking-Huxley} an additional four dimensional system is
analyzed using our predictors. As in the example treated in \cref{sec:examples},
we will show that the approximation order of the homoclinic asymptotic lifts
correctly to the parameter-dependent center manifold, see
\cref{fig:HodgkinHuxleyConvergencePlot}.

Lastly, in \cref{sec:case_study_BT2}, we will use Julia to numerically study
the algorithm outlined in \cref{sec:PolynomailLindstedtPoincare} for the
quadratic codimension 2 Bogdanov--Takens normal form. We will show that
the homoclinic asymptotic has a positive finite convergence radius.
