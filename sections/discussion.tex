\epigraph{ ``Solving intelligence, and then using that to solve everything
else.''}{{\it Demis Hassabis}}
At this stage we have robust predictors for switching to  non-hyperbolic cycles
emanating from generalized Hopf, fold-Hopf, transcrical-Hopf, and Hopf-Hopf
bifurcation points, and also to homoclinic orbits emanating from generic and
transcritical codimension two Bogdanov--Takens bifurcation points in classical
DDEs. The implementation in \DDEBIFTOOL provides a powerful tool to analyze
discrete delay differential equations. The next obvious step is to generalize
\cite{Kuznetsov2005,DeWitte2013,DeWitte2014} by deriving normal forms for
bifurcation of periodic orbits for DDEs. As announced in the introduction, we
already made great progress on this problem and expect to publish our results
soon.

At times, it felt somewhat redundant working with the numerical software packages
\DDEBIFTOOL and \MATCONT. Indeed, a large part of the code to initialize the
homoclinic predictors is identical in both software packages. We therefore
believe it would be beneficial to merge these two separate software packages into
one larger package. At the time, \MATLAB was the obvious programming environment
for those packages to be coded in. It could run on multiple hardware
architectures and operating systems. Additionally, the scripting language makes
is easy for students to participate in the projects, without having to have a
deep understanding of programming. However, in our opinion, Julia would be the
more suitable choice now. Indeed, next to the benefits mentioned above for
\MATLAB, Julia supports multiple-dispatch, is very fast (you don't need to
vectorize your code in order to obtain speed as in MATLAB), it is freely
available, and has a great community and rapidly growing ecosystem. With a focus
on large scale systems, there is already the Julia package
\texttt{BifurcationKit.jl} \cite{veltz:hal-02902346}. We
envision this package to grow to support a graphical user interface, as in
\MATCONT, and to be able to analyze delayed systems as in \DDEBIFTOOL.

A question that remains open, is what the exact radius of convergence of the
homoclinic asymptotic near the generic quadratic codimension two Bogdanov-Takens
bifurcation  point is. Peter de Maesschalck began looking into this and has
already found some promising results. This may then be used to expand the radius
of convergence for example. The same type of problem arises when studying
homocinic and heteroclinic orbits in the degenerated Bogdanov--Takens
bifurcations of Bogdanov--Takens bifurcation with symmetries, see for example
\cite{chow_li_wang_1994}. It would be interesting to study these problems as
well. 

On a related note, we would like to revisit the problem of obtaining homoclinic
asymptotics near the zero-Hopf point. It was all but trivial to locate the
homoclinic order in the chaotic bi-antisymmetric tensor model.

Lastly, the rapid growth of artificial intelligence seems to be able to solve
more and more problems every day. This also applies to the field of mathematics,
see for example \cite{Davies2021}, and \cite{631606893d774be2a2c919789d14b2d6}.
We therefore wonder --- ``Would it be possible to obtain a homoclinic predictor
near the generic codimension two Bogdanov--Takens bifurcation in infinite
dimensional systems generated by DDEs training a neural network?''.
